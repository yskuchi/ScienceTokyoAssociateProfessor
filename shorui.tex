\documentclass[11pt,a4paper,uplatex,dvipdfmx]{ujarticle} 		% for uplatex
%\documentclass[11pt,a4j,dvipdfmx]{jarticle} 					% for platex
%\documentclass[11pt,pdflatex,a4paper,ja=standard]{bxjsarticle}
\usepackage{amsmath}  
%=======================================
% form00_header.tex
%	General header for kakenhiLaTeX,  Moved over from form00_2010_header.tex.
%	2009-09-06 Taku Yamanaka (Osaka Univ.)
%==== General Version History ======================================
% 2006-05-30 Taku Yamanaka (Physics Dept., Osaka Univ.)
% 2006-06-02 V1.0
% 2006-06-14 V1.1 Use automatic calculation for cost tables.
% 2006-06-18 V1.2 Split user's contents and the format.
% 2006-06-20 V1.3 Reorganized user and format files
% 2006-06-25 V1.4 Readjusted all the table column widths with p{...}.
%				With \KLTabR and \KLTabRNum, now the items can be right-justified
%				in the cell defined by p{...}.
% 2006-06-26 V1.5 Use \newlength and \setlength, instead of \newcommand, to define positions.
% 2006-08-19 V1.6 Remade it for 2007 JFY version.
% 2006-09-05 V1.7 Added font declarations suggested by Hoshino@Meisei Univ.
% 2006-09-06 V1.8 Introduced usePDFform flag to switch the form file format.
% 2006-09-09 V1.9 Changed p.7, to allow different heights between years. (Thanks to Ytow.)
% 2006-09-11 V2.0 Added an option to show budget summary.
% 2006-09-13 V2.1 Added an option to show the group.
% 2006-09-14 V2.1.1 Cleaned up Kenkyush Chosho.
% 2006-09-21 V2.2 Generated under a new automatic development system.

% 2007-03-24 V3.0 Switched to a method using "picture" environment.

% 2007-08-14 V3.1 Switched to kakenhi3.sty.
% 2007-09-17 V3.2 Added \KLMaxYearCount
% 2008-03-08 V3.3 Remade it for 2009 JFY version\
% 2008-09-08 V3.4 Added \KLXf ... \KLXh.
% 2011-10-20 V5.0 Use kakenhi5.sty, to utilize array package in tabular environment.
% 2012-08-14 v5.1 Moved preamble and kakenhi5 into the current directory, instead of the parent directory.
% 2012-11-10 v6.0 Switched to kakenhi6.sty.
% 2015-08-26 v6.1 Added KLFirstPageIsLongPage flag.
% 2017-05-27 v7.0 Simplified for the new format.
%=======================================
% Dummy section and subsection commands.
% With these, some editors (such as TeXShop, etc.) can jump to the (sub)sections.
\newcommand{\dummy}{dummy}% 
%\renewcommand{\section}[1]{\renewcommand{\dummy}{#1}}

\usepackage{calc}
\usepackage{geometry}                % See geometry.pdf to learn the layout options. There are lots.
\usepackage[dvipdfmx]{graphicx}
\usepackage{color}
\usepackage{ifthen}
\usepackage{udline}
\usepackage{array}
\usepackage{longtable}
\usepackage{fancyhdr}
 % pieces
%==================================================
% kakenhi7.sty
%==================================================
% v1
% Minimum amount of macros for writing Kakenhi forms.
%
% 2005-10-24 Taku Yamanaka, Physics Dept. Osaka Univ.
%		taku@hep.sci.osaka-u.ac.jp
% 		Macros such as XYBC, etc. were imported from Kakenhi Macro at
% 		http://www.yukawa.kyoto-u.ac.jp/contents/researcher/kakenhi.html .
% 2006-06-04 Taku
%		Added macros to draw boxes if \DrawBox is in the source.
%		This is useful when designing the LaTeX forms.
% 2006-06-14 Taku
%		Added LaTeX macros to add costs 
%		(\KLResetGrandSum, \KLCostItem, \KLSum, \KLGrandSum).
%		Added a macro \Number to supply commas every 3 digits (imported from kkh.mac).
% 2006-06-25 Taku
%		Added \KLTabC, \KLTabR, \KLTabRNum to specify alignments in tables.
%		Please note that \phantom{...} is required for the column, 
%		or otherwise somehow p{...mm} is ignored.
% 2006-08-13 Taku
%		Added \KLItemNumUnitCost . This requires calc.sty.
% 2006-09-06 Taku
%		Added KLGrandTotalValue to add ALL the costs.
%		Also added \KLPrintGrandTotal to print the total on the console.
% 2006-09-09 Taku
%		Added macros to handle "efforts".
% 2006-09-10 Taku
%		Added \KLnewcounter to make series of counters.
%		Modified \KLResetGrandSum and \KLSum to add the sum for 
%		each category and year.
% 2006-09-11 Taku
%		Added \NumC to display LaTeX counter with commas.
% 2006-09-12 Taku
%		Added simple macros to make group table.
% 2006-09-23 Taku
%		Added the 'Fair License' notification.
% 2006-10-22 Taku
%		Initialize KLNumPeople to -1, so that the first header row will not be included in the count.
%====================================================
% v2
% 2007-03-24 Taku
%		Instead of using Kakenhi Macros to position items, 
%		switched to a new method using
%		"picture" environment.  The origin of the coordinate is set to 
%		the lower left corner of the paper.  The positions are given in "points", 
%		as can be read by gv. These methods were suggested by 
%		Tsutomu Sakurai at Saitama Univ..
% 2007-03-30 Taku
%		The sum of each category and year is made using 
%		macros \KLItemCost, etc., instead of \KLSum.  This is a step toward
%		automatically aligning category sums in the same year, in some forms.
% 2007-04-02 Taku
%		Added \KLBudgetMiniTabular, \KLMiniSum, etc. to handle
%		budget tables with multiple category columns.
% 2007-05-04 Taku
%		Added \multicolumnDottedLine .
%==================================================
% v3 
% 2007-08-14 Taku
%		Simplified page handling, by introducing \KLBeginSinglePage,
%		\KLPageRange, etc..
% 2007-09-01 Taku
%		Added a new command, \KLItemNumUnitCostLocation , 
%		and \KLAddCost to clean things up.
% 2007-09-06 Taku
%		Set \KLEven/OddLeft/RightEdge parameters in \KLWaterMark.
%		Without it, if \KLLeftEdge or \KLRightEdge is used inside watermark,
%		it generated a very obscure error message, which was hard to track down.
% 2007-09-09 Taku
%		Removed clearring \thiswatermark in \KLClearWaterMarks.
% 2007-09-12 Taku
%		Added \KLPriorityItemNumUnitCostTwo for Tokusui.
% 2007-09-14 Taku
%		Added \KLItemNumUnitCostTwo for tokutei_koubo.
% 2007-09-17 Taku
%		In \KLAddCost, costs are added only if it is within the defined year range.
% 2008-09-02 Taku
%		  Added \KLMonthPriorityItemNumUnitCostTwo for tokutei_keizoku.
% 2008-09-07 Taku
%		   Use \KLJFY to print year in budget tables.
% 2008-10-21 Taku
%			Added KLItemNumUnitCostInParen for shorei.
% 2009-09-03 Taku
%			Added \dottedLine .
%==================================================
% v4
% 2009-09-06 Taku
%			Added macros for partial typesetting.
% 2009-09-12 Taku
%			Added macros for showing coordinates and edges.
% 2009-09-13 Taku
%			Added macros to show boxes and minipage frames and their corner coordinates.
% 2010-03-04 Taku
%			Moved macros for calculating lengths and positions from form07_header.tex to here.
% 2010-04-11 Taku
%			Added \KLItemCostOne for jisedai, and necessary flags to print budget sums before 
%			the detailed budget table.
%==================================================
% v5
% 2011-10-20 Taku
%			By using array package within tabular environment, the following macros were simplified:
%			\KLTabC, \KLTabR, \KLBudgetMiniTabular, 
%			New Macros:
%			\KLCC, \KLCR
% 2011-10-24 Taku
%			Removed using \KLTabR from most of the budget tables.
% 2011-10-26 Taku
%			Added KLMyBudget.
%			Modified KLYearItemNumUnitCostTwo to just put the JFY if the second item is blank.  (For Kiban S)
% 2012-03-10 Taku
%			Changed the tabcolsep for \KLMyBudget to 0pt.
% 2012-08-14 Taku
%			Moved xxx_forms_pdf and _eps directories to under mother.
% 2012-09-09	Taku
%			Added \KLbibitem(B) and \KLcite(B).
% 2012-09-16	Taku
%			Added \KLOtherApplication, \KLOtherApplicationReasons, and \KLOtherFundReasons
%			for tokusui (and maybe for others in the future).
%==================================================
% v6
% 2012-11-10	Taku
% 			Modified \KLOtherApplicationReasons and \KLOtherFundReasons to make the tables compact.
%			These were in hook3.tex for the 2013 version.
%			Added \KLOtherApplicationDiff for many shumokus.
%			Removed \KLbibitemB and \KLciteB.
%			Changed \KLbibitem to use a dedicated column for numbering.
% 2012-11-11	Taku
%			Added \KLItemSetCostLocationInfo and \KLItemCostInfo.
% 2013-09-19	Taku
%			Added \KLOtherPD and \KLOtherPDShort to enter JSPS PD for other funds.
% 2013-10-02	Taku
%			Changed \KLbibItem, not to use a dedicated column for numbering, 
%			because otherwise the label defined in \label{...} cannot be used in @currentlabel.
% 2014-09-22	Taku
%			Added \KLCL for filling narrow tabular cells in English.  
%			(Suggested by Frank Bennett.)
% 2014-11-08	Taku
%			Added an instruction in \KLCheckPageLimit.
% 2015-08-23	Taku
%			Added NumCk to show numbers divided by 1000 (truncated).
% 2015-08-24	Taku
%			Introduced \KLLongPage and \KLSimpleLongPage to offer floating environment in 
%			single-page-frames.
%=====================================================
% v7
%	Frames are gone!  Simplify kakenhiLaTeX to benefit from 
%	the new style.
% 2017-05-03	Taku
%			Added \KLAnotherFund .
% 2017-05-27	Taku
%			Made \KLBeginSubject and \KLEndSubject to handle new 
%			mother file style.
% 2017-08-17 Taku
%	Added \KLItemCostNoYear and \KLEndBudgetNoYear for kokusai_kyoudou.
% 2017-08-19	Taku
%			Updated \KLBeginSubject and \KLEndSubject to handle 
%			various headers.
% 2017-08-20	Taku
%			\KLBeginSubject calls \KLFirstPageStyle and \KLDefaultPageStyle
%			which should be defined for each shumoku (or JSPS/MEXT).
%			This is to pass the subject name etc. to the header.
% 2017-08-27	Taku
%			Set section number in \KLBeginSubject.
% 2017-08-29	Taku
%			Moved over KLShumokuFirstPageStyle and KLShumokuDefaultPageStyle.
%			Added jsps-abs-p1-header, jps-abs-subject-header, and 
%			jsps-abs-default-header as arguments to \KLShumoku***Header.
% 2017-09-02	Taku
%			Added \KLBeginSubjectWithHeaderCommands for more flexible header style.
% 2017-09-03	Taku
%			Added \vspace*{-4mm} after \includegraphics in \KLBeginSubject*.
% 2017-09-05	Taku
%			Removed now-the-old commands.
%=====================================================

%=====================================================
% Macro to supply commas every 3 digits (up to 9 digits)
%	Imported from kkh.mac for Kakenhi Macro.
%=====================================================
%
\newif\ifNumWithCommas \NumWithCommastrue
\def\NumWithCommas{\NumWithCommastrue}
\def\NumWithoutCommas{\NumWithCommasfalse}
\newcount\Numa
\newcount\Numb
\def\Numempty{}%output blank if "-0" is given
\def\Number#1{\edef\Numpar{#1}\ifx\Numempty\Numpar\else%
\ifNumWithCommas\Numa=#1\relax
\ifnum\Numa>999999\divide\Numa by 1000000
\number\Numa,%
\multiply\Numa by -1000000\advance\Numa by #1\relax
\Numb=\Numa\divide\Numa by 1000
\ifnum\Numa<100 \ifnum\Numa<10 0\fi0\fi\number\Numa,%
\multiply\Numa by -1000\advance\Numa by \Numb
\ifnum\Numa<100 \ifnum\Numa<10 0\fi0\fi\number\Numa%
\else\ifnum\Numa>999\divide\Numa by 1000
\number\Numa,%
\multiply\Numa by -1000\advance\Numa by #1\relax
\ifnum\Numa<100 \ifnum\Numa<10 0\fi0\fi\number\Numa%
\else\number\Numa\fi\fi\else\number#1\fi\fi}

%======================================================
% Macro to display LaTeX counter with commas every 3 digits.
%======================================================
\newcommand{\NumC}[1]{\Number{\value{#1}}}

\newcounter{kyen}
\newcommand{\NumCk}[1]{%
	\setcounter{kyen}{\arabic{#1}/1000}
	\Number{\value{kyen}}
}

%======================================================
% Macros to align (right-justify, center) elements within a tabular cell
% whose width is defined by p{...}.
% 2006-06-25 Taku
% 	These are necessary, because the cell width should be given explicitly
% 	by p{...mm} to match the given table in a tabular environment.  
% 	One could allocate a width with \phantom{...},
% 	but it is a little tricky, since it depends on the font size.
%======================================================

%---------------------------------------------------------------------
% Center text within a tabular cell allocated by p{...}
%\newcommand{\KLTabC}[1]{\multicolumn{1}{c}{#1}}
\newcommand{\KLTabC}[1]{\centering\arraybackslash#1}
% This new method does not require a dummy table row to put them in correct columns.
%
% This should be used in tabular definition, as:
%	\begin{tabular}[t]{>{\KLCC}p{30pt}p{50pt}}
\newcommand{\KLCC}{\centering\arraybackslash}

%---------------------------------------------------------------------
% Right justify text within a tabular cell allocated  by p{...}
%\newcommand{\KLTabR}[1]{\multicolumn{1}{r@{\ }}{#1}}
\newcommand{\KLTabR}[1]{\raggedleft\arraybackslash#1}

% This should be used in tabular definition, as:
%	\begin{tabular}[t]{>{\KLCR}p{30pt}p{50pt}}
\newcommand{\KLCR}{\raggedleft\arraybackslash}%

%---------------------------------------------------------------------
% Right justify number (with comma every 3 digits) 
% within a tabular cell allocated by p{...}
\newcommand{\KLTabRNum}[1]{\KLTabR{\Number{#1}}}

%---------------------------------------------------------------------
% Left justify text within a tabular cell allocated by p{...}
% This should be used in tabular definition, as:
%	\begin{tabular}[t]{>{\KLCL}p{30pt}p{50pt}}
\newcommand{\KLCL}{\raggedright\arraybackslash}%

%=================================================
%  counter tools
%=================================================
\newcounter{KLtmp}

%------------------------------------------------------------------------------
% Makes a set of counters, with prefix #1, followed by 
% suffix ranging from 0 to #2 - 1.
% For example, \KLnewcounter{mine}{3} makes counters
% mine0, mine1, and mine2 .
%-------------------------------------------------------------------------------
\newcommand{\KLnewcounter}[2]{
	\setcounter{KLtmp}{0}
	
	\whiledo{\value{KLtmp} < #2}{
		\newcounter{#1\arabic{KLtmp}}
		\stepcounter{KLtmp}
	}
}

%------------------------------------------------------------------------------
% Dumps the contents of the counters.
%------------------------------------------------------------------------------
\newcommand{\KLdumpcounter}[2]{
	\setcounter{KLtmp}{0}
	
	\whiledo{\value{KLtmp} < #2}{
		#1\arabic{KLtmp} : \arabic{#1\arabic{KLtmp}}\\
		\stepcounter{KLtmp}
	}
}

%=======================================================
% LaTeX macros to add costs.
%	2006-06-14 Taku Yamanaka
%=======================================================
\newcounter{KLCost}				% to calculate cost = #units x unit cost
\newcounter{KLGrandTotalValue}		% for the grand total of all the categories in all years
\setcounter{KLGrandTotalValue}{0}

\newcommand{\KLCostCategory}{KLequipments}
\newcounter{KLYearCount}
\newcounter{KLPrintYear}

% Make counters for annual sums for each category-----------------------
\newcommand{\KLMaxYear}{8}
\KLnewcounter{KLequipments}{\KLMaxYear}
\KLnewcounter{KLexpendables}{\KLMaxYear}
\KLnewcounter{KLdomestic}{\KLMaxYear}
\KLnewcounter{KLforeign}{\KLMaxYear}
\KLnewcounter{KLtravel}{\KLMaxYear}
\KLnewcounter{KLgratitude}{\KLMaxYear}
\KLnewcounter{KLmisc}{\KLMaxYear}
\KLnewcounter{KLAnnualSum}{\KLMaxYear}

%------------------------------------------------
% Add up the given cost to category-year sum, category sum, year-sum, and total.
% 2007-09-01 Taku
% 2007-09-17 Taku: Add costs only if it is within the defined year range.
%------------------------------------------------
\newcommand{\KLAddCost}[1]{%
	\ifthenelse{\value{KLYearCount} > \value{KLMaxYearCount}}{%
		%pass
	}{%
		\addtocounter{\KLCostCategory0}{#1}%
		\addtocounter{\KLCostCategory\arabic{KLYearCount}}{#1}%
		\addtocounter{KLAnnualSum\arabic{KLYearCount}}{#1}%
		\addtocounter{KLAnnualSum0}{#1}%
		\ifthenelse{\equal{\KLCostCategory}{KLdomestic}}{%
			\addtocounter{KLtravel0}{#1}%
			\addtocounter{KLtravel\arabic{KLYearCount}}{#1}%
		}{}%
		\ifthenelse{\equal{\KLCostCategory}{KLforeign}}{%
			\addtocounter{KLtravel0}{#1}%
			\addtocounter{KLtravel\arabic{KLYearCount}}{#1}%
		}{}%
	}%
}


\newcommand{\KLClearWaterMarks}{%
	%--empty watermarks
	\watermark{}
%	\thiswatermark{}
	\rightwatermark{}
	\leftwatermark{}
}

\newcommand{\KLInput}[1]{%	The macros defined inside the file are only valid within the file.
	\begingroup
	\input{#1}
	\endgroup
}

%================================
% For 2017 new style without frames
%================================
\newcommand{\KLShumokuFirstPageStyle}[5]{%
%	Defines the header for the first page.
%	Called from \KLBeginSubject.
%--------------------------------
%	#1: page style name
%	#2: 様式
%	#3: 研究種目名
%	#4: 項目名
%	#5: sectionNo
%--------------------------------
	\ifthenelse{\equal{#1}{jsps-p1-header}}{%
		\JSPSVeryFirstPageStyle{#1}{#2}{#3}{#4}{#5}
	}{%
		\ifthenelse{\equal{#1}{jsps-abs-p1-header}}{%
			\JSPSVeryFirstPageStyle{#1}{#2}{#3 概要}{#4}{#5}
		}{%
            		\ifthenelse{\equal{#1}{jsps-subject-header}}{%
            			\JSPSFirstSubjectPageStyle{#1}{#2}{#3}{#4}{#5}
            		}{%
				\ifthenelse{\equal{#1}{jsps-abs-subject-header}}{%
            				\JSPSFirstSubjectPageStyle{#1}{#2}{#3 概要}{#4}{#5}
				}{%
                    			\thispagestyle{#1}
				}
            		}
		}
	}
}

\newcommand{\KLShumokuDefaultPageStyle}[5]{%
%	Defines the default header.
%	Called from \KLBeginSubject.
%--------------------------------
%	#1: page style name
%	#2: 様式
%	#3: 研究種目名
%	#4: 項目名
%	#5: sectionNo
%--------------------------------
	\ifthenelse{\equal{#1}{jsps-default-header}}{%
		\JSPSDefaultPageStyle{#1}{#2}{#3}{#4}{#5}
	}{%
		\ifthenelse{\equal{#1}{jsps-abs-default-header}}{%
			\JSPSDefaultPageStyle{#1}{#2}{#3 概要}{#4}{#5}
		}{%
            		\pagestyle{#1}
		}
	}
}

\newcommand{\KLSubjectName}{}
\newcommand{\KLSubjectMaxPages}{}
\newcommand{\KLSubjectEndPage}{}
\newcounter{KLSubjectEndPage}
\setcounter{KLSubjectEndPage}{0}

\newcommand{\KLSubjectCheckNPages}{%
%	\arabic{page}, \arabic{KLSubjectEndPage}\\
	\ifthenelse{\value{page}>\value{KLSubjectEndPage}}{
		{\LARGE「\KLSubjectName」は \KLSubjectMaxPages\ ページ以内で書いてください。}
		\clearpage
	}{%
	}
}

\newcommand{\KLSubjectAdvancePages}{%
	\renewcommand{\KLSubjectEndPage}{\value{KLSubjectEndPage}}
	\ifthenelse{\value{page}<\KLSubjectEndPage}{%
		\phantom{x}\clearpage
	}{}
	% Advance page if necessary
	\ifthenelse{\value{page}<\KLSubjectEndPage}{%
		\phantom{x}\clearpage
	}{}
	% Advance page if necessary
	\ifthenelse{\value{page}<\KLSubjectEndPage}{%
		\phantom{x}\clearpage
	}{}
	% Advance page if necessary
	\ifthenelse{\value{page}<\KLSubjectEndPage}{%
		\phantom{x}\clearpage
	}{}
	% Advance page if necessary
	\ifthenelse{\value{page}<\KLSubjectEndPage}{%
		\phantom{x}\clearpage
	}{}
}	

\newcommand{\KLJInt}[1]{%
% Returns full-width numerical character.
	\ifthenelse{\equal{#1}{1}}{1}{%
	\ifthenelse{\equal{#1}{2}}{2}{%
	\ifthenelse{\equal{#1}{3}}{3}{%
	\ifthenelse{\equal{#1}{4}}{4}{%
	\ifthenelse{\equal{#1}{5}}{5}{%
	\ifthenelse{\equal{#1}{6}}{6}{%
	\ifthenelse{\equal{#1}{7}}{7}{%
	\ifthenelse{\equal{#1}{8}}{8}{%
	\ifthenelse{\equal{#1}{9}}{9}{%
	\ifthenelse{\equal{#1}{10}}{10}{%
	\ifthenelse{\equal{#1}{11}}{11}{%
	\ifthenelse{\equal{#1}{12}}{12}{%
	#1}}}}}}}}}}}}%
}


\newcommand{\KLBeginSubject}[8]{%
%----------------------------------------------------
%	#1: subjectNo
%	#2: sectionNo
%	#3: sectionJ
%	#4: maxPages
%	#5: pageLengthStyle ('V' for variable, 'F' for fixed)
%	#6: pageCounter (set page counter to this value if the argument exists.
%	#7: subjectFirstPageHeader (header for the first page)
%	#8: defaultPageHeader
%----------------------------------------------------
	\setcounter{section}{#2}
	\setcounter{subsection}{0}
	\setcounter{subsubsection}{0}
	\renewcommand{\KLSubjectName}{#3}
	\renewcommand{\KLSubjectMaxPages}{#4}
	
	\ifthenelse{\equal{#6}{}}{%
	}{%
		\setcounter{page}{#6}
	}
	
	\setcounter{KLSubjectEndPage}{\value{page}}
	\addtocounter{KLSubjectEndPage}{#4}
	
	\ifthenelse{\equal{#7}{}}{%
		% pass
	}{%
		\KLShumokuFirstPageStyle{#7}{\様式}{\研究種目header}{#3}{#2}
	}
	
	\ifthenelse{\equal{#8}{}}{%
		% pass
	}{%
		\KLShumokuDefaultPageStyle{#8}{\様式}{\研究種目header}{#3}{#2}
	}
	
%	\noindent
%	\includegraphics[width=\linewidth]{subject_headers/\KLYoshiki_#1.pdf}\\
%	\vspace*{-4mm}
}

\newcommand{\KLNullHeader}[5]{}
% Dummy command for No header.
% This was introduced to avoid error caused in statement \ifthenelse{\equal{#8}{}} .

\newcommand{\KLBeginSubjectWithHeaderCommands}[8]{%
%----------------------------------------------------
%	#1: subjectNo
%	#2: sectionNo
%	#3: sectionJ
%	#4: maxPages
%	#5: pageLengthStyle ('V' for variable, 'F' for fixed)
%	#6: pageCounter (set page counter to this value if the argument exists.
%	#7: LaTeX command for subjectFirstPageHeader (header for the first page)
%	#8: LaTeX command for defaultPageHeader
%----------------------------------------------------
	\setcounter{section}{#2}
	\setcounter{subsection}{0}
	\setcounter{subsubsection}{0}
	\renewcommand{\KLSubjectName}{#3}
	\renewcommand{\KLSubjectMaxPages}{#4}
	
	\ifthenelse{\equal{#6}{}}{%
	}{%
		\setcounter{page}{#6}
	}
	
	\setcounter{KLSubjectEndPage}{\value{page}}
	\addtocounter{KLSubjectEndPage}{#4}
%	
%	#7{#7}{\様式}{\研究種目header}{#3}{#2}
%	#8{#8}{\様式}{\研究種目header}{#3}{#2}
%	
%	\noindent
%	\includegraphics[width=\linewidth]{subject_headers/\KLYoshiki_#1.pdf}\\
%	\vspace*{-4mm}
}

\newcommand{\KLEndSubject}[1]{%
%	#1: pageLengthStyle ('V' for variable, 'F' for fixed)
		\clearpage % This should be done to update page counter for checking.
		\KLSubjectCheckNPages
		\ifthenelse{\equal{#1}{F}}{%
			\KLSubjectAdvancePages
		}{%
		}
}

%==================================================
% Miscellaneous macros
%==================================================

%----------------------------------------------------------------------
% Draw dotted lines across a multiple column table
%----------------------------------------------------------------------
\newcommand{\multicolumnDottedLine}[1]{%
%	\multicolumn{#1}{@{\hspace{-2mm}}c}{\dotfill}\\%
	\multicolumn{#1}{@{}c}{\dotfill}\\%
}

\newcommand{\dottedLine}{%
	\\\noindent
	\dotfill\\
}

%----------------------------------------------------------------------
% Solid line
%----------------------------------------------------------------------
\newlength{\KLLineLength}
\newcommand{\solidLine}[1]{
%----------- keep an empty line between here and \noindent so that it works after normal text and list.

	\noindent
	\hspace*{-10pt}
	\rule[10pt]{\textwidth}{#1}% #1 = 0.5pt, ....
	\vspace*{-10pt}
}

\newcommand{\KLLine}{%
	\solidLine{1pt}
}

%----------------------------------------------------------------------
% publication list (Thanks to Tetsuo Iwakuma [bulletin board #876])
%----------------------------------------------------------------------
\newcounter{KLBibCounter}

\makeatletter	
	\newcommand{\KLbibitem}{%
		\stepcounter{KLBibCounter}%
		\let \@currentlabel \theKLBibCounter
		\arabic{KLBibCounter}. %
	}
\makeatother

\newcommand{\KLcite}[1]{[\ref{#1}]}

%==================================================
%Fair License

%<Copyright Information>

%Usage of the works is permitted provided that this
%instrument is retained with the works, so that any entity
%that uses the works is notified of this instrument.

%DISCLAIMER: THE WORKS ARE WITHOUT WARRANTY.

%[2004, Fair License: rhid.com/fair]
%==================================================
% You may edit/modify this package at your own risk.
% If there are important fixes or changes that you think should be 
% reflected in the standard distribution, please notify:
%	taku@hep.sci.osaka-u.ac.jp  .
%==================================================
 % pieces
% form01_header.tex
% 2017-05-28 Split from form00_header.tex to move \input{kakenhiLaTeX7.sty} to mother_1.tex.
% ===== Parameters for KL (Kakenhi LaTeX) ========================
%%\geometry{noheadfoot,scale=1}  %scale=1 resets margins to 0
\setlength{\unitlength}{1pt}

\newlength{\KLCella}
\newlength{\KLCellb}
\newlength{\KLCellc}
\newlength{\KLCelld}
\newlength{\KLCelle}
\newlength{\KLCellf}

\newcounter{KLMaxYearCount}	% # of years for the proposal
\newcommand{\KLCLLang}{}	% language-dependent left-justification in tabular

% ===== format and header =========
\setlength{\oddsidemargin}{-8pt}
\setlength{\evensidemargin}{-8pt}
\setlength{\textwidth}{466pt}
\setlength{\topmargin}{-61pt}
\setlength{\textheight}{256mm}

\setlength{\headheight}{48pt}
\setlength{\headsep}{3pt}

\cfoot{}
\renewcommand{\headrulewidth}{0pt}

\pagestyle{empty}
% ==== other applications table =========
\newcommand{\KLTableHeaderFont}{\fontsize{8.2}{11}\selectfont}
\newcommand{\KLTableHeaderSmallFont}{\fontsize{7.5}{10}\selectfont}
\newcommand{\KLTableHeaderSmallerFont}{\fontsize{7}{10}\selectfont}

 % pieces
%% ===== Global year-dependent definitions for the Kakenhi form ===========
% 基本情報
\newcommand{\研究開始年度}{2020}
\newcommand{\研究開始元号年度}{02}	%令和

\newcommand{\一年目西暦}{2020}
%\newcommand{\2年目西暦}{2021}
%\newcommand{\3年目西暦}{2022}
%\newcommand{\4年目西暦}{2023}
%\newcommand{\5年目西暦}{2024}
%\newcommand{\6年目西暦}{2025}

\newcommand{\一年目}{2}
%\newcommand{\2年目}{3}
%\newcommand{\3年目}{4}
%\newcommand{\4年目}{5}
%\newcommand{\5年目}{6}
%\newcommand{\6年目}{7}

\newcommand{\一年目J}{2}
%\newcommand{\2年目J}{3}
%\newcommand{\3年目J}{4}
%\newcommand{\4年目J}{5}
%\newcommand{\5年目J}{6}
%\newcommand{\6年目J}{7}


 % pieces
%% hook3: after including packages ===================
 % pieces
%#Name: kiban_a
% form04_jsps_headers.tex
% 2017-08-20 Taku
% 2017-08-29 Taku
%			Added a check against jsps-abs-p1-header.
% 2017-09-02 Taku
%			Added sectionNo to the commands to make them compatible with 
%			\KLBeginSubjectWithHeaderCommands.
%			Use \KLJInt.
% 2018-09-01 Taku
%			Adjusted the heights of the headers by inserting \vspace{-3pt} and \rule.
%
\newcommand{\headerfont}{\fontsize{11}{11}\selectfont}
% ===== Headers =====================================
\newcommand{\JSPSVeryFirstPageStyle}[5]{%
%	Defines the header for the very first page of the form.
%	Called from \KLShumokuFirstPageStyle in form04_***.
%--------------------------------
%	#1: page style name
%	#2: 様式
%	#3: 研究種目名
%	#4: 項目名
%	#5: sectionNo
%--------------------------------
	\fancypagestyle{JSPSVeryFirstPageStyle}{% The name is not taken from #1, because 
		\fancyhf{}
		\fancyhead[L]{\hspace{-37pt}\headerfont#2\ \\
				\rule{0pt}{18pt}\\}
%				\rule{0pt}{0pt}\\}
		\fancyhead[R]{\headerfont\textbf{#3\ \KLJInt{\thepage}}\vspace{-5pt}\\
			\rule{0pt}{0pt}\\}
%		\fancyhead[R]{\headerfont\textbf{#3\ \KLJInt{\thepage}\\}}
	}
	\thispagestyle{JSPSVeryFirstPageStyle}
}

\newcommand{\JSPSFirstSubjectPageStyle}[5]{%
%	Defines the header for the first page for the subject.
%	Called from \KLShumokuFirstPageStyle in form04_***.
%--------------------------------
%	#1: page style name
%	#2: 様式
%	#3: 研究種目名
%	#4: 項目名
%	#5: sectionNo
%--------------------------------
	\fancypagestyle{JSPSFirstSubjectPageStyle}{%
		\fancyhf{}
		\fancyhead[R]{\headerfont\textbf{#3\ \KLJInt{\thepage}}\vspace{-5pt}\\
			\rule{0pt}{0pt}\\}
%		\fancyhead[R]{\headerfont\textbf{#3\ \KLJInt{\thepage}\\}}
	}
	\thispagestyle{JSPSFirstSubjectPageStyle}
}

\newcommand{\JSPSDefaultPageStyle}[5]{%
%	Defines the default header for the subject.
%	Called from \KLShumokuDefaultPageStyle in form04_***.
%--------------------------------
%	#1: page style name
%	#2: 様式
%	#3: 研究種目名
%	#4: 項目名
%	#5: sectionNo
%--------------------------------
	\fancypagestyle{JSPSDefaultPageStyle}{%
		\fancyhf{}
%		\fancyhead[L]{\headerfont\textbf{【#4(つづき)\ 】}\vspace{-7pt}\\}
		\fancyhead[R]{\headerfont\textbf{#3\ \KLJInt{\thepage}}\vspace{-5pt}\\
			\rule{0pt}{0pt}\\}
%		\fancyhead[R]{\headerfont\textbf{#3\ \KLJInt{\thepage}\\}}	
        }
        \pagestyle{JSPSDefaultPageStyle}
}

 % pieces
% form04_kiban_a_header.tex

% ===== Global definitions for the Kakenhi form ======================
% 基本情報
\newcommand{\様式}{}
\newcommand{\研究種目}{素核研22-12}
\newcommand{\研究種目後半}{}
\newcommand{\研究種別}{}
\newcommand{\研究種目header}{\研究種目 応募者氏名 \研究代表者氏名 }

\newcommand{\KLMainFile}{kiban\_a.tex}
\newcommand{\KLYoshiki}{kiban_a_header}

%==========================================================
 % pieces
% ===== Global definitions for the Kakenhi form ======================
% 基本情報
%
%------ 研究課題名  -------------------------------------------
\newcommand{\研究課題名}{象の卵}

%----- 研究機関名と研究代表者の氏名-----------------------
\newcommand{\研究機関名}{東京大学}
\newcommand{\研究代表者氏名}{内山雄祐}
\newcommand{\me}{\underline{Y.~Uchiyama}} 
%---- 研究期間の最終年度 ----------------
\newcommand{\研究期間の最終元号年度}{6}  %令和で,半角数字のみ
%========================================

%% inst_general.tex
%--------------------------------------------------------------------
% For writing instructions
%--------------------------------------------------------------------
\newcommand{\KLInstWOGeneral}[1]{%
	\noindent
 ーー ※留意事項 ーーーーーーーーーーーーーーーーーーーーーーーーーーーーーーーーー\\
		#1\\
 ーーーーーーーーーーーーーーーーーーーーーーーーーーーーーーーーーーーーーーーーーー
}

\newcommand{\KLInst}[1]{%
	\noindent
	\ifthenelse{\equal{#1}{}}{%
 ーー ※留意事項 ーーーーーーーーーーーーーーーーーーーーーーーーーーーーーーーーーー\\
	}{%
 ーー ※留意事項\textcircled{1} ーーーーーーーーーーーーーーーーーーーーーーーーーーーーーーーーー\\
		#1\\
		
	\noindent
 ーー ※留意事項\textcircled{2} ーーーーーーーーーーーーーーーーーーーーーーーーーーーーーーーーー\\
	}
}

\newcommand{\GeneralInstructions}{%
  1.作成に当たっては、研究計画調書作成・記入要領を必ず確認すること。\\
  2.本文全体は11ポイント以上の大きさの文字等を使用すること。\\
  3.各頁の上部のタイトルと指示書きは動かさないこと。\\
  4. 指示書きで定められた頁数は超えないこと。なお、空白の頁が生じても削除しないこと。\\
  \textcolor{red}{5.本留意事項は、研究計画調書の作成時には削除すること。(\texttt{\textbackslash JSPSInstructions}を消す)}\\
 ーーーーーーーーーーーーーーーーーーーーーーーーーーーーーーーーーーーーーーーーーー
}

\newcommand{\PapersInstructions}{%
 ーー ※留意事項 ーーーーーーーーーーーーーーーーーーーーーーーーーーーーーーーーーー\\
1. 研究業績(論文、著書、産業財産権、招待講演等)は、網羅的に記載するのではなく、\\
 本研究計画の実行可能性を説明する上で、その根拠となる文献等の主要なものを適宜記\\
 載すること。\\
2. 研究業績の記述に当たっては、当該研究業績を同定するに十分な情報を記載すること。\\
 例として、学術論文の場合は論文名、著者名、掲載誌名、巻号や頁等、発表年(西暦)、\\
 著書の場合はその書誌情報、など。\\
3. 論文は、既に掲載されているもの又は掲載が確定しているものに限って記載すること。\\
\textcolor{red}{4. 本留意事項は、研究計画調書の作成時には削除すること。
 (\texttt{\textbackslash PapersInstructions}を消す)}\\
 ーーーーーーーーーーーーーーーーーーーーーーーーーーーーーーーーーーーーーーーーーー\\
} % pieces
%%inst_kiban_a.tex
\newcommand{\JSPSInstructions}{%
	\\
	\KLInst{
  1.基盤研究(A)は審査区分表の中区分により、広い分野の委員構成で多角的視点から\\
    審査が行われることに留意の上、研究計画調書を作成すること。
	}
%	\noindent
	\GeneralInstructions
}
 % pieces
% user07_header
% ===== my favorite packages ====================================
% ここに,自分の使いたいパッケージを宣言して下さい。
\usepackage{wrapfig}
%\usepackage{amsmath}
%\usepackage{amssymb}
%\usepackage{txfonts}
\usepackage{newtxtext,newtxmath}
\usepackage{pifont} % 丸数字を使うためのパッケージ
%\usepackage{mb}
%\DeclareGraphicsRule{.tif}{png}{.png}{`convert #1 `dirname #1`/`basename #1 .tif`.png}
\usepackage{lineno}
\usepackage{hyperref}
\usepackage{url}
\usepackage{pxjahyper}
\usepackage{xcolor}
\hypersetup{
    colorlinks=true,
    citecolor=blue,
    linkcolor=blue,
    urlcolor=blue,
}
\renewcommand{\emph}[1]{{\sffamily\bfseries{#1}}}

\usepackage{titlesec}
\titleformat*{\section}{\huge\sffamily\bfseries}
\titleformat*{\subsection}{\large\sffamily\bfseries}
%\renewcommand{\thesection}{\textsf{\arabic{section}}}

% ===== my personal definitions ==================================
% ここに,自分のよく使う記号などを定義して下さい。
\newcommand{\klpionn}{K_L \to \pi^0 \nu \overline{\nu}}
\newcommand{\kppipnn}{K^+ \to \pi^+ \nu \overline{\nu}}
%%% new symbols
%\newcommand*{\meg}{\mu\to e \gamma}
%\newcommand*{\megc}{\ifmmode\mu^+ \to e^+ \gamma\else$\mu^+ \to e^+ \gamma$\fi}
\newcommand*{\meg}{\muup\to \mathrm{e} \gammaup}
\newcommand*{\megc}{\ifmmode\muup^+ \to \mathrm{e}^+ \gammaup\else$\muup^+ \to \mathrm{e}^+ \gammaup$\fi}
\newcommand*{\rmd}{\ifmmode\muup^+ \to \mathrm{e}^+ \nuup_\mathrm{e} \bar{\nuup}_\muup \gammaup\else$\muup^+ \to \mathrm{e}^+ \nuup_\mathrm{e} \bar{\nuup}_\muup \gammaup$\fi}
\newcommand*{\ee}{\mathrm{e^+e^-}}
%e%% end new symbols


% ===== 欄外メモ ==================
\newcommand{\memo}[1]{\marginpar{#1}}
%\renewcommand{\memo}[1]{}	% 全てのメモを表示させないようにするには,行頭の"%"を消す


%\input{../../sample/simple/contents}	% skip
% hook5 : right before \begin{document} ==============
 % pieces

\begin{document}
% hook7 : right after \begin{document} ==============
 % pieces


%#Split: 03_abilities  
%#PieceName: p03_abilities
% p01_purpose_plan_00.tex
\KLBeginSubject{01}{1}{1 研究目的、研究方法など}{5}{F}{3}{jsps-p1-header}{jsps-default-header}

%% p03_abilities_00.tex
\KLBeginSubject{02}{2}{}{2}{F}{}{jsps-subject-header}{jsps-default-header}

\noindent

%begin 研究業績リスト ====================

\section{研究業績一覧}
	
\subsection{学術論文(査読あり) \textcolor{red}{$\bigcirc$}付き項目は主要論文}

	\begin{enumerate}
			\setlength{\itemsep}{-1pt}

\bibitem{YU:shield}
        \me, Y.~Fukao, M.~Shunsuke, S.~Mihara, 
        ``Radiation Shielding System for the COMET Pion Capture Solenoid'', 
        JPS Conf.\ Proc. (in press), \textbf{ 筆頭著者}, 研究責任者.
\bibitem{YU:commissioning}
         Y.~Fukao, ..., \me\ \textit{et al.}, 
        ``Construction and Beam Commissioning for the COMET Experiment'', 
        JPS Conf.\ Proc. (in press).
\bibitem{YU:target}
         S.~Makimura, ..., \me\ \textit{et al.}, 
        ``Recent Upgrade on Muon Target at J-PARC'', 
        JPS Conf.\ Proc. (in press).

\bibitem{acps_2025}
R. Sakakibara, ..., \me, {\it et al.},
``Development of an active converter pair spectrometer for the future search for \megc,'' Nucl.\ Instrm.\ Methods A {\bf 1082} (2026) 170961,
\href{https://doi.org/10.1016/j.nima.2025.170961}{doi:10.1016/j.nima.2025.170961}
\bibitem{x17_2025} 
K. Afanaciev, ..., \me, {\it et al}.\ (MEG II Collaboration),
  ``Search for the X17 particle in $^{7}\textrm{Li}(\textrm{p},\textrm{e}^+ \textrm{e}^{-}) ^{8}\textrm{Be}$ processes  the MEG II detector,''
  Eur.\ Phys.\ J.\ C {\bf 85}  (2025) 763,
  \href{https://doi.org/10.1140/epjc/s10052-025-14345-0}{doi:10.1140/epjc/s10052-025-14345-0}.

\bibitem{cdch_2024}
A.~M. Baldini, ..., \me, {\it et al.},
``Performances of a new generation tracking detector: the MEG II cylindrical drift chamber,''
  Eur.\ Phys.\ J.\ C {\bf 84}  (2024) 473,
\href{https://doi.org/10.1140/epjc/s10052-024-12711-y}{doi:10.1140/epjc/s10052-024-12711-y}.

\bibitem{meg2_2024} 
\textcolor{red}{$\bigcirc$} 
K. Afanaciev, ..., \me, {\it et al}.\ (MEG II Collaboration),
  ``A search for \megc\ with the first dataset of the MEG II experiment,''
  Eur.\ Phys.\ J.\ C {\bf 84}  (2024) 216,
  \href{https://doi.org/10.1140/epjc/s10052-024-12416-2}{doi:10.1140/epjc/s10052-024-12416-2},
  引用数XXX.

\bibitem{meg2detector_2024} 
\textcolor{red}{$\bigcirc$} 
K. Afanaciev, ..., \me, {\it et al}.\ (MEG II Collaboration),
  ``Operation and performance of the MEG II detector,''
  Eur.\ Phys.\ J.\ C {\bf 84}  (2024) 190,
  \href{https://doi.org/10.1140/epjc/s10052-024-12415-3}{doi:10.1140/epjc/s10052-024-12415-3},
  \textbf{協同筆頭著者}, 引用数XXX.

\bibitem{ptc_2024}
T. Yonemoto, ..., \me, {\it et al.},
``The latest performance and refurbishment of the pixelated Timing Counter (pTC) in the MEG II experiment,'' Nucl.\ Instrm.\ Methods A {\bf 1068} (2024) 169720,
\href{https://doi.org/10.1016/j.nima.2024.169720}{doi:10.1016/j.nima.2024.169720},
研究責任者, 筆頭著者T.~Yonemoto(大学院生)の指導.

\bibitem{alignment_2024}
A. Ventrurini, ..., \me, {\it et al.},
``Alignment of the MEG II cylindrical drift chamber,'' Nucl.\ Instrm.\ Methods A {\bf 1067} (2024) 169680,
\href{https://doi.org/10.1016/j.nima.2024.169680}{doi:10.1016/j.nima.2024.169680}.

\bibitem{cdch_2023}
M.~Chiappini, ..., \me, {\it et al.},
``The Cylindrical Drift Chamber of the MEG II experiment,'' Nucl.\ Instrm.\ Methods A {\bf 1047} (2023) 167740,
\href{https://doi.org/10.1016/j.nima.2022.167740}{doi:10.1016/j.nima.2022.167740}.

\bibitem{ptc_2023}
P.~W.~Cattaneo, G.~Boca, M.~De~Gerone, M.~Francesconi, L.~Galli, F.~Gatti, W.~Ootani, M.~Rossella, \me, Y.~Uchiyama, M.~Usami, T.~Yonemoto,
``Operational results with the pixelated Time Detector of MEG II experiment during the first year of physics data taking,'' Nucl.\ Instrm.\ Methods A {\bf 1046} (2023) 167751,
\href{https://doi.org/10.1016/j.nima.2022.167751}{doi:10.1016/j.nima.2022.167751},
研究責任者.
\bibitem{trigger_2023}
M.~Francesconi, ..., \me, {\it et al.},
``The trigger system for the MEG II experiment,'' Nucl.\ Instrm.\ Methods A {\bf 1046} (2023) 167736,
\href{https://doi.org/10.1016/j.nima.2022.167736}{doi:10.1016/j.nima.2022.167736}.

\bibitem{xec_2023}
T.~Iwamoto, ..., \me, {\it et al.},
``The liquid xenon detector for the MEG II experiment to detect 52.8 MeV $\gamma$ with large larea VUV-sensitive MPPCs,'' Nucl.\ Instrm.\ Methods A {\bf 1046} (2023) 167720,
\href{https://doi.org/10.1016/j.nima.2022.167720}{doi:10.1016/j.nima.2022.167720}.


%\bibitem{himb}
%M.~Abe, ..., \me, {\it et al.},
%``Science case for the new High-Intensity Muon Beam HIMB at PSI,''
%arXiv:2111.05788, 
%%\href{https://arxiv.org/abs/2111.05788}{arXiv:2111.05788} 
%\href{https://doi.org/10.48550/arXiv.2111.0578}{doi:10.48550/arXiv.2111.0578} 
%(2021), 将来実験構想.

\bibitem{symmetry}
A.~M.~Baldini, ..., \me, {\it et al.} (MEG II Collaboration),
``The search for \megc\ with $10^{-14}$ sensitivity: the upgrade of the MEG experiment,''
Symmetry {\bf 13} (2021) 1591
\href{https://doi.org/10.3390/sym13091591}{doi:10.3390/sym13091591}, 
\textbf{協同筆頭著者}.

\bibitem{radiation}
G.~Boca, P.~W.~Cattaneo, M.~De~Gerone, F.~Gatti, M.~Nakao, M.~Nishimura, W.~Ootani, M.~Rossella, \me, M.~Usami, K.~Yanai, ``Timing resolution of a plastic scintillator counter read out by radiation damaged SiPMs connected in series,'' Nucl.\ Instrum.\ Methods A {\bf 999} (2021) 165173,
\href{https://doi.org/10.1016/j.nima.2021.165173}{doi:10.1016/j.nima.2021.165173},
研究責任者, 筆頭著者M.~Usami(大学院生)の指導.

\bibitem{MEx2G}
A.~M.~Baldini, ..., \me, {\it et al}.\ (MEG Collaboration),
``Search for lepton flavour violating muon decay mediated by a new light particle in the MEG experiment,'' Eur.\ Phys.\ J.\ C \textbf{80} (2020) 858,
\href{https://doi.org/10.1140/epjc/s10052-020-8364-1}{doi:10.1140/epjc/s10052-020-8364-1},
コラボレーション内評価・編集委員,  筆頭著者M.~Nakao(大学院生)の指導.

\bibitem{vci2019}
M.~Nishimura, ..., \me, {\it et al}., 
``Full system of positron timing counter in MEG II having time resolution below 40 ps with fast plastic scintillator readout by SiPMs,''
Nucl.\ Instrum.\ Methods A \textbf{958} (2020) 162785,
\href{https://doi.org/10.1016/j.nima.2019.162785}{doi:10.1016/j.nima.2019.162785},
研究責任者,  筆頭著者M.~Nishimura(大学院生)の指導.

\bibitem{laser} 
  G.~Boca, P.~W.~Cattaneo, M.~De~Gerone, M.~Francesconi, L.~Galli, F.~Gatti, J.~Koga,
  M.~Nakao, M.~Nishimura, W.~Ootani, M.~Rossella, \me, M.~Usami, K.~Yanai, K.~Yoshida,
  ``The laser-based time calibration system for the MEG II pixelated Timing Counter,''
  Nucl.\ Instrum.\ Methods A {\bf 947}  (2019) 162672,
  \href{https://doi.org/10.1016/j.nima.2019.162672}{doi:10.1016/j.nima.2019.162672},
  研究責任者,  筆頭著者M.~Nakao(大学院生)の指導.
	 
\bibitem{Baldini:2018nnn}  
A.\,M.~Baldini, ..., \me, {\it et al}.\ (MEG II Collaboration),
  ``The design of the MEG II experiment,''
  Eur.\ Phys.\ J.\ C {\bf 78}  (2018) 380,
  \href{https://doi.org/10.1140/epjc/s10052-018-5845-6}{doi:10.1140/epjc/s10052-018-5845-6},
  \textbf{協同筆頭著者}.

\bibitem{vci2016} 
  \me, G.~Boca, P.~W.~Cattaneo, M.~De~Gerone F.~Gatti, M.~Nakao, M.~Nishimura, W.~Ootani, G.~Pizzigoni, M.~Rossella, M.~Simonetta, K.~Yoshida, 
  ``30-ps time resolution with segmented scintillation counter for MEG II,''
  Nucl.\ Instrum.\ Methods A {\bf 845} (2017) 507--510,
  \href{https://doi.org/10.1016/j.nima.2016.06.072}{doi:10.1016/j.nima.2016.06.072},
  \textbf{筆頭著者}, 研究責任者.

\bibitem{btf2014} 
  P.~W.~Cattaneo, M.~De~Gerone F.~Gatti, M.~Nishimura, W.~Ootani, M.~Rossella, S.~Shirabe, \me, 
  ``Time resolution of time-of-flight detector based on multiple scintillation counters readout by SiPMs,''
  Nucl.\ Instrum.\ Meth.\ A {\bf 828} (2016) 191--200,
  \href{http://dx.doi.org/10.1016/j.nima.2016.05.038}{doi:10.1016/j.nima.2016.05.038},
  \textbf{筆頭著者}, 研究責任者.

\bibitem{TheMEG:2016wtm} 
\textcolor{red}{$\bigcirc$} 
  A.\,M.~Baldini, ..., \me, {\it et al}.\ (MEG Collaboration),
  ``Search for the lepton flavour violating decay $\megc$ with the full dataset of the MEG experiment,''
  Eur.\ Phys.\ J.\ C {\bf 76} (2016) 434,
  \href{https://doi.org/10.1140/epjc/s10052-016-4271-x}{doi:10.1140/epjc/s10052-016-4271-x}
  引用数881, Sect.\ 3.1(ガンマ線再構成), 4.4(崩壊探索背景事象), 4.6(規格化)執筆担当.

\bibitem{polarization}
  A.~M.~Baldini, ..., \me, {\it et al}.\ (MEG Collaboration),
  ``Muon polarization in the MEG experiment: predictions and measurements,''
  Eur.\ Phys.\ J.\ C {\bf 76} (2016) 223,
  \href{http://dx.doi.org/10.1140/epjc/s10052-016-4047-3}{doi:10.1140/epjc/s10052-016-4047-3}.

\bibitem{Adam:2013gfn} 
\textcolor{red}{$\bigcirc$}
  A.~M.~Baldini, ..., \me, {\it et al}.\ (MEG Collaboration),
  ``Measurement of the radiative decay of polarized muons in the MEG experiment,''
  Eur.\ Phys.\ J.\ C {\bf 76} (2016) 108,
  \href{https://doi.org/10.1140/epjc/s10052-016-3947-6}{doi:10.1140/epjc/s10052-016-3947-6},
\textbf{筆頭著者}.

\bibitem{Ootani:2015cia} 
  W.~Ootani, ..., \me,  {\it et al}., %K.~Ieki, T.~Iwamoto, D.~Kaneko, T.~Mori, S.~Nakaura, M.~Nishimura, S.~Ogawa, R.~Sawada, N.~Shibata, \me, K.~Yoshida, K.~Sato, R.~Yamada,
  ``Development of deep-UV sensitive MPPC for liquid xenon scintillation detector,''
  Nucl.\ Instrum.\ Methods A {\bf 787} (2015) 220--223,
  \href{https://doi.org/10.1016/j.nima.2014.12.007}{doi:10.1016/j.nima.2014.12.007}.

  \bibitem{tc-single} 
  \textcolor{red}{$\bigcirc$} 
  P.~W.~Cattaneo, M.~De~Gerone F.~Gatti, M.~Nishimura, W.~Ootani, M.~Rossella, \me, 
  ``Development of high precision timing counter based on plastic scintillator with SiPM readout,''
  IEEE Trans.\ Nucl.\ Sci.\ {\bf 61} (2014) 2657--2666,
  \href{http://dx.doi.org/10.1109/TNS.2014.2347576}{doi:10.1109/TNS.2014.2347576},
  \textbf{筆頭著者}, 研究責任者, 引用数XXX.

\bibitem{rmd1}
  J.~Adam, ..., \me, {\it et al.},
  ``Measurement of inner Bremsstrahlung in polarized muon decay with MEG,''
  Nucl.\ Phys.\ B Proc.\ Suppl.\ \textbf{248--250} (2014) 108--111,
  \href{https://doi.org/10.1016/j.nuclphysbps.2014.02.019}{doi:10.1016/j.nuclphysbps.2014.02.019},
  \textbf{筆頭著者}.

\bibitem{Adam:2013vqa} 
  J.~Adam, ..., \me,  {\it et al.},
  ``The MEG detector for \megc\ decay search,''
  Eur.\ Phys.\ J.\ C {\bf 73} (2013) 2365,
  \href{https://doi.org/10.1140/epjc/s10052-013-2365-2}{doi:10.1140/epjc/s10052-013-2365-2}.

\bibitem{Adam:2013mnn} 
  J.~Adam, ..., \me,  {\it et al}.\ (MEG Collaboration),
  ``New constraint on the existence of the \megc\ decay,''
  Phys.\ Rev.\ Lett.\  {\bf 110} (2013) 201801,
  \href{https://doi.org/10.1103/PhysRevLett.110.201801}{doi:10.1103/PhysRevLett.110.201801}.

%\bibitem{upgrade_proposal}
% A.~M.~Baldini, ..., \me, {\it et al}.\ (MEG Collaboration),
% ``MEG upgrade proposal,'' Research proposal to PSI R-99-05.2 (2012),
% \href{https://doi.org/10.48550/arXiv.1301.7225}{doi:10.48550/arXiv.1301.7225},
% 引用数346, Sect.\ VI-B(タイミングカウンター), XIII-D(シリコンバーテックス検出器)執筆担当.

\bibitem{TC2}
M.~De~Gerone, ..., \me, {\it et al}. %, S.~Dussoni, K.~Fratini, F.~Gatti, R.~Valle, G.~Boca, P.~W.~Cattaneo, R.~Nardo, 
%M.~Rossella, L.~Galli, M.~Grassi, D.~Nicolo, \me, D.~Zanello, 
``Development and commissioning of the Timing Counter for the MEG experiment,''
IEEE Trans.\ Nucl.\ Sci.\ \textbf{59} (2012) 379--388,
\href{http://dx.doi.org/10.1109/TNS.2012.2187311}{doi:10.1109/TNS.2012.2187311}.

\bibitem{meg2011}
J.~Adam, ..., \me, ,  {\it et al}.\ (MEG Collaboration),
``New limit on the lepton-flavor-violating decay \megc,''
Phys.\ Rev.\ Lett.\ \textbf{107} (2011) 171801,
\href{http://dx.doi.org/10.1103/PhysRevLett.107.171801}{doi:10.1103/PhysRevLett.107.171801}.

\bibitem{cw}
J.~Adam, ..., \me, {\it et al}.\ (MEG Collaboration),
``Calibration and monitoring of the MEG experiment by a proton beam from a Cockcroft--Walton accelerator,''
Nucl.\ Instrum.\ Methods A \textbf{641} (2011) 19--32,
\href{https://doi.org/10.1016/j.nima.2011.03.048}{doi:10.1016/j.nima.2011.03.048}.

\bibitem{TC1}
M.~De~Gerone, ..., \me, {\it et al}., %S.~Dussoni, K.~Fratini, F.~Gatti, R.~Valle, G.~Boca, P.~W.~Cattaneo, 
%M.~Rossella, R. Nardo, A.~Papa, G.~Signorelli, G.~Cavoto, G.~Pirreda, F.~Renga, C.~Voena, \me, 
``The MEG timing counter calibration and performance,''
Nucl.\ Instrum.\ Methods A \textbf{638} (2011) 41--46,
\href{https://doi.org/10.1016/j.nima.2011.02.044}{doi: 10.1016/j.nima.2011.02.044}.

\bibitem{meg2010}
J.~Adam, ..., \me, ,  {\it et al}.\ (MEG Collaboration),
``A limit for the $\meg$ decay from MEG experiment,''
Nucl.\ Phys.\ B \textbf{834} (2010) 1--12,
\href{https://doi.org/10.1016/j.nuclphysb.2010.03.030}{doi:10.1016/j.nuclphysb.2010.03.030},
博士論文内容相当.

\bibitem{gamma_rec}
\me, ``Gamma ray reconstruction with liquid xenon gamma ray calorimeter for the MEG experiment,''
Nucl.\ Instrum.\ Methods A \textbf{617} (2010) 118--119,
\href{https://doi.org/10.1016/j.nima.2009.09.100}{doi:10.1016/j.nima.2009.09.100},
\textbf{筆頭著者}.

\bibitem{LXeStorage}
T.~Iwamoto, ..., \me,  {\it et al}.,  %, R.~Sawada, T.~Haruyama, S.~Mihara, T.~Doke, Y.~Hisamatsu, K.~Kasami, A.~Maki, T.~Mori, H.~Natori, H.~Nishiguchi, Y.~Nishimura, W.~Ootani, K.~Terasawa, \me, S.~Yamada,
``Development of a large volume zero boil-off liquid xenon storage system for muon rare decay experiment (MEG),''
Cryogenics \textbf{49} (2009) 254--258,\\
\href{https://doi.org/10.1016/j.cryogenics.2008.09.003}{doi:10.1016/j.cryogenics.2008.09.003}.

\bibitem{LXePump}
S.~Mihara, ..., \me,  {\it et al}., %T.~Haruyama, T.~Iwamoto, \me, W.~Ootani, K.~Kasami, R.~Sawada, K.~Terasawa, T.~Doke, H.~Natori, H.~Nishiguchi, A.~Maki, T.~Mori, S.~Yamada,
``Development of a method for liquid xenon purification using a cryogenic centrifugal pump,''
Cryogenics \textbf{46} (2006) 688--693,
\href{https://doi.org/10.1016/j.cryogenics.2006.04.003}{doi:10.1016/j.cryogenics.2006.04.003}.

\end{enumerate}



\subsection{国際会議発表, 国内会議発表}
\begin{enumerate}
\setcounter{enumi}{38}
	\setlength{\itemsep}{-1pt}
	
	\bibitem{hql2025} 
 	 \me, ``Muon LFV/LFU measurements at J-PARC, PSI, FNAL,''
  	17th International Conference on Heavy Quarks and Leptons, Beijing, Sep. 2025, \textbf{招待講演}.

	\bibitem{phits2025-2} 
  	\underline{内山雄祐},  	``J-PARC COMET実験における放射線計算,''
  	第2回 EGS5-Geant4-PHITS合同研究会, 筑波,  2025年5月.
  	\bibitem{jps2025s} 
  	\underline{内山雄祐},  	``COMET実験パイオン生成システムの遮蔽設計,''
  	日本物理学会2025年春季大会, online,  2025年3月.
	\bibitem{phits2025-1} 
  	\me, ``Radiation study for the pion production system of the COMET experiment at J-PARC,''
 	 PHITS Workshop and Intermediate Course, Tokai, Feb. 2025.
  	\bibitem{fpws2024} 
  	\underline{内山雄祐},  	``レプトンフレーバー物理,''
  	Flavor Physics Workshop 2024, 蒲郡,  2024年12月, \textbf{招待講演}.
  	\bibitem{jparc2024}
  	\me, ``Radiation ShieldingSystem for the COMET Pion Capture Solenoid,''
  	4th J-PARC Sympojium, Oct. 2024.
	\bibitem{hokkaido2024} 
 	 \me, ``Search for charged lepton flavor violation with muons,''
  	Hokkaido Workshop on Particle Physics at Crossroads, Sapporo, Mar. 2024, \textbf{招待講演}.
	\bibitem{kekeph2023} 
 	 \me, ``High energy physics with muons,''
  	KEK Theory Meeting on Particle Physics Phenomenology, Tsukuba, Nov. 2023, \textbf{招待講演}.
	\bibitem{twoinfinities2023} 
 	 \me, ``Particle Physics with Muons,''
  	Int.\ Conf.\ on the Physics of the Two Infinities, Kyoto, Mar. 2023, \textbf{招待講演}.
  	\bibitem{jps2023s} 
  	\underline{内山雄祐},  	``\megc 探索実験MEG II Run 2022のまとめと今後の展望,''
  	日本物理学会2023年春季大会, online,  2025年3月.
  	\bibitem{jps2022a} 
  	\underline{内山雄祐},  	``MEG II 陽電子スペクトロメータにおける機械学習を活用したヒット再構成の改善,''
  	日本物理学会2022年秋季大会,  岡山,  2022年9月.
  	\bibitem{jps2021a} 
  	\underline{内山雄祐},  	``MEG II 実験:2021 年エンジニアリングランの現状と計画,''
  	日本物理学会2021年秋季大会,  online,  2021年9月.
  	\bibitem{jps2021s} 
  	\underline{内山雄祐},  	``機械学習を活用した高計数率ドリフトチェンバーのヒット再構成,''
  	日本物理学会第76回年次大会,  online,  2021年3月.
	\bibitem{aps2019} 
 	 \me, ``The MEG II experiment in serach of $\meg$,''
  	APS Division of Particles \& Fields Meeting, Boston, Jul. 2019.
  	
%  	\bibitem{jps2019a} 
%  	\underline{内山雄祐},  	``\megc 探索実験 MEG II 現状と今後の見込み,''
%  	日本物理学会2019年秋季大会,  山形,  2019年9月.

	\bibitem{icasipm2018} 
 	 \me, ``Large scale characterization of SiPMs in the MEG II experiment,''
  	Int. Conf. on the Advancement of Silicon Photomultipliers, Schwetzingen, Jun. 2018, \textbf{招待講演}.

%  	\bibitem{jps2018s} 
%  	\underline{内山雄祐},  	``\megc 探索実験 MEG II 2018年度の展望,''
%  	日本物理学会第73回年次大会,  野田,  2018年3月.
	\bibitem{vci2016} 
 	 \me, ``30-ps Time Resolution with Segmented Scintillation Counter for MEG II,''
  	14th Vienna Conf. on Instrumentation, Vienna, Feb. 2016.
	\bibitem{ckm2014} 
 	 \me, ``Lepton Flavor Violating Muon Processes,''
  	8th Int. Workshop on the CKM Unitarity Triangl, Vienna, Sep. 2014, \textbf{招待講演}.
	\bibitem{ieee2013} 
 	 \me, ``High Precision Measurement of Positron Time in MEG Upgrade,''
  	IEEE NSS/MIC?RTSD, Seoul, Oct. 2013.
	\bibitem{eps2013} 
 	 \me, ``Upgrade of MEG experiment,''
  	European Physical Society Conf. on High Energy Physics, Stockholm, Jul. 2013.


  	\bibitem{jps2013s} 
  	\underline{内山雄祐},  	``荷電レプトンフレーバー非保存探索による LHC 時代の素粒子物理シンポジウム: DCミューオンビームによる cLFV 探索,''
  	日本物理学会第68回年次大会,  広島,  2013年3月, \textbf{招待講演}.
  	\bibitem{jps2011a} 
  	\underline{内山雄祐},  	``Analysis of the First MEG Physics Data to Search for the Decay \megc,''
  	日本物理学会2011年秋季大会,  弘前,  2011年9月, \textbf{招待講演}.


\end{enumerate}

\subsection{総説・解説}
\begin{enumerate}
\setcounter{enumi}{59}
	\setlength{\itemsep}{-1pt}
 \bibitem{cern_courier} 
  A. Papa, F. Renga, \me,
  ``Hunting the muon's forbidden decay,''
  CERN Courier May/June (2019) 45--47
 \bibitem{highenergy} 
  家城佳, \underline{内山雄祐},
  ``MEG II 実験 ---分岐比$10^{-14}$ 台での$\megc$崩壊探索---,''
  高エネルギーニュース {\bf 37} No.\,1 (2018) 1--10
\end{enumerate}

\vspace{-8mm}
\subsection{著書}
\vspace{-3mm}
なし
\vspace{-8mm}
\subsection{特許}
\vspace{-3mm}
なし
\vspace{-4mm}

\subsection{全学術論文の被引用回数の合計, h-index}
\noindent
総被引用回数: 1,895, h-index: 16 (Scopus, 2025年9月28日付)

\newpage


%end 研究業績リスト ====================

\section{競争的研究資金および外部研究資金の獲得リスト}
\subsection{科学研究費補助金}
\begin{itemize}
  \setlength{\parskip}{0cm} % 段落間
  \setlength{\itemsep}{0cm} % 項目間
\item 基盤研究(S), 「高分解能キセノン測定器と大強度パイ中間子ビームによるレプトン普遍性破れの精密検証」, 2024年4月--2029年3月, 分担, 森俊則, 分担額:6,500千円
\item 新学術領域研究(ニュートリノで拓く素粒子と宇宙)公募研究, 「極低物質量・高計数率飛跡検出器で挑む荷電レプトンフレーバーの破れの探索」, 2021年4月--2023年3月, 代表, 総額:6,370千円
\item 基盤研究(A), 「高分解能大型液体キセノン測定器によるレプトン普遍性の破れの精密検証」, 2020年4月--2024年3月, 分担, 森俊則, 分担額:4,800千円
\item 若手研究(B),  「ミュー粒子稀崩壊探索実験のさらなる高輝度化に向けたソフトウェアトリガーの開発」, 2017年4月--2019年3月, 代表, 総額:4,030千円
\item 特別研究員奨励費 「ミュー粒子稀崩壊探索実験による超対称性理論の検証」研究代表者2007年4月--2009年3月, 代表, 総額:1,800千円
\end{itemize}

\subsection{受託研究費}
なし
\subsection{その他の競争的資金}
なし

	
\newpage

\section{研究歴}
\vspace{2zw}

私はこれまで一貫して,超高エネルギー物理の解明を目指した素粒子実験の研究をしてきた。標準理論では禁止されている,荷電レプトンのフレーバーを破るミュー粒子の稀崩壊事象を探索し,超対称大統一理論やニュートリノ質量の背後にある物理を通したバリオン数生成など,根源的な宇宙の謎の解明に挑んできた。
おもな実績は,
\begin{itemize}
  \setlength{\parskip}{0cm} % 段落間
  \setlength{\itemsep}{0cm} % 項目間
\item	\emph{MEG実験の完遂により世界最高感度での\megc 崩壊探索を実現した}
\item	\emph{MEG II実験の考案から立ち上げまでをやり遂げ,物理データ取得開始を実現した}
\end{itemize}
ことである。
国際共同実験であるMEG・MEG II実験の立案から測定器開発,物理解析に至るまで中心となって推進してきた中核となるメンバーである。革新的検出器を開発し\cite{Adam:2013vqa},MEG実験ではそれまでの上限値を30倍更新する新しい上限値を設定した\cite{TheMEG:2016wtm}。
MEG II実験はさらにその感度を10倍更新する計画で\cite{upgrade_proposal, Baldini:2018nnn},測定器の開発・制作を完了し\cite{Mori:highenergy},ついに昨年度,最初の物理データ取得に成功した\cite{cdch_2023,ptc_2023,trigger_2023,xec_2023}。

\subsection{MEG実験}
MEG実験では\emph{液体キセノン検出器を用いたガンマ線の解析を責任者として担当}した。詳細な波形解析手法と較正手法を確立し,検出効率64\%,エネルギー分解能1.6\%,時間分解能63~psを達成した\cite{gamma_rec, TheMEG:2016wtm}。
検出器の高い性能を発揮するには,較正が最も大事である。様々な較正データを統合的に分析し,大型検出器における非一様性およびエネルギースケールを系統不確かさ0.3\%にまで抑えることに成功した。
新しい波形解析アルゴリズムはより高いレートでの実験を可能にするもので,MEG II実験における高レートビーム下でもパイルアップを分離することで性能を落とすことなく液体キセノン検出器を運転できることを実証した。

以下の物理解析を主導し物理結果をパブリッシュした。
\paragraph{\megc 崩壊探索}
検出器の場所による応答の違いや変数間の相関を研究し尤度関数に組み込むことで探索感度を最大限に高めた解析手法を開発した。世界最高感度を達成し,分岐比上限値:
\begin{equation}
\mathcal{B}(\megc) < 4.2×10^{-13} \quad  (90\%\ \mathrm{C.L.})
\end{equation}
を与えた\cite{TheMEG:2016wtm}。これはあらゆる粒子の崩壊分岐比に対する最も小さな上限値となっており,様々な新物理理論に厳しい制限を与える結果となっている。

\paragraph{ミュー粒子放射ミシェル崩壊の解析}
放射ミシェル崩壊\rmd は\megc 事象の背景事象となる一方で,$\mathrm{e}$--$\gammaup$間の時間較正源として活用できる。さらに,その崩壊確率と分布は電弱相互作用で計算できることに着目し,独自にデータを解析して実験全体の正当性を実証した。この崩壊を用いたミュー粒子の規格化因子および偏極度の新しい測定手法を確立した\cite{rmd1}。7\%を計上していた規格因子不確かさを4\%に抑えることに成功した。この崩壊モードにおける最高精度の測定を実現し\cite{Adam:2013gfn},結果はParticle Data Bookにも掲載されている。

\paragraph{新しい崩壊モード$\muup^+\to\mathrm{e}^+\mathrm{X},\mathrm{X}\to\gammaup\gammaup$}
このモード独自の2光子再構成手法やキネマティクスを用いた事象再構成手法を大学院生とともに開発。データに基づいた背景事象推定や系統不確かさの取り扱いなど統計手法を主導した。このモードの世界最初の探索結果をパブリッシュした\cite{MEx2G}。


\subsection {MEG II実験}
%\begin{wrapfigure}[13]{r}{9cm}
%	\centering
%	 \vspace{-0zw}
%	\includegraphics[width=1.0\linewidth]{figs/pTCPic}
%	 \caption{MEG IIタイミングカウンターと開発を担当した大学院生(2020年2月,本人・中央)}
%	\label{fig:pTCPic}         
%\end{wrapfigure}


12年度よりアップグレード実験MEG IIを推進\cite{upgrade_proposal}。陽電子の時間を高精度で測定する\emph{細分型シンチレーションタイミングカウンターの開発を責任者としてイタリアグループとの国際協力を構築・主導}しながら進めてきた。SiPM読出しの時間分解能を追究し,独自の読出し・解析手法を確立し\cite{tc-single},35~psの時間分解能を実現する検出器を開発した\cite{laser, vci2016}。
15年よりMEG IIビームを用いたパイロットランを他の検出器に先駆け実施し,読み出し回路やトリガー・DAQの開発を推し進めるなど,実験全体を先駆してきた。17年に実機を完成させた\cite{vci2019}(図 \ref{fig:pTCPic})。
開発した技術は論文にまとめ\cite{radiation, laser, btf2014, tc-single},国内外の学会で広く報告・議論をしてきた。その結果,PANDA TOF,Mu2e ECAL, J-PARC E42/45 ホドスコープ,ハイパー核寿命測定実験 TDLなど素粒子・原子核実験の多くの測定器に波及し,関連技術のスタンダードとなっている。

19年度よりMEG II実験成功のカギを握っているドリフトチェンバーの解析に着手。新学術公募研究において,他に類を見ない高計数率下での高効率解析手法を実データに基づいて開発している。デジタルフィルターを駆使した波形解析に加え,深層学習を活用した全く新しい波形解析を開発した。

MEG II実験の\emph{コンピューティング・ソフトウェアコーディネーター}も兼任しており,若手研究(B)では大量のデータをリアルタイムで解析することで,信号事象を取りこぼすことなくデータ量を削減する手法とシステムを構築した。
最新のソフトウェアを導入しながら解析・シミュレーションコードの開発を指揮している。MEG II実験に必要な計算機資源を見積もり,PSI計算機グループと共同で専用の計算機システムを構築した。オンラインシステムからアーカイブシステムまでの自動データ転送システムやオフライン解析の自動化システムを開発し運用している。最近では,コンテナ技術を用いた解析環境の可用性の拡張・保存や,深層学習を解析に取り入れるための統一解析フレームワークの開発などを進めている。

\subsection{将来計画}
日本物理学会にて「DCミューオンビームによるcLFV探索」(第68回(2013)次大会【招待講演】)や「崩壊分岐比感度$10^{-15}$の新しい\megc 探索実験の検討」(2014年秋季大会)でMEG II実験の先の実験設計を発表・議論するなど,次世代の荷電レプトンフレーバー物理実験を主導している\cite{himb}。



\subsection{教育実績}
東京大学の大学院生指導を13年間にわたり行い,学生の教育と研究者育成にも力を注いできた。MEG実験のデータ解析やMEG IIタイミングカウンターの開発を研究した学生とは,文字通り寝食を共にしながら研究を進め,各局面における課題を解決する実践力を身に着けさせることを意識して指導をしてきた。%主体性を育むために常に学生自身に調べ・考え・手を動かすことを求めている。
これまでにMEG実験物理解析で博士論文6本(うち3本日本物理学会若手奨励賞受賞),タイミングカウンターの開発で修士論文6本(うち1本測定器開発優秀修士論文賞受賞),博士論文2本を直接指導し輩出した実績をもつ。

大学院生を対象にした「稀崩壊探索実験に必要な統計解析」講座(全8回)や素粒子実験用ソフトウェア・解析講習会(全7回)を実施したり,夏にPSI研究所を訪れた総研大や九州大の学生を含めた学生ゼミを主催したりした経験がある。エキスパティーズとしては,素粒子物理学・素粒子実験・粒子検出器・光検出器・データ解析・統計解析・プログラミング・真空技術など。


\subsection{受賞・獲得研究資金}
\paragraph{受賞歴}
\begin{itemize}
  \setlength{\parskip}{0cm} % 段落間
  \setlength{\itemsep}{0cm} % 項目間
\item Best poster award in the 1st workshop on the Charged Lepton Flavor Violation (2013) \cite{rmd1}
\item 第5回 日本物理学会若手奨励賞 (素粒子実験領域) (2011)
\item 第12回 高エネルギー物理学奨励賞 (2010)
\item NIM-A Young Scientist Award (於 第11回ピサ会議) (2009) \cite{gamma_rec}
\end{itemize}

\paragraph{獲得研究資金}
\begin{itemize}
  \setlength{\parskip}{0cm} % 段落間
  \setlength{\itemsep}{0cm} % 項目間
\item 新学術領域研究(ニュートリノで拓く素粒子と宇宙)公募研究「極低物質量・高計数率飛跡検出器で挑む荷電レプトンフレーバーの破れの探索」研究代表者2021年4月--2023年3月
\item 基盤研究(A)「高分解能大型液体キセノン測定器によるレプトン普遍性の破れの精密検証」
研究分担者(研究代表者:森俊則)2020年4月--2024年3月
\item 若手研究(B) 「ミュー粒子稀崩壊探索実験のさらなる高輝度化に向けたソフトウェアトリガーの開発」研究代表者2017年4月--2019年3月
\item 特別研究員奨励費 「ミュー粒子稀崩壊探索実験による超対称性理論の検証」研究代表者2007年4月--2009年3月
\end{itemize}

\newpage


\section{着任後の抱負}

%s02_purpose_plan_with_abstract
\noindent
%\textbf{(本文)}
%begin 研究目的と研究計画	====================
\vspace{-2zw}
\subsection{学術的興味と研究対象}

%%ガンマ線測定
%\paragraph{ガンマ線測定}
%ガンマ線測定は素粒子実験における根幹技術であるが,電気的に中性で透過率が高いため,その測定精度が多くの実験の感度を律速している。
%%エネルギー帯により異なる技術が用いられるが,
%$\mathcal{O}(10~\mathrm{MeV})$のガンマ線の検出・測定には従来よりカロリメータまたはペアスペクトロメータが使われてきた。
%カロリメータは結晶シンチレーターなどの有感物質中で電磁シャワーを起こさせ,そのエネルギーを測定する。高い検出効率が可能であるが,エネルギー分解能はより低いエネルギー領域で用いられる半導体検出器などに比べると劣る。また,ガンマ線の到来方向や反応位置に対する測定精度も悪い。
%一方,ペアスペクトロメータは重い物質による薄いコンバージョン層とそれに続く飛跡検出層で構成され,入射ガンマ線の対生成反応からの$\ee$対を磁気スペクトロメータで測定する。カロリメータに比べて,エネルギー・方向・反応位置の精密測定が可能であるが,一般に検出効率は桁違いに低くなる。
%検出効率を稼ぐためにコンバージョン層を厚くするとエネルギー分解能が悪くなるというトレードオフの関係が従来のペアスペクトロメータの限界といえる。
%
%ペアスペクトロメータは素粒子・原子核・天文などの分野において幅広く応用され科学技術の発展に寄与してきた。従来の限界をこえた高効率・高分解能ペアスペクトロメータの実現はガンマ線測定技術のブレークスルーとなり新たな実験を可能にする。本研究では特に,次世代の荷電レプトンフレーバー非保存事象探索実験への応用を目指す。

%\paragraph{荷電レプトンにおけるフレーバーの破れ}
力および物質の統一を図る大統一理論は非常に魅力的な理論である。%り,「この宇宙で大統一が実現されているか」という問いに答えを出したい。
しかし,どんなに美しい,または,都合の良い理論でも自然を再現しなければ意味がない。実験によって実証して初めて物理である。
超高エネルギーで成立している大統一を検証しようとすると,可能な実験は限られてくる。荷電レプトンのフレーバー混合(CLFV)は大統一に感度を有する数少ない現象でありユニークな研究対象である。また,新物理を探索するにはレプトンセクターが重要だと考える。カギとなるのはニュートリノ(の相方)の特殊性である。ニュートリノは,右巻きの相方が標準理論におけるゲージ一重項であり,この右巻きニュートリノが現在の宇宙の形成に重要な働きをしたと考えられる。この存在は荷電レプトンのフレーバー構造にも必ず影響を残す。CLFVを研究することで宇宙形成の謎に迫ることができる。

%荷電レプトン($\mathrm{e}, \muup, \tauup$)におけるフレーバー保存則の破れ(CLFV)は素粒子の標準模型では禁止されている。しかし,保存を保証する原理は無く,標準模型の適用範囲を超えた高いエネルギー領域で成立している新物理においては,フレーバー保存を破る相互作用が自然と存在することが期待される。したがって,CLFVを研究することで新物理の検証が可能であり,いまだ発見されていないCLFV過程の発見は新物理の確固たる証拠となる。
%電子陽電子コライダーLEPにおける超対称大統一理論の示唆,そして,ニュートリノ振動の発見は観測可能なCLFVを示唆するため,90年代末からCLFVの実験検証の重要性が認識され始めた。
%これに応えるように,国際共同実験MEGが2008--2016年に実施され,世界最高感度でCLFVを探索した。発見には至らず,従来の理論に対し厳しい制限を与える結果となった。その一方,
%さらに,
%ヒッグス粒子が超対称大統一理論の予想する質量領域で発見されたことやニュートリノ混合角$\theta_{13}$が大きかったことからCLFV探索はますます重要性を増している。%現在,COMET,Mu2e,Mu3e,そしてMEG IIと複数の実験が計画され,数年内に実験が開始されるという状況にある。

%一方,LHC実験により,超対称粒子に代表される新粒子がこれまで期待されてきた質量スケール(sub TeV)には存在せず,より重たいことが判明してきているため,CLFV過程は抑制され,現行の実験では感度が及ばない可能性もある。その場合,超対称大統一理論を徹底的に検証するには,超対称粒子に対して数十TeVまで感度を持たせた新しい実験が必要となる。

%本研究は現行実験の物理感度を飛躍的(数十倍)改善する新世代CLFV実験を可能にする実験技術を開発することで,
このようにCLFV実験は学術的に非常に重要で,今後の素粒子物理研究のメインストリームとなりえる,必ず遂行すべきプログラムである。MEG IIに続きCOMET実験を成功させ,自らの手で「\emph{力の大統一は実現されているか}」,「\emph{現在までに分かっている電弱スケールと大統一スケールを結びつける物理法則は何か}」という問いに答えを出したい。
%さらにその先の将来計画につながる研究をすすめたい。本職では,この目的を直接的に果たす研究ができる。それが本職を志望する動機である。
%現行CLFV実験の物理感度を数十倍改善する次世代実験を今準備し,現行実験と間髪をあけずに実施することが重要である。

\subsection{COMET実験}
COMET実験はCLFV過程の一つである$\muup$--e転換過程を探索する実験で,超対称大統一理論など多くの新物理モデルを検証できる実験である。そのPhase I実験は現在のリミットを2桁更新する計画であり,またさらに2桁高い感度を目指すPhase II実験へ向けた重要なステップでもある。
物理的には,MEG IIを超えた感度で超対称大統一理論を検証できるPhase IIまで実現させることが重要となる。そのためにも,\emph{Phase Iの早期開始と成功が現在の最重要課題}である。
%これまでにCOMETコラボレーションにより,Phase I実現にあと一歩のところまで準備が進められてきた。
私が着任した場合は,新規にこのプロジェクトに加わり,\emph{Phase Iの計画通りの実現に向けた新たな原動力}となりたい。そのために必要なことには何にでも挑戦していく所存である。
また,その結果と経験からPhase IIの10年以内の実現に向けて尽力したい。
MEG・MEG II実験で実験遂行・物理解析をこの手で行ってきた実績から,それを成し遂げる能力は十分あると自負している。

\paragraph{ビームライン・施設建設}
実験開始に向けてビームラインや実験エリアの建設が急務である。これは本職でしか成し遂げられない重要な仕事である。
%実験建設では各グループとの連携が大切である。そのためには,実験全体を把握することとコミュニケーションが大事になってくる。参入後数か月で,共同研究者と積極的にコミュニケーションを取り情報を得るとともに信頼関係を構築する。また,外部の業者や行政との相互理解にも尽力したい。
新しい施設を立ち上げる際には様々な問題が生じる。実際にやってみて初めてわかる問題も多々ある。現場でどれだけ早急に最適なソリューションを提供するかが成果に大きく関わってくる。
そこでは忍耐強さと,何としても計画を進める強い推進力が必要である。ここに私の経験と特性を活かすことができると思っている。


高品質なパルスミュービームが実験成功のカギを握る。
1次陽子加速器の運転スキームから,2次パイオンの生成・捕獲,高品質3次ミュービームまで各担当者と協力して作り上げていきたい。
実験中はビームの強度・安定性・extinction・純度を常時計測し,安定したデータ収集を保障する。
データを解析し,ビームに起因する背景事象の見積もりと統計量の計算を行う。
背景事象のパルス時刻に対する時間分布などを分析することで,その起源や組成を調べたり,必要であればより詳細な解析を可能とする新たな測定や検出器の導入などを速やかに実施する。
Phase Iで得たデータと経験を元に,Phase II設計にフィードバックをかける。

大強度化の進むPhase II用ビームラインでは放射線耐性やメインテナンスのしやすい(もしくはいらない)設計が持続可能な施設として重要となる。
これらの研究領域に対して経験はないが,シミュレーション
を活用した設計などは検出器の設計の経験が生かせるし,経験者から教えてもらいながら,また綿密な議論をしながら一からシステムを組んでいくという作業はこれまでの実験くみ上げと同じであり一つ一つ経験を積みながら成し遂げたい。

J-PARCという世界に誇る施設を拡張して新たな実験エリアを作り上げることはCOMET実験を超えた意味をもち,その経験は今後,新たな施設や実験の計画・建設などに活かすことができるだろう。素粒子実験屋としての技量を大きく発展させることになると期待している。

\paragraph{物理解析}
施設建設・運営にとどまらず,実験における物理解析を主導していきたい。
$\muup^-\mathrm{N}\to\mathrm{e^-N}$過程探索のみならず,物理アウトプットを最大化する指揮を取りたい。Phase IIでは探索できないモードもあるため,Phase Iでしっかりと物理結果をパブリッシュしていくことが重要である。Mu2eに先行して結果を出していくことが重要である一方,競合する実験の研究者と徹底的に議論し,双方の解析や実験理解を共有しながらより良い計画にブラッシュアップしていく。結果をパブリッシュするには系統誤差の理解や様々なデータを用いた検証・確認など,実験準備やR\&Dとは全くことなる綿密な解析が不可欠であることをMEG/MEG IIの経験より学んできた。これまでCOMETを推進してきた研究者とは違った視点でデータや解析を見ることができ,必ずコラボレーションをより良い方向へと船頭できると自負している。複数の物理解析を主導してきた経験を活かしやり遂げたい。


%\paragraph{}
%
%ホスト機関の研究者として国際協力を牽引し,同時にMu2eとの国際競争にも負けないタフな研究者と自らなるとともに,この絶好の機会を活かして若手研究者を育成したい。そのためには,上記項目だけにとどまらず,実験全体を統括する役割(テクニカルコーディネーション,ランコーディネーション,解析コーディネーションなど)を果たしたい。これらの経験を活かしてPhase II実験を指揮する指導的研究者となる。


\subsection{ハドロン施設拡張計画}
ハドロングループの一員としてハドロン施設拡張計画にも積極的に参加したい。
優れた実験施設を何十年にもわたって活用し,そこから創出される物理成果を維持していくには,施設の改善・拡張を段階的に行っていくことが不可欠であり,またそうすることで
当初の目的を達成するのみならず施設の価値を何倍にも高めることができる。
さまざまなレベルに置ける物質の構造究明を可能にする世界からも注目されている実験施設を実現し,物理学の発展に寄与したい。



\subsection{コミュニティへの貢献}
私はこれまでの研究の大半を実験現場であるスイス・PSI研究所に滞在して行ってきた。そこでは様々な側面でホスト機関および現地研究者のサポートを受けてはじめて研究を進めることができた。ホスト機関の重要性を身をもって感じている。本職では,上記のホスト研究者でしかできない研究・役割を全うすることに加え,国内外からの共同研究者の受け入れに力をいれ,実験全体がスムーズに進められることに尽力することで,還元していきたい。

引き続き,学生の指導と研究者の育成にも力を入れたい。COMET実験は最適な教育の場でもありうる。実験が大型化・長期化している現在の素粒子実験研究において,実験の準備から物理結果導出まで一通り経験できる機会は稀有となっている。貴重な経験を通して「生きた」技術を会得させていきたい。
自国でまた母国語で研究できるという貴重な機会を大事にしつつ,そこで閉じこもることなく,国際的な発信力とリーダーシップを持った研究者を育成していきたい。

%より長い目で見れば,現行実験と将来実験の研究を並行して進めることで,物理成果を継続的に出しながら将来計画へとつなげていかねばならない。これは,(自分も含めた)人材育成に不可欠であると考える。
%どのようにこれを実現していくか,自身の推進するプロジェクトの域を超えて,広く考え議論することで日本のそして国際的な素粒子実験コミュニティに貢献していきたい。
%総研大や共同研究の学生のみならず


%
%\renewcommand{\refname}{}
%	\vspace{2zw}
%	\begin{thebibliography}{99}
%		\setlength{\itemsep}{-1pt}
%		\bibitem{matsuoka_RPC-PMT} 松岡広大,「高時間分解能ガス電子増幅型光検出器RPC-PMT用光電面の開発」 新学術領域研究 19H05099.
%		\bibitem{gasPMT}K. Matsumoto \textit{et al.}, ``Ion-feedback suppression for gaseous photomultipliers with micro pattern gas detectors,'' Phys.\ Procedia \textbf{37} (2012) 499--505 
%		\bibitem{oya}A. Oya, ``Development of ultra-low material RPC for background identification in MEG II experiment,''  3rd Int.\ Conf.\ on Charged Lepton Flavor Violation, 2019
%		\bibitem{ILCMRPC} Z. Liu \textit{et\ al.}, ``Multigap Resistive Plate Chamber read out by $1\times1~\mathrm{cm^2}$ pads with the NINO ASIC,'' Nucl.\ Instrum.\ Methods A \textbf{920} (2019) 115--118
%	\end{thebibliography}
%end 研究目的と研究計画	====================

%% p01_purpose_plan_01.tex
\KLEndSubject{F}

%
%%#Split: 02_background  
%%#PieceName: p02_background
%% p02_background_00.tex
\KLBeginSubject{02}{2}{2 本研究の着想に至った経緯など}{1}{F}{}{jsps-subject-header}{jsps-default-header}

%\section{2 本研究の着想に至った経緯など}
%%    <<最大 1ページ>>
%\vspace{-1zw}
%%s03_background
%%begin 本研究の着想に至った経緯など ====================
%\subsection{着想に至った経緯と準備状況}
%		本研究グループはこれまでMEG・MEG II実験においてガンマ線を測定する液体キセノンガンマ線検出器を開発してきた。液体キセノンシンチレーションカロリメータは,高い一様性・放射線耐性・検出器のスケーラビリティなどの点で従来の結晶シンチレーターカロリメータと一線を画し,50~MeV付近のガンマ線に対して高い性能を発揮した。さらに,検出器の改善策を徹底的に研究しMEG II実験へのアップグレードに成功した。
%		%キセノンの発する真空紫外光に感度を持ったSiPMを開発することでアップグレードに成功したが,
%		その一方で,これ以上の大幅性能向上は難しいことも分かってきた。
%		実際,アップグレードにより一様性などの改善は達成したが本質的なエネルギー分解能は850~keVから変わっていない。$\mathrm{e^+}$の分解能が300 keVから80 keVへと改善するのと対照的であり,同じエネルギーに対して分解能は10倍も悪い。ガンマ線測定の難しさを物語っている。
%		次世代の\megc 実験には異なる手法によるガンマ線測定が不可欠であるとの認識に至った。
%		
%		MEG II実験では現在利用可能なミュー粒子ビームの強度を最大限利用するため,さらなる高感度化には新しいミュー粒子源が必要となる。2012年よりPSI研究所において,強度を最大2桁増強する新しいミュー粒子ビームラインの建設計画(HIMB計画)が進められてきた。計画では2025年にビームラインが完成する見込みである。
%		そこで,本研究グループではこのHIMBビームラインを活用した新しい実験の構想に2014年より着手し,検討を進めてきた。前実験(MEGA実験)で採用されたペアスペクトロメータの検討も進めたが,パッシブなコンバーターを用いた設計ではどうしても検出効率とエネルギー分解能の両立という従来からの課題を解決できない。アクティブコンバーターのアイデアは検討当初より持っていたが,ここ2年でMEG II実験用の超軽量・高計数率DLC RPCの開発が進み実用化の目途が見えてきたことと,松岡が開発を進めているガスPMの技術の応用を松岡と議論することで,ACPSの開発という本申請に至った。
%		ACPSは挑戦的な設計であるが,各要素のベースとなる技術は本研究グループおよび協力研究者のこれまでの研究で得られている。
%		
%\subsection{関連する国内外の研究動向と本研究の位置づけ}
%	上記HIMB計画以外にも,米フェルミ研究所の次期加速器計画PIP IIにおいて大強度DCミュー粒子源が検討されており,次世代\megc 実験の検討は日・欧・米で進められている(例えば\cite{quest})。現在進行中の米素粒子物理将来計画策定プロセスでも次世代\megc 実験案が議論されているが,提案されている案はどれも従来のパッシブコンバーターを用いたペアスペクトロメータをガンマ線検出器に想定している。%その性能見積もりは過大評価されており,明らかにシミュレーション等を用いた詳細な検証が欠如していると言わざるを得ない。
%	本研究グループではシミュレーションや過去のMEGA実験の詳細な検証を通して,これまで提案されてきた案ではシミュレーションにおいて重要なプロセスが無視されていることにより性能を過大評価している点を指摘し\cite{JPS},より現実的な検出器設計を検討してきた。本研究でACPSが開発されれば他の案とは一線を画す実験設計が出来上がる。
%	
%	MEG II実験は2024年頃にデータ収集を終了する。その時期に合わせるように本研究で新しい実験技術を確立することで,次世代の実験へとスムーズにつなげていくことが可能となる。
%	より長期スパンで計画されているCOMET/Mu2eやMu3e実験より一桁以上高い物理感度の実験をこれらの実験と同じタイムスケールでおこなうことにより,本研究グループが世界をリードしてきたCLFV実験物理領域をこれからも最先端でリードしていく。本研究はその基盤となるものである。
%	
%	
%	
%\renewcommand{\refname}{}
%	\vspace{0cm}
%	\begin{thebibliography}{99}
%		\setlength{\itemsep}{-1pt}
%		\bibitem{quest} G. Cavoto \textit{et\ al.}, ``The quests for $\meg$ and its experimental limiting factors at future high intensity muon beams,'' Eur.\ Phys.\ J.\ C \textbf{78} (2018) 37 
%		\bibitem{JPS} \underline{\underline{内山雄祐}},\underline{家城佳},\underline{岩本敏幸} 他,「崩壊分岐比感度$10^{-15}$の新しい\megc 探索実験の検討」,日本物理学会2014年秋季大会
%	\end{thebibliography}
%%end 本研究の着想に至った経緯など ====================



%#Split: 99_tail
% hook9 : right before \end{document} ============
 % pieces
\end{document}

