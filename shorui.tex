\documentclass[11pt,a4paper,uplatex,dvipdfmx]{ujarticle} 		% for uplatex
%\documentclass[11pt,a4j,dvipdfmx]{jarticle} 					% for platex
%\documentclass[11pt,pdflatex,a4paper,ja=standard]{bxjsarticle}
\usepackage{amsmath}  
\input{pieces/form00_header} % pieces
\input{pieces/kakenhi7} % pieces
\input{pieces/form01_header} % pieces
%% ===== Global year-dependent definitions for the Kakenhi form ===========
% 基本情報
\newcommand{\研究開始年度}{2020}
\newcommand{\研究開始元号年度}{02}	%令和

\newcommand{\一年目西暦}{2020}
%\newcommand{\2年目西暦}{2021}
%\newcommand{\3年目西暦}{2022}
%\newcommand{\4年目西暦}{2023}
%\newcommand{\5年目西暦}{2024}
%\newcommand{\6年目西暦}{2025}

\newcommand{\一年目}{2}
%\newcommand{\2年目}{3}
%\newcommand{\3年目}{4}
%\newcommand{\4年目}{5}
%\newcommand{\5年目}{6}
%\newcommand{\6年目}{7}

\newcommand{\一年目J}{2}
%\newcommand{\2年目J}{3}
%\newcommand{\3年目J}{4}
%\newcommand{\4年目J}{5}
%\newcommand{\5年目J}{6}
%\newcommand{\6年目J}{7}


 % pieces
%\input{pieces/hook3} % pieces
%#Name: kiban_a
\input{pieces/form04_jsps_headers} % pieces
% form04_kiban_a_header.tex

% ===== Global definitions for the Kakenhi form ======================
% 基本情報
\newcommand{\様式}{}
\newcommand{\研究種目}{素核研22-12}
\newcommand{\研究種目後半}{}
\newcommand{\研究種別}{}
\newcommand{\研究種目header}{\研究種目 応募者氏名 \研究代表者氏名 }

\newcommand{\KLMainFile}{kiban\_a.tex}
\newcommand{\KLYoshiki}{kiban_a_header}

%==========================================================
 % pieces
% ===== Global definitions for the Kakenhi form ======================
% 基本情報
%
%------ 研究課題名  -------------------------------------------
\newcommand{\研究課題名}{象の卵}

%----- 研究機関名と研究代表者の氏名-----------------------
\newcommand{\研究機関名}{東京大学}
\newcommand{\研究代表者氏名}{内山雄祐}
\newcommand{\me}{\underline{Y.~Uchiyama}} 
%---- 研究期間の最終年度 ----------------
\newcommand{\研究期間の最終元号年度}{6}  %令和で,半角数字のみ
%========================================

%% inst_general.tex
%--------------------------------------------------------------------
% For writing instructions
%--------------------------------------------------------------------
\newcommand{\KLInstWOGeneral}[1]{%
	\noindent
 ーー ※留意事項 ーーーーーーーーーーーーーーーーーーーーーーーーーーーーーーーーー\\
		#1\\
 ーーーーーーーーーーーーーーーーーーーーーーーーーーーーーーーーーーーーーーーーーー
}

\newcommand{\KLInst}[1]{%
	\noindent
	\ifthenelse{\equal{#1}{}}{%
 ーー ※留意事項 ーーーーーーーーーーーーーーーーーーーーーーーーーーーーーーーーーー\\
	}{%
 ーー ※留意事項\textcircled{1} ーーーーーーーーーーーーーーーーーーーーーーーーーーーーーーーーー\\
		#1\\
		
	\noindent
 ーー ※留意事項\textcircled{2} ーーーーーーーーーーーーーーーーーーーーーーーーーーーーーーーーー\\
	}
}

\newcommand{\GeneralInstructions}{%
  1.作成に当たっては、研究計画調書作成・記入要領を必ず確認すること。\\
  2.本文全体は11ポイント以上の大きさの文字等を使用すること。\\
  3.各頁の上部のタイトルと指示書きは動かさないこと。\\
  4. 指示書きで定められた頁数は超えないこと。なお、空白の頁が生じても削除しないこと。\\
  \textcolor{red}{5.本留意事項は、研究計画調書の作成時には削除すること。(\texttt{\textbackslash JSPSInstructions}を消す)}\\
 ーーーーーーーーーーーーーーーーーーーーーーーーーーーーーーーーーーーーーーーーーー
}

\newcommand{\PapersInstructions}{%
 ーー ※留意事項 ーーーーーーーーーーーーーーーーーーーーーーーーーーーーーーーーーー\\
1. 研究業績(論文、著書、産業財産権、招待講演等)は、網羅的に記載するのではなく、\\
 本研究計画の実行可能性を説明する上で、その根拠となる文献等の主要なものを適宜記\\
 載すること。\\
2. 研究業績の記述に当たっては、当該研究業績を同定するに十分な情報を記載すること。\\
 例として、学術論文の場合は論文名、著者名、掲載誌名、巻号や頁等、発表年(西暦)、\\
 著書の場合はその書誌情報、など。\\
3. 論文は、既に掲載されているもの又は掲載が確定しているものに限って記載すること。\\
\textcolor{red}{4. 本留意事項は、研究計画調書の作成時には削除すること。
 (\texttt{\textbackslash PapersInstructions}を消す)}\\
 ーーーーーーーーーーーーーーーーーーーーーーーーーーーーーーーーーーーーーーーーーー\\
} % pieces
%\input{pieces/inst_kiban_a} % pieces
% user07_header
% ===== my favorite packages ====================================
% ここに,自分の使いたいパッケージを宣言して下さい。
\usepackage{wrapfig}
%\usepackage{amsmath}
%\usepackage{amssymb}
%\usepackage{txfonts}
\usepackage{newtxtext,newtxmath}
\usepackage{pifont} % 丸数字を使うためのパッケージ
%\usepackage{mb}
%\DeclareGraphicsRule{.tif}{png}{.png}{`convert #1 `dirname #1`/`basename #1 .tif`.png}
\usepackage{lineno}
\usepackage{hyperref}
\usepackage{url}
\usepackage{pxjahyper}
\usepackage{xcolor}
\usepackage{multirow}
\hypersetup{
    colorlinks=true,
    citecolor=blue,
    linkcolor=blue,
    urlcolor=blue,
}
\renewcommand{\emph}[1]{{\sffamily\bfseries{#1}}}

\usepackage{titlesec}
\titleformat*{\section}{\huge\sffamily\bfseries}
\titleformat*{\subsection}{\large\sffamily\bfseries}
%\renewcommand{\thesection}{\textsf{\arabic{section}}}

% ===== my personal definitions ==================================
% ここに,自分のよく使う記号などを定義して下さい。
\newcommand{\klpionn}{K_L \to \pi^0 \nu \overline{\nu}}
\newcommand{\kppipnn}{K^+ \to \pi^+ \nu \overline{\nu}}
%%% new symbols
%\newcommand*{\meg}{\mu\to e \gamma}
%\newcommand*{\megc}{\ifmmode\mu^+ \to e^+ \gamma\else$\mu^+ \to e^+ \gamma$\fi}
\newcommand*{\meg}{\muup\to \mathrm{e} \gammaup}
\newcommand*{\megc}{\ifmmode\muup^+ \to \mathrm{e}^+ \gammaup\else$\muup^+ \to \mathrm{e}^+ \gammaup$\fi}
\newcommand*{\rmd}{\ifmmode\muup^+ \to \mathrm{e}^+ \nuup_\mathrm{e} \bar{\nuup}_\muup \gammaup\else$\muup^+ \to \mathrm{e}^+ \nuup_\mathrm{e} \bar{\nuup}_\muup \gammaup$\fi}
\newcommand*{\mueconv}{\muup^-\mathrm{N \to e^-N}}
\newcommand*{\ee}{\mathrm{e^+e^-}}
%e%% end new symbols


% ===== 欄外メモ ==================
\newcommand{\memo}[1]{\marginpar{#1}}
%\renewcommand{\memo}[1]{}	% 全てのメモを表示させないようにするには,行頭の"%"を消す


%\input{../../sample/simple/contents}	% skip
\input{pieces/hook5} % pieces

\begin{document}
\input{pieces/hook7} % pieces


%#Split: 03_abilities  
%#PieceName: p03_abilities
\input{pieces/p01_purpose_plan_00}
%% p03_abilities_00.tex
\KLBeginSubject{02}{2}{}{2}{F}{}{jsps-subject-header}{jsps-default-header}

\noindent

%begin 研究業績リスト ====================

\section{研究業績一覧}
	
\subsection{学術論文(査読あり) \textcolor{red}{$\bigcirc$}付き項目は主要論文}

	\begin{enumerate}
			\setlength{\itemsep}{-1pt}

\bibitem{YU:shield}
        \me, Y.~Fukao, M.~Shunsuke, S.~Mihara, 
        ``Radiation Shielding System for the COMET Pion Capture Solenoid'', 
        JPS Conf.\ Proc. (in press), \textbf{ 筆頭著者}, 研究責任者.
\bibitem{YU:commissioning}
         Y.~Fukao, ..., \me\ \textit{et al.}, 
        ``Construction and Beam Commissioning for the COMET Experiment'', 
        JPS Conf.\ Proc. (in press).
\bibitem{YU:target}
         S.~Makimura, ..., \me\ \textit{et al.}, 
        ``Recent Upgrade on Muon Target at J-PARC'', 
        JPS Conf.\ Proc. (in press).

\bibitem{acps_2025}
R. Sakakibara, ..., \me, {\it et al.},
``Development of an active converter pair spectrometer for the future search for \megc,'' Nucl.\ Instrm.\ Methods A {\bf 1082} (2026) 170961,
\href{https://doi.org/10.1016/j.nima.2025.170961}{doi:10.1016/j.nima.2025.170961}
\bibitem{x17_2025} 
K. Afanaciev, ..., \me, {\it et al}.\ (MEG II Collaboration),
  ``Search for the X17 particle in $^{7}\textrm{Li}(\textrm{p},\textrm{e}^+ \textrm{e}^{-}) ^{8}\textrm{Be}$ processes  the MEG II detector,''
  Eur.\ Phys.\ J.\ C {\bf 85}  (2025) 763,
  \href{https://doi.org/10.1140/epjc/s10052-025-14345-0}{doi:10.1140/epjc/s10052-025-14345-0}.

\bibitem{cdch_2024}
A.~M. Baldini, ..., \me, {\it et al.},
``Performances of a new generation tracking detector: the MEG II cylindrical drift chamber,''
  Eur.\ Phys.\ J.\ C {\bf 84}  (2024) 473,
\href{https://doi.org/10.1140/epjc/s10052-024-12711-y}{doi:10.1140/epjc/s10052-024-12711-y}.

\bibitem{meg2_2024} 
\textcolor{red}{$\bigcirc$} 
K. Afanaciev, ..., \me, {\it et al}.\ (MEG II Collaboration),
  ``A search for \megc\ with the first dataset of the MEG II experiment,''
  Eur.\ Phys.\ J.\ C {\bf 84}  (2024) 216,
  \href{https://doi.org/10.1140/epjc/s10052-024-12416-2}{doi:10.1140/epjc/s10052-024-12416-2},
  引用数XXX.

\bibitem{meg2detector_2024} 
\textcolor{red}{$\bigcirc$} 
K. Afanaciev, ..., \me, {\it et al}.\ (MEG II Collaboration),
  ``Operation and performance of the MEG II detector,''
  Eur.\ Phys.\ J.\ C {\bf 84}  (2024) 190,
  \href{https://doi.org/10.1140/epjc/s10052-024-12415-3}{doi:10.1140/epjc/s10052-024-12415-3},
  \textbf{協同筆頭著者}, 引用数XXX.

\bibitem{ptc_2024}
T. Yonemoto, ..., \me, {\it et al.},
``The latest performance and refurbishment of the pixelated Timing Counter (pTC) in the MEG II experiment,'' Nucl.\ Instrm.\ Methods A {\bf 1068} (2024) 169720,
\href{https://doi.org/10.1016/j.nima.2024.169720}{doi:10.1016/j.nima.2024.169720},
研究責任者, 筆頭著者T.~Yonemoto(大学院生)の指導.

\bibitem{alignment_2024}
A. Ventrurini, ..., \me, {\it et al.},
``Alignment of the MEG II cylindrical drift chamber,'' Nucl.\ Instrm.\ Methods A {\bf 1067} (2024) 169680,
\href{https://doi.org/10.1016/j.nima.2024.169680}{doi:10.1016/j.nima.2024.169680}.

\bibitem{cdch_2023}
M.~Chiappini, ..., \me, {\it et al.},
``The Cylindrical Drift Chamber of the MEG II experiment,'' Nucl.\ Instrm.\ Methods A {\bf 1047} (2023) 167740,
\href{https://doi.org/10.1016/j.nima.2022.167740}{doi:10.1016/j.nima.2022.167740}.

\bibitem{ptc_2023}
P.~W.~Cattaneo, G.~Boca, M.~De~Gerone, M.~Francesconi, L.~Galli, F.~Gatti, W.~Ootani, M.~Rossella, \me, Y.~Uchiyama, M.~Usami, T.~Yonemoto,
``Operational results with the pixelated Time Detector of MEG II experiment during the first year of physics data taking,'' Nucl.\ Instrm.\ Methods A {\bf 1046} (2023) 167751,
\href{https://doi.org/10.1016/j.nima.2022.167751}{doi:10.1016/j.nima.2022.167751},
研究責任者.
\bibitem{trigger_2023}
M.~Francesconi, ..., \me, {\it et al.},
``The trigger system for the MEG II experiment,'' Nucl.\ Instrm.\ Methods A {\bf 1046} (2023) 167736,
\href{https://doi.org/10.1016/j.nima.2022.167736}{doi:10.1016/j.nima.2022.167736}.

\bibitem{xec_2023}
T.~Iwamoto, ..., \me, {\it et al.},
``The liquid xenon detector for the MEG II experiment to detect 52.8 MeV $\gamma$ with large larea VUV-sensitive MPPCs,'' Nucl.\ Instrm.\ Methods A {\bf 1046} (2023) 167720,
\href{https://doi.org/10.1016/j.nima.2022.167720}{doi:10.1016/j.nima.2022.167720}.


%\bibitem{himb}
%M.~Abe, ..., \me, {\it et al.},
%``Science case for the new High-Intensity Muon Beam HIMB at PSI,''
%arXiv:2111.05788, 
%%\href{https://arxiv.org/abs/2111.05788}{arXiv:2111.05788} 
%\href{https://doi.org/10.48550/arXiv.2111.0578}{doi:10.48550/arXiv.2111.0578} 
%(2021), 将来実験構想.

\bibitem{symmetry}
A.~M.~Baldini, ..., \me, {\it et al.} (MEG II Collaboration),
``The search for \megc\ with $10^{-14}$ sensitivity: the upgrade of the MEG experiment,''
Symmetry {\bf 13} (2021) 1591
\href{https://doi.org/10.3390/sym13091591}{doi:10.3390/sym13091591}, 
\textbf{協同筆頭著者}.

\bibitem{radiation}
G.~Boca, P.~W.~Cattaneo, M.~De~Gerone, F.~Gatti, M.~Nakao, M.~Nishimura, W.~Ootani, M.~Rossella, \me, M.~Usami, K.~Yanai, ``Timing resolution of a plastic scintillator counter read out by radiation damaged SiPMs connected in series,'' Nucl.\ Instrum.\ Methods A {\bf 999} (2021) 165173,
\href{https://doi.org/10.1016/j.nima.2021.165173}{doi:10.1016/j.nima.2021.165173},
研究責任者, 筆頭著者M.~Usami(大学院生)の指導.

\bibitem{MEx2G}
A.~M.~Baldini, ..., \me, {\it et al}.\ (MEG Collaboration),
``Search for lepton flavour violating muon decay mediated by a new light particle in the MEG experiment,'' Eur.\ Phys.\ J.\ C \textbf{80} (2020) 858,
\href{https://doi.org/10.1140/epjc/s10052-020-8364-1}{doi:10.1140/epjc/s10052-020-8364-1},
コラボレーション内評価・編集委員,  筆頭著者M.~Nakao(大学院生)の指導.

\bibitem{vci2019}
M.~Nishimura, ..., \me, {\it et al}., 
``Full system of positron timing counter in MEG II having time resolution below 40 ps with fast plastic scintillator readout by SiPMs,''
Nucl.\ Instrum.\ Methods A \textbf{958} (2020) 162785,
\href{https://doi.org/10.1016/j.nima.2019.162785}{doi:10.1016/j.nima.2019.162785},
研究責任者,  筆頭著者M.~Nishimura(大学院生)の指導.

\bibitem{laser} 
  G.~Boca, P.~W.~Cattaneo, M.~De~Gerone, M.~Francesconi, L.~Galli, F.~Gatti, J.~Koga,
  M.~Nakao, M.~Nishimura, W.~Ootani, M.~Rossella, \me, M.~Usami, K.~Yanai, K.~Yoshida,
  ``The laser-based time calibration system for the MEG II pixelated Timing Counter,''
  Nucl.\ Instrum.\ Methods A {\bf 947}  (2019) 162672,
  \href{https://doi.org/10.1016/j.nima.2019.162672}{doi:10.1016/j.nima.2019.162672},
  研究責任者,  筆頭著者M.~Nakao(大学院生)の指導.
	 
\bibitem{Baldini:2018nnn}  
A.\,M.~Baldini, ..., \me, {\it et al}.\ (MEG II Collaboration),
  ``The design of the MEG II experiment,''
  Eur.\ Phys.\ J.\ C {\bf 78}  (2018) 380,
  \href{https://doi.org/10.1140/epjc/s10052-018-5845-6}{doi:10.1140/epjc/s10052-018-5845-6},
  \textbf{協同筆頭著者}.

\bibitem{vci2016} 
  \me, G.~Boca, P.~W.~Cattaneo, M.~De~Gerone F.~Gatti, M.~Nakao, M.~Nishimura, W.~Ootani, G.~Pizzigoni, M.~Rossella, M.~Simonetta, K.~Yoshida, 
  ``30-ps time resolution with segmented scintillation counter for MEG II,''
  Nucl.\ Instrum.\ Methods A {\bf 845} (2017) 507--510,
  \href{https://doi.org/10.1016/j.nima.2016.06.072}{doi:10.1016/j.nima.2016.06.072},
  \textbf{筆頭著者}, 研究責任者.

\bibitem{btf2014} 
  P.~W.~Cattaneo, M.~De~Gerone F.~Gatti, M.~Nishimura, W.~Ootani, M.~Rossella, S.~Shirabe, \me, 
  ``Time resolution of time-of-flight detector based on multiple scintillation counters readout by SiPMs,''
  Nucl.\ Instrum.\ Meth.\ A {\bf 828} (2016) 191--200,
  \href{http://dx.doi.org/10.1016/j.nima.2016.05.038}{doi:10.1016/j.nima.2016.05.038},
  \textbf{筆頭著者}, 研究責任者.

\bibitem{TheMEG:2016wtm} 
\textcolor{red}{$\bigcirc$} 
  A.\,M.~Baldini, ..., \me, {\it et al}.\ (MEG Collaboration),
  ``Search for the lepton flavour violating decay $\megc$ with the full dataset of the MEG experiment,''
  Eur.\ Phys.\ J.\ C {\bf 76} (2016) 434,
  \href{https://doi.org/10.1140/epjc/s10052-016-4271-x}{doi:10.1140/epjc/s10052-016-4271-x}
  引用数881, Sect.\ 3.1(ガンマ線再構成), 4.4(崩壊探索背景事象), 4.6(規格化)執筆担当.

\bibitem{polarization}
  A.~M.~Baldini, ..., \me, {\it et al}.\ (MEG Collaboration),
  ``Muon polarization in the MEG experiment: predictions and measurements,''
  Eur.\ Phys.\ J.\ C {\bf 76} (2016) 223,
  \href{http://dx.doi.org/10.1140/epjc/s10052-016-4047-3}{doi:10.1140/epjc/s10052-016-4047-3}.

\bibitem{Adam:2013gfn} 
\textcolor{red}{$\bigcirc$}
  A.~M.~Baldini, ..., \me, {\it et al}.\ (MEG Collaboration),
  ``Measurement of the radiative decay of polarized muons in the MEG experiment,''
  Eur.\ Phys.\ J.\ C {\bf 76} (2016) 108,
  \href{https://doi.org/10.1140/epjc/s10052-016-3947-6}{doi:10.1140/epjc/s10052-016-3947-6},
\textbf{筆頭著者}.

\bibitem{Ootani:2015cia} 
  W.~Ootani, ..., \me,  {\it et al}., %K.~Ieki, T.~Iwamoto, D.~Kaneko, T.~Mori, S.~Nakaura, M.~Nishimura, S.~Ogawa, R.~Sawada, N.~Shibata, \me, K.~Yoshida, K.~Sato, R.~Yamada,
  ``Development of deep-UV sensitive MPPC for liquid xenon scintillation detector,''
  Nucl.\ Instrum.\ Methods A {\bf 787} (2015) 220--223,
  \href{https://doi.org/10.1016/j.nima.2014.12.007}{doi:10.1016/j.nima.2014.12.007}.

  \bibitem{tc-single} 
  \textcolor{red}{$\bigcirc$} 
  P.~W.~Cattaneo, M.~De~Gerone F.~Gatti, M.~Nishimura, W.~Ootani, M.~Rossella, \me, 
  ``Development of high precision timing counter based on plastic scintillator with SiPM readout,''
  IEEE Trans.\ Nucl.\ Sci.\ {\bf 61} (2014) 2657--2666,
  \href{http://dx.doi.org/10.1109/TNS.2014.2347576}{doi:10.1109/TNS.2014.2347576},
  \textbf{筆頭著者}, 研究責任者, 引用数XXX.

\bibitem{rmd1}
  J.~Adam, ..., \me, {\it et al.},
  ``Measurement of inner Bremsstrahlung in polarized muon decay with MEG,''
  Nucl.\ Phys.\ B Proc.\ Suppl.\ \textbf{248--250} (2014) 108--111,
  \href{https://doi.org/10.1016/j.nuclphysbps.2014.02.019}{doi:10.1016/j.nuclphysbps.2014.02.019},
  \textbf{筆頭著者}.

\bibitem{Adam:2013vqa} 
  J.~Adam, ..., \me,  {\it et al.},
  ``The MEG detector for \megc\ decay search,''
  Eur.\ Phys.\ J.\ C {\bf 73} (2013) 2365,
  \href{https://doi.org/10.1140/epjc/s10052-013-2365-2}{doi:10.1140/epjc/s10052-013-2365-2}.

\bibitem{Adam:2013mnn} 
  J.~Adam, ..., \me,  {\it et al}.\ (MEG Collaboration),
  ``New constraint on the existence of the \megc\ decay,''
  Phys.\ Rev.\ Lett.\  {\bf 110} (2013) 201801,
  \href{https://doi.org/10.1103/PhysRevLett.110.201801}{doi:10.1103/PhysRevLett.110.201801}.

%\bibitem{upgrade_proposal}
% A.~M.~Baldini, ..., \me, {\it et al}.\ (MEG Collaboration),
% ``MEG upgrade proposal,'' Research proposal to PSI R-99-05.2 (2012),
% \href{https://doi.org/10.48550/arXiv.1301.7225}{doi:10.48550/arXiv.1301.7225},
% 引用数346, Sect.\ VI-B(タイミングカウンター), XIII-D(シリコンバーテックス検出器)執筆担当.

\bibitem{TC2}
M.~De~Gerone, ..., \me, {\it et al}. %, S.~Dussoni, K.~Fratini, F.~Gatti, R.~Valle, G.~Boca, P.~W.~Cattaneo, R.~Nardo, 
%M.~Rossella, L.~Galli, M.~Grassi, D.~Nicolo, \me, D.~Zanello, 
``Development and commissioning of the Timing Counter for the MEG experiment,''
IEEE Trans.\ Nucl.\ Sci.\ \textbf{59} (2012) 379--388,
\href{http://dx.doi.org/10.1109/TNS.2012.2187311}{doi:10.1109/TNS.2012.2187311}.

\bibitem{meg2011}
J.~Adam, ..., \me, ,  {\it et al}.\ (MEG Collaboration),
``New limit on the lepton-flavor-violating decay \megc,''
Phys.\ Rev.\ Lett.\ \textbf{107} (2011) 171801,
\href{http://dx.doi.org/10.1103/PhysRevLett.107.171801}{doi:10.1103/PhysRevLett.107.171801}.

\bibitem{cw}
J.~Adam, ..., \me, {\it et al}.\ (MEG Collaboration),
``Calibration and monitoring of the MEG experiment by a proton beam from a Cockcroft--Walton accelerator,''
Nucl.\ Instrum.\ Methods A \textbf{641} (2011) 19--32,
\href{https://doi.org/10.1016/j.nima.2011.03.048}{doi:10.1016/j.nima.2011.03.048}.

\bibitem{TC1}
M.~De~Gerone, ..., \me, {\it et al}., %S.~Dussoni, K.~Fratini, F.~Gatti, R.~Valle, G.~Boca, P.~W.~Cattaneo, 
%M.~Rossella, R. Nardo, A.~Papa, G.~Signorelli, G.~Cavoto, G.~Pirreda, F.~Renga, C.~Voena, \me, 
``The MEG timing counter calibration and performance,''
Nucl.\ Instrum.\ Methods A \textbf{638} (2011) 41--46,
\href{https://doi.org/10.1016/j.nima.2011.02.044}{doi: 10.1016/j.nima.2011.02.044}.

\bibitem{meg2010}
J.~Adam, ..., \me, ,  {\it et al}.\ (MEG Collaboration),
``A limit for the $\meg$ decay from MEG experiment,''
Nucl.\ Phys.\ B \textbf{834} (2010) 1--12,
\href{https://doi.org/10.1016/j.nuclphysb.2010.03.030}{doi:10.1016/j.nuclphysb.2010.03.030},
博士論文内容相当.

\bibitem{gamma_rec}
\me, ``Gamma ray reconstruction with liquid xenon gamma ray calorimeter for the MEG experiment,''
Nucl.\ Instrum.\ Methods A \textbf{617} (2010) 118--119,
\href{https://doi.org/10.1016/j.nima.2009.09.100}{doi:10.1016/j.nima.2009.09.100},
\textbf{筆頭著者}.

\bibitem{LXeStorage}
T.~Iwamoto, ..., \me,  {\it et al}.,  %, R.~Sawada, T.~Haruyama, S.~Mihara, T.~Doke, Y.~Hisamatsu, K.~Kasami, A.~Maki, T.~Mori, H.~Natori, H.~Nishiguchi, Y.~Nishimura, W.~Ootani, K.~Terasawa, \me, S.~Yamada,
``Development of a large volume zero boil-off liquid xenon storage system for muon rare decay experiment (MEG),''
Cryogenics \textbf{49} (2009) 254--258,\\
\href{https://doi.org/10.1016/j.cryogenics.2008.09.003}{doi:10.1016/j.cryogenics.2008.09.003}.

\bibitem{LXePump}
S.~Mihara, ..., \me,  {\it et al}., %T.~Haruyama, T.~Iwamoto, \me, W.~Ootani, K.~Kasami, R.~Sawada, K.~Terasawa, T.~Doke, H.~Natori, H.~Nishiguchi, A.~Maki, T.~Mori, S.~Yamada,
``Development of a method for liquid xenon purification using a cryogenic centrifugal pump,''
Cryogenics \textbf{46} (2006) 688--693,
\href{https://doi.org/10.1016/j.cryogenics.2006.04.003}{doi:10.1016/j.cryogenics.2006.04.003}.

\end{enumerate}



\subsection{国際会議発表, 国内会議発表}
\begin{enumerate}
\setcounter{enumi}{38}
	\setlength{\itemsep}{-1pt}
	
	\bibitem{hql2025} 
 	 \me, ``Muon LFV/LFU measurements at J-PARC, PSI, FNAL,''
  	17th International Conference on Heavy Quarks and Leptons, Beijing, Sep. 2025, \textbf{招待講演}.

	\bibitem{phits2025-2} 
  	\underline{内山雄祐},  	``J-PARC COMET実験における放射線計算,''
  	第2回 EGS5-Geant4-PHITS合同研究会, 筑波,  2025年5月.
  	\bibitem{jps2025s} 
  	\underline{内山雄祐},  	``COMET実験パイオン生成システムの遮蔽設計,''
  	日本物理学会2025年春季大会, online,  2025年3月.
	\bibitem{phits2025-1} 
  	\me, ``Radiation study for the pion production system of the COMET experiment at J-PARC,''
 	 PHITS Workshop and Intermediate Course, Tokai, Feb. 2025.
  	\bibitem{fpws2024} 
  	\underline{内山雄祐},  	``レプトンフレーバー物理,''
  	Flavor Physics Workshop 2024, 蒲郡,  2024年12月, \textbf{招待講演}.
  	\bibitem{jparc2024}
  	\me, ``Radiation ShieldingSystem for the COMET Pion Capture Solenoid,''
  	4th J-PARC Sympojium, Oct. 2024.
	\bibitem{hokkaido2024} 
 	 \me, ``Search for charged lepton flavor violation with muons,''
  	Hokkaido Workshop on Particle Physics at Crossroads, Sapporo, Mar. 2024, \textbf{招待講演}.
	\bibitem{kekeph2023} 
 	 \me, ``High energy physics with muons,''
  	KEK Theory Meeting on Particle Physics Phenomenology, Tsukuba, Nov. 2023, \textbf{招待講演}.
	\bibitem{twoinfinities2023} 
 	 \me, ``Particle Physics with Muons,''
  	Int.\ Conf.\ on the Physics of the Two Infinities, Kyoto, Mar. 2023, \textbf{招待講演}.
  	\bibitem{jps2023s} 
  	\underline{内山雄祐},  	``\megc 探索実験MEG II Run 2022のまとめと今後の展望,''
  	日本物理学会2023年春季大会, online,  2025年3月.
  	\bibitem{jps2022a} 
  	\underline{内山雄祐},  	``MEG II 陽電子スペクトロメータにおける機械学習を活用したヒット再構成の改善,''
  	日本物理学会2022年秋季大会,  岡山,  2022年9月.
  	\bibitem{jps2021a} 
  	\underline{内山雄祐},  	``MEG II 実験:2021 年エンジニアリングランの現状と計画,''
  	日本物理学会2021年秋季大会,  online,  2021年9月.
  	\bibitem{jps2021s} 
  	\underline{内山雄祐},  	``機械学習を活用した高計数率ドリフトチェンバーのヒット再構成,''
  	日本物理学会第76回年次大会,  online,  2021年3月.
	\bibitem{aps2019} 
 	 \me, ``The MEG II experiment in serach of $\meg$,''
  	APS Division of Particles \& Fields Meeting, Boston, Jul. 2019.
  	
%  	\bibitem{jps2019a} 
%  	\underline{内山雄祐},  	``\megc 探索実験 MEG II 現状と今後の見込み,''
%  	日本物理学会2019年秋季大会,  山形,  2019年9月.

	\bibitem{icasipm2018} 
 	 \me, ``Large scale characterization of SiPMs in the MEG II experiment,''
  	Int. Conf. on the Advancement of Silicon Photomultipliers, Schwetzingen, Jun. 2018, \textbf{招待講演}.

%  	\bibitem{jps2018s} 
%  	\underline{内山雄祐},  	``\megc 探索実験 MEG II 2018年度の展望,''
%  	日本物理学会第73回年次大会,  野田,  2018年3月.
	\bibitem{vci2016} 
 	 \me, ``30-ps Time Resolution with Segmented Scintillation Counter for MEG II,''
  	14th Vienna Conf. on Instrumentation, Vienna, Feb. 2016.
	\bibitem{ckm2014} 
 	 \me, ``Lepton Flavor Violating Muon Processes,''
  	8th Int. Workshop on the CKM Unitarity Triangl, Vienna, Sep. 2014, \textbf{招待講演}.
	\bibitem{ieee2013} 
 	 \me, ``High Precision Measurement of Positron Time in MEG Upgrade,''
  	IEEE NSS/MIC?RTSD, Seoul, Oct. 2013.
	\bibitem{eps2013} 
 	 \me, ``Upgrade of MEG experiment,''
  	European Physical Society Conf. on High Energy Physics, Stockholm, Jul. 2013.


  	\bibitem{jps2013s} 
  	\underline{内山雄祐},  	``荷電レプトンフレーバー非保存探索による LHC 時代の素粒子物理シンポジウム: DCミューオンビームによる cLFV 探索,''
  	日本物理学会第68回年次大会,  広島,  2013年3月, \textbf{招待講演}.
  	\bibitem{jps2011a} 
  	\underline{内山雄祐},  	``Analysis of the First MEG Physics Data to Search for the Decay \megc,''
  	日本物理学会2011年秋季大会,  弘前,  2011年9月, \textbf{招待講演}.


\end{enumerate}

\subsection{総説・解説}
\begin{enumerate}
\setcounter{enumi}{59}
	\setlength{\itemsep}{-1pt}
 \bibitem{cern_courier} 
  A. Papa, F. Renga, \me,
  ``Hunting the muon's forbidden decay,''
  CERN Courier May/June (2019) 45--47
 \bibitem{highenergy} 
  家城佳, \underline{内山雄祐},
  ``MEG II 実験 ---分岐比$10^{-14}$ 台での$\megc$崩壊探索---,''
  高エネルギーニュース {\bf 37} No.\,1 (2018) 1--10
\end{enumerate}

\vspace{-8mm}
\subsection{著書}
\vspace{-3mm}
なし
\vspace{-8mm}
\subsection{特許}
\vspace{-3mm}
なし
\vspace{-4mm}

\subsection{全学術論文の被引用回数の合計, h-index}
\noindent
総被引用回数: 1,895, h-index: 16 (Scopus, 2025年9月28日付)

\newpage


%end 研究業績リスト ====================

\section{競争的研究資金および外部研究資金の獲得リスト}
\subsection{科学研究費補助金}
\begin{itemize}
  \setlength{\parskip}{0cm} % 段落間
  \setlength{\itemsep}{0cm} % 項目間
\item 基盤研究(S), 「高分解能キセノン測定器と大強度パイ中間子ビームによるレプトン普遍性破れの精密検証」, 2024年4月--2029年3月, 分担, 森俊則, 分担額:6,500千円
\item 新学術領域研究(ニュートリノで拓く素粒子と宇宙)公募研究, 「極低物質量・高計数率飛跡検出器で挑む荷電レプトンフレーバーの破れの探索」, 2021年4月--2023年3月, 代表, 総額:6,370千円
\item 基盤研究(A), 「高分解能大型液体キセノン測定器によるレプトン普遍性の破れの精密検証」, 2020年4月--2024年3月, 分担, 森俊則, 分担額:4,800千円
\item 若手研究(B),  「ミュー粒子稀崩壊探索実験のさらなる高輝度化に向けたソフトウェアトリガーの開発」, 2017年4月--2019年3月, 代表, 総額:4,030千円
\item 特別研究員奨励費 「ミュー粒子稀崩壊探索実験による超対称性理論の検証」研究代表者2007年4月--2009年3月, 代表, 総額:1,800千円
\end{itemize}

\subsection{受託研究費}
なし
\subsection{その他の競争的資金}
なし

	
\newpage

\section{研究実績および着任後の研究計画}
\subsection{研究実績}

私はこれまで一貫して,レプトンフレーバーに着目し,超高エネルギー物理の解明を目指した素粒子実験の研究をしてきた。
%標準理論では禁止されている,荷電レプトンのフレーバーを破るミュー粒子の稀崩壊事象を探索し,超対称大統一理論やニュートリノ質量の背後にある物理を通したバリオン数生成など,根源的な宇宙の謎の解明に挑んできた。
おもな実績は,
\begin{itemize}
  \setlength{\parskip}{0cm} % 段落間
  \setlength{\itemsep}{0cm} % 項目間
\item	\emph{MEG実験の完遂による世界最高感度での\megc 崩壊探索の遂行}
\item	\emph{MEG II実験の考案から解析までやり遂げ世界最高感度での\megc 崩壊探索を実現}
\item       \emph{J-PARCにおける施設建設・運営}
\item       \emph{COMET実験開始に向けた戦略的展開}
\end{itemize}
である。
国際共同実験であるMEG・MEG IIを立ち上げから推進してきた中核メンバーであり,考案・開発・実現・結果出版の全過程で中心的役割を果たしながら2サイクル経験し,スイスに長期間常駐しながら様々な問題を現場で一つずつ解決することでいずれも成功させた。
%革新的検出器を開発し\cite{Adam:2013vqa},MEG実験ではそれまでの上限値を30倍更新する新しい上限値を設定した\cite{TheMEG:2016wtm}。
%MEG II実験はさらにその感度を10倍更新する計画で\cite{upgrade_proposal, Baldini:2018nnn},測定器の開発・制作を完了し,ついに昨年度,最初の物理データ取得に成功した\cite{cdch_2023,ptc_2023,trigger_2023,xec_2023}。
MEG IIの次の実験としてJ-PARC COMET実験が重要であると考え,実験開始に向けて尽力している。

\vspace{-1zw}
\subsubsection{MEG実験 (2004 -- 2016)}
レプトンフレーバー保存則を破るミューオン崩壊\megc\ を探索するMEG実験において\emph{液体キセノン検出器を用いたガンマ線の解析を責任者として担当}した。波形解析手法と較正手法を確立し,検出効率64\%,エネルギー分解能1.6\%,時間分解能63~psを達成した\cite{gamma_rec, TheMEG:2016wtm}。
%検出器の高い性能を発揮するには,較正が最も大事である。
様々な較正データを統合的に分析し,非一様性およびエネルギースケールの系統不確かさを0.3\%に抑えることにも成功した。これらの業績は\cite{jps, jahep, pisa2009}において高く評価された。
%新しい波形解析アルゴリズムはより高いレートでの実験を可能にするもので,MEG II実験における高レートビーム下でもパイルアップを分離することで性能を落とすことなく液体キセノン検出器を運転できることを実証した。
また,以下の物理解析を主導し物理結果を出版した。
\vspace{-1zw}
\paragraph{\megc 崩壊探索}
検出器の場所による応答の違いや変数間の相関を尤度関数に組み込むことで探索感度を最大限に高めた解析手法を開発した。当時の記録を30倍更新する上限値$\mathcal{B}(\megc) < 4.2×10^{-13}   (90\%\ \mathrm{C.L.})$
%\begin{equation}
%\mathcal{B}(\megc) < 4.2×10^{-13} \quad  (90\%\ \mathrm{C.L.})
%\end{equation}
を与えた\cite{TheMEG:2016wtm}。
一連の分岐比上限値更新結果\cite{TheMEG:2016wtm,Adam:2013mnn,meg2011}は合計被引用数2200件を超え,素粒子物理学における大きなインパクトを与えた。
\vspace{-1zw}
\paragraph{放射ミシェル崩壊の解析}
放射ミシェル崩壊\rmd は\megc の背景事象となる一方で,$\mathrm{e^+}$--$\gammaup$間の時間較正源として活用できる。さらに,その崩壊確率と分布は電弱相互作用で正確に計算できることに着目し,独自にデータを解析して実験全体の正当性を実証した。この崩壊を用いたミュー粒子数規格化因子および偏極度の新しい測定手法を確立し\cite{rmd1},\cite{clfv2013}で評価された。%7\%を計上していた規格因子不確かさを4\%に抑えることに成功した。
最高精度の測定を実現し\cite{Adam:2013gfn},結果はParticle Data Bookにも掲載されている。
\vspace{-1zw}
\paragraph{新しい崩壊モード$\muup^+\to\mathrm{e}^+\mathrm{X},\mathrm{X}\to\gammaup\gammaup$}
このモード独自の2光子再構成手法やキネマティクスを用いた事象再構成手法を大学院生とともに開発。データに基づいた背景事象推定や系統不確かさの取り扱いなど統計手法を主導した。このモードの世界最初の探索結果を出版した\cite{MEx2G}。

\vspace{-1zw}
\subsubsection {MEG II実験 (2012 -- 現在)}
%\begin{wrapfigure}[13]{r}{9cm}
%	\centering
%	 \vspace{-0zw}
%	\includegraphics[width=1.0\linewidth]{figs/pTCPic}
%	 \caption{MEG IIタイミングカウンターと開発を担当した大学院生(2020年2月,本人・中央)}
%	\label{fig:pTCPic}         
%\end{wrapfigure}


2012年よりアップグレード実験MEG II[\href{https://arxiv.org/abs/1301.7225}{1301.7225}被引用数375]を推進してきた。陽電子の時間を高精度で計測する\emph{細分型タイミングカウンターの開発を責任者としてイタリアグループとの国際協力を構築・主導}し進めた。SiPMの独自の読出し・解析手法を確立し\cite{tc-single},35~psの時間分解能を実現する検出器を完成させた\cite{laser, vci2016}。
%2015年よりMEG IIビームを用いたパイロットランを他の検出器に先駆け実施し,読み出し回路やトリガー・DAQの開発を推し進めるなど,実験全体を先駆してきた。%17年に実機を完成させた\cite{vci2019}。
開発した技術は原著論文としてまとめ\cite{radiation, laser, btf2014, tc-single},国内外の学会で広く報告・議論をしてきた。その結果,PANDA TOF,Mu2e ECAL, COMET CTH, J-PARC E42/45 ホドスコープ,ハイパー核寿命測定実験 TDLなど素粒子原子核実験の多くの測定器に波及し,関連技術のスタンダードとなっている。

実験成功のカギとなると判断し,2019年よりドリフトチェンバーの解析にも着手した。他に類を見ない高計数率下での高効率解析手法として\emph{深層学習を活用した全く新しい波形解析手法を開発し}実現した\cite{cdch_2024,meg2detector_2024,jps2022a}。応募者の経験と技術,洞察力を駆使したこの成果なくして今のMEG II実験の成功はなかったといえる。

\emph{コンピューティング・ソフトウェアコーディネーター}も兼任しており,%手研究(B)では大量のデータをリアルタイムで解析することで,信号事象を取りこぼすことなくデータ量を削減する手法とシステムを構築した。
解析・シミュレーションコードの開発を指揮している。また,PSI計算機グループと共同で専用の計算機システムを構築し運用している。オンラインからアーカイブまでの自動データ転送システムやオフライン解析の自動化システムを開発し実験の安定運転に貢献している。コンテナ技術を用いた解析環境の可用性の拡張・保存や,深層学習を解析に取り入れるための統一解析フレームワークの開発なども行った\cite{meg2detector_2024}。

2021年より物理データ収集を開始し,検出器の保守・較正,データ収集の最適化を進めながらデータを取得している。\emph{MEGを超えた探索感度を実現し},これまでに分岐比上限値$\mathcal{B}(\megc) < 1.5×10^{-13}   (90\%\ \mathrm{C.L.})$を達成した[\href{https://arxiv.org/abs/2504.15711}{2504.15711}]。これはあらゆる粒子の崩壊分岐比に対する最も小さな上限値となっており,様々な新物理理論に厳しい制限を与える結果となっている。

\vspace{-1zw}
\subsubsection{J-PARCハドロン実験施設およびCOMET実験 (2023 -- 現在)}
COMET実験はレプトンフレーバー保存則を破る過程の一つである$\muup$--e転換過程を探索する実験計画でJ-PARCで準備が進められている。
%\noindent
%{\bf COMETビームラインおよびビーム室の設計と建設}\\
%MEG IIの次の実験としてJ-PARC COMET実験が重要であると考え,2023年よりCOMET実験に参画している。
パルス陽子ビームから大強度ミューオンビームを生成する施設がCOMET実験の核であると考え,J-PARCハドロン実験施設内COMETビーム室の遮蔽設計の責任者として2023年より建設に携わり,以下の項目に取り組んできた。
\begin{itemize}%[topsep=3pt, itemsep=3pt, parsep=0pt]
\setlength{\itemsep}{0cm}
	 \item 超伝導捕獲ソレノイド内部遮蔽体の設計・制作 \cite{YU:shield,jps2025s,jparc2024}。
    \item ビーム室・実験室の放射線計算,標的やコイルへの損傷評価\cite{phits2025-2,phits2025-1}。
    \item ビームコミッショニング (Phase-$\alpha$) データ解析\cite{YU:commissioning}。
\end{itemize}
構造計算・CAD・CAEなどの基本的な工学設計技術,真空技術の習得やクレーン運転および玉掛けの資格を取得するなど施設建設・運営に必要な技術の習得にも務めてきた。
さらに,電気安全講習や体感型安全講習などに積極的に参加し,加速器施設における安全を学び日々の研究現場に生かしている。
これまで交流のなかった原子力分野の研究者と放射線計算・遮蔽設計で積極的な交流を持ち,お互いの分野の解析手法やニーズの意見交換を通し国際的に注目を集めているソフトウェアPHITSの開発に貢献している。
J-PARCという世界に誇る施設を拡張して新たな実験施設を作り上げることはCOMET実験を超えた意味をもち,その経験は今後,新たな施設や実験の計画・建設に生かすことができる。%素粒子実験屋としての技量を大きく発展させる経験となっている。
また,ホスト機関職員としてハドロン実験施設へのビーム安定供給や技術サポート,国内外の研究者の受入れに力を入れている。


COMET国際コラボレーション内では「Phase-$\alpha$データ解析評価委員長」と「運動量較正手法の考案・実装ワーキンググループ長」を兼任するなど,信頼を構築しつつ,実験全体を牽引する立場を得てきた。また,実験実現に向け,KEK素核研内とコラボレーション内双方の議論で技術的および物理サイエンス的検討を進め相互の理解を深める架け橋の役割を果たすとともに,早期実現とそこからの発展展開シナリオの構築を進めている。

\subsubsection{将来計画}
MEG IIの先の実験の検討などで次世代の荷電レプトンフレーバー物理実験を主導してきた。
PSIのHIMBおよびFNALのPIP IIにおける次世代\megc 実験の検討が日・欧・米それぞれでなされた。
しかし,他で提案された案はどれも従来のパッシブコンバーターを用いたペアスペクトロメータをガンマ線検出器に想定しており,その性能評価にシミュレーション等を用いた詳細な検証が欠如していた。
応募者はシミュレーションや過去の実験の検証を通して,これまで提案されてきた案では性能を過大評価している点を指摘し\cite{jps2014a},より現実的な検出器設計を検討してきた。現在,応募者らの提案に欧州研究者が加わったものが次期\megc 実験研究グループとなっている[\href{https://arxiv.org/abs/2111.05788}{2111.05788}, \href{https://arxiv.org/abs/2504.18831}{2504.18831}]。

\newpage
\subsection{着任後の研究計画}

%s02_purpose_plan_with_abstract
\noindent
%\textbf{(本文)}
%begin 研究目的と研究計画	====================

\vspace{-2zw}
\subsubsection{学術的興味と研究対象}

%\paragraph{荷電レプトンにおけるフレーバーの破れ}
力および物質の統一を図る大統一理論は非常に魅力的な理論である。%り,「この宇宙で大統一が実現されているか」という問いに答えを出したい。
しかし,どんなに美しい,または,都合の良い理論でも自然を再現しなければ意味がない。%実験によって実証して初めて物理である。
%超高エネルギーで成立している大統一を検証しようとすると,可能な実験は限られてくる。
荷電レプトンのフレーバー混合(CLFV)は大統一に感度を有する数少ない現象でありユニークな研究対象である。加えて,新物理を探索するにはレプトンセクターが重要だと考える。カギとなるのはニュートリノ(の相方)の特殊性である。ニュートリノは,右巻きの相方が標準理論におけるゲージ一重項であり,この右巻きニュートリノが現在の宇宙の形成に重要な働きをしたと考えられる。この存在は荷電レプトンのフレーバー構造にも必ず影響を残す。レプトンフレーバーを研究することで宇宙形成の謎に迫っていく。

%荷電レプトン($\mathrm{e}, \muup, \tauup$)におけるフレーバー保存則の破れ(CLFV)は素粒子の標準模型では禁止されている。しかし,保存を保証する原理は無く,標準模型の適用範囲を超えた高いエネルギー領域で成立している新物理においては,フレーバー保存を破る相互作用が自然と存在することが期待される。したがって,CLFVを研究することで新物理の検証が可能であり,いまだ発見されていないCLFV過程の発見は新物理の確固たる証拠となる。
%電子陽電子コライダーLEPにおける超対称大統一理論の示唆,そして,ニュートリノ振動の発見は観測可能なCLFVを示唆するため,90年代末からCLFVの実験検証の重要性が認識され始めた。
%これに応えるように,国際共同実験MEGが2008--2016年に実施され,世界最高感度でCLFVを探索した。発見には至らず,従来の理論に対し厳しい制限を与える結果となった。その一方,
%さらに,
%ヒッグス粒子が超対称大統一理論の予想する質量領域で発見されたことやニュートリノ混合角$\theta_{13}$が大きかったことからCLFV探索はますます重要性を増している。%現在,COMET,Mu2e,Mu3e,そしてMEG IIと複数の実験が計画され,数年内に実験が開始されるという状況にある。

%一方,LHC実験により,超対称粒子に代表される新粒子がこれまで期待されてきた質量スケール(sub TeV)には存在せず,より重たいことが判明してきているため,CLFV過程は抑制され,現行の実験では感度が及ばない可能性もある。その場合,超対称大統一理論を徹底的に検証するには,超対称粒子に対して数十TeVまで感度を持たせた新しい実験が必要となる。

%%本研究は現行実験の物理感度を飛躍的(数十倍)改善する新世代CLFV実験を可能にする実験技術を開発することで,
%このようにCLFV実験は学術的に非常に重要で,今後の素粒子物理研究のメインストリームとなりえる,必ず遂行すべきプログラムである。MEG IIに続きCOMET実験を成功させ,自らの手で「\emph{力の大統一は実現されているか}」,「\emph{現在までに分かっている電弱スケールと大統一スケールを結びつける物理法則は何か}」という問いに答えを出したい。
%%さらにその先の将来計画につながる研究をすすめたい。本職では,この目的を直接的に果たす研究ができる。それが本職を志望する動機である。
%%現行CLFV実験の物理感度を数十倍改善する次世代実験を今準備し,現行実験と間髪をあけずに実施することが重要である。

\subsubsection{レプトンフレーバー物理の展開 --- 本邦での展開とグローバルネットワーク構築}
このようにCLFV実験は学術的に重要で,今後の素粒子物理研究のメインストリームとなりえる,必ず遂行すべきプログラムである。
人材育成・技術継承だけでなく分野の活性化のためには,\emph{走っている実験を常時維持していく}ことが最重要だと考えているが,
一方で,技術の先鋭化や装置の高度化で実験準備期間の長期化と予算規模の増大が問題化している。
この問題の解決には,国際協力が必須であると同時に,国際ネットワーク構築のためには本邦での展開が本質的に重要である。
そのためには国内プロジェクトに大学が積極的に関わっていく必要がある。大学・研究所の役割分担はあるが,人材育成や人的協力,相互理解と相乗効果のために研究所と大学が研究を一緒に進めることが大事であることを実感した。そこで,私が着任した場合にはCOMET実験をJ-PARCセンターと密に協力しながら実現させ,日・欧・米のグローバルネットワークを構築し,東京科学大を国内の中心拠点の一つとする。J-PARC・PSI・FNAL(10年後には中国の施設も加わってくる可能性も高い)での技術の集結と人的協力で次世代の実験の実現可能性を高める。
%図\ref{fig:schedule}に着任後10年間の年次計画を示す。
以下,着任後10年間の具体的な研究計画を記述する。

%\begin{figure}[b]
%\centering
%    \includegraphics[width=0.75\columnwidth]{schedule.pdf}
%    \caption{年次計画}
%    \label{fig:schedule}
%\end{figure}
\subsubsection{MEG II実験}
データ収集を2026年に完了させる。最終結果を出すまで責任をもって実験を完遂する。
これまでの経験をいかして発見感度を最大限に高める解析手法の開発を進める。具体的には,陽電子対消滅事象に起因する背景事象の同定と光子の到来方向再構成の研究である。
これらの研究を東京大学院生とともに進め,2028年にMEG II最終結果を公表・出版する。この結果はその後10年間はレプトンフレーバーに関する最重要論文となることは確実で,発見・未発見の結果に関わらず素粒子物理学に大きな影響を与える。さらに,その研究内容を次世代実験の検出器設計に生かす。

\megc\ 崩壊探索以外にも,MEG IIデータを用いて可能な探索はある。イタリアの学生が担当しているAxion-Like Particle探索ではコラボレーション内評価委員を務めており,解析の妥当性の確認とより良い解析への助言を行い,最初の結果を2026年度に出版する。

\subsubsection{COMET実験}
生成標的から検出器まで一連の超伝導ソレノイドで繋ぎ二次ビームを輸送するという画期的なアイデアがCOMETの肝である。%が,アイデア誕生から36年の歳月を経ても未だ実験実現には至っていない。%露・米での計画を経て,現在,日・米で準備が進められている。
米国の計画であるMu2eに対して,COMETはすでに
要求を満たす一次陽子ビームを実現していることやビーム試運転を経験していることなどで準備段階として先行している。さらに,大阪大学において
原理実証にも成功し,\emph{実現まであと一歩のところまで来ている}。
%原理実証と実際の$\mueconv$実験の決定的な違いはビームパワーであり,強力なビーム照射から発生する大量の放射線の取り扱いである。
応募者は具体的な施設設計と放射線計算を繰り返すことで,実験開始までの残された課題として「ビーム室内から実験室や外部に届く放射線量」を浮き彫りにした。%さらに,ビームパワーを上げるには「\emph{超伝導コイルに対する放射線による熱負荷・放射線損傷}」および「\emph{ビーム室内の物質の放射化}」の課題を解決する必要がある。
この問題を着任後1,2年で解決し\emph{実験を2028年に開始する}。

%\paragraph{ビームライン・施設建設}
%実験開始に向けてビームラインや実験エリアの建設が急務である。これは本職でしか成し遂げられない重要な仕事である。
%実験建設では各グループとの連携が大切である。そのためには,実験全体を把握することとコミュニケーションが大事になってくる。参入後数か月で,共同研究者と積極的にコミュニケーションを取り情報を得るとともに信頼関係を構築する。また,外部の業者や行政との相互理解にも尽力したい。
新しい施設における実験を立ち上げる際には様々な問題が生じる。%実際にやってみて初めてわかる問題も多々ある。
現場でどれだけ早急に最適なソリューションを提供するかが成果に大きく関わってくる。
そこでは忍耐強さと,何としても計画を進める強い推進力が必要である。ここに私の経験と特性を活かすことができると思っている。
%高品質なパルスミュービームが実験成功のカギを握る。
%1次陽子加速器の運転スキームから,2次パイオンの生成・捕獲,高品質3次ミュービームまで各担当者と協力して作り上げていきたい。
%実験中はビームの強度・安定性・extinction・純度を常時計測し,安定したデータ収集を保障する。
ビームに起因する背景事象を実測し,時間分布などを分析することでその起源や組成を調べたり,必要であれば新たな測定や検出器の導入などを速やかに実施する。コミッショニングと測定器・背景事象の理解に2,3年程度はかかると予想している。
%Phase Iで得たデータと経験を元に,Phase II設計にフィードバックをかける。

%大強度化の進むPhase II用ビームラインでは放射線耐性やメインテナンスのしやすい(もしくはいらない)設計が持続可能な施設として重要となる。
%これらの研究領域に対して経験はないが,シミュレーション
%を活用した設計などは検出器の設計の経験が生かせるし,経験者から教えてもらいながら,また綿密な議論をしながら一からシステムを組んでいくという作業はこれまでの実験くみ上げと同じであり一つ一つ経験を積みながら成し遂げたい。
%
%J-PARCという世界に誇る施設を拡張して新たな実験エリアを作り上げることはCOMET実験を超えた意味をもち,その経験は今後,新たな施設や実験の計画・建設などに活かすことができるだろう。素粒子実験屋としての技量を大きく発展させることになると期待している。

%\paragraph{物理解析}
%実験における物理解析を主導していきたい。
%$\muup^-\mathrm{N}\to\mathrm{e^-N}$過程探索のみならず,物理アウトプットを最大化する指揮を取りたい。Phase IIでは探索できないモードもあるため,Phase Iでしっかりと物理結果を出版していくことが重要である。
現在の上限値の更新に集中して実験の最適化と解析を進める。
Mu2eに先行して結果を出していくことが重要である一方,競合する実験の研究者と徹底的に議論し,双方の解析や実験理解を共有しながらより良いセットアップに改善していく。また,高度化に向けた開発を協同で進める。結果を出版するには系統誤差の理解や様々なデータを用いた検証・確認など,実験準備やR\&Dとは全くことなる綿密な解析が不可欠であることをMEG/MEG IIの経験より学んできた。これまでCOMETを推進してきた研究者とは違った視点で,必ずコラボレーションをより良い方向へと船頭できると自負している。%複数の物理解析を主導してきた経験を活かしやり遂げたい。
東京科学大としてはCOMETに新規参入となるが実験開始までに研究のイニシアチブ取ることは十分可能である。それにより,COMETを希望する学生に実験の根幹となる研究に携わってもらい,次世代の研究者を育成したい。

$\mueconv$は$\megc$と異なり,レプトンとクォークからなる系での反応を見るため,$\megc$探索では見つけられない相互作用にも幅広く感度を持つ。そのため,MEG IIの結果とは独立な新物理の探索を進めつつ,他実験の結果や計画も考慮して次世代の実験の方向性を見極める。
COMET実験を開始することで,応募者らが世界に先駆け切り開いた\emph{レプトンフレーバー物理を,今後10年間以上にわたり本邦で展開し},引き続きこの分野の研究を牽引していく。
また,さまざまな学術的・産業的利用で注目を集めているミューオン施設の発展に寄与するとともに,ミューオン衝突型加速器やミューオン触媒核融合の研究への応用発展も期待できる。
%\paragraph{}
%
%ホスト機関の研究者として国際協力を牽引し,同時にMu2eとの国際競争にも負けないタフな研究者と自らなるとともに,この絶好の機会を活かして若手研究者を育成したい。そのためには,上記項目だけにとどまらず,実験全体を統括する役割(テクニカルコーディネーション,ランコーディネーション,解析コーディネーションなど)を果たしたい。これらの経験を活かしてPhase II実験を指揮する指導的研究者となる。

%
%\subsubsection{ハドロン施設拡張計画}
%ハドロングループの一員としてハドロン施設拡張計画にも積極的に参加したい。
%優れた実験施設を何十年にもわたって活用し,そこから創出される物理成果を維持していくには,施設の改善・拡張を段階的に行っていくことが不可欠であり,またそうすることで
%当初の目的を達成するのみならず施設の価値を何倍にも高めることができる。
%さまざまなレベルに置ける物質の構造究明を可能にする世界からも注目されている実験施設を実現し,物理学の発展に寄与したい。
%


\subsubsection{次世代実験}
現在検討を進めている次期\megc\ 実験計画(MEG3eとする)の概念設計は応募者が中心となり2020年頃より考案した設計がベースになっている。
MEG IIを超えた感度を実現するには東京大学が中心となって進めているアクティブコンバージョンスペクトロメータの開発\cite{acps_2025}が最重要課題であるので協力して開発を進める。
とくにトラッキング層の開発と,統合した試作機の開発,コンバージョンスペクトロメータとしての再構成手法の確立を東京科学大で担当して進め,5年程度で試作機による開発・性能実証を行い,その後最初の段階の実験準備を進め,\emph{10年以内に次の実験を開始する}。

一方,陽電子の測定は高計数率測定が重要な鍵となる。
レート耐性を持った測定器の開発と,高計数率下での解析技術の発展が不可欠となる。
最初のセットアップとしてドリフトチェンバーを採用する場合は,MEG IIでの経験を生かしたトラッキング解析技術の開発を進める。特に,深層学習によるアルゴリズムを効率よく適用する解析ファームウェアの準備研究をアルゴリズム自身の開発と並行して進める。%MEG3eに向けて研究チームを形成し系統的に開発を進める。
他方,その後のアップグレードでシリコン検出器を採用する場合は,アクティブターゲットの実現が重要であると考えている。技術的難易度は高くより長期的な開発を,Mu3eチームと協力して進めていきたい。

次のマイルストーンは国際コラボレーションの形成とそのマネージメントである。MEG・COMETの経験をふまえて能動的・機能的な研究体制を確立し,東京科学大が中核を担うコラボレーションを構築する。
%
%
%\subsubsection{コミュニティへの貢献}
%%私はこれまでの研究の大半を実験現場であるスイス・PSI研究所に滞在して行ってきた。そこでは様々な側面でホスト機関および現地研究者のサポートを受けてはじめて研究を進めることができた。ホスト機関の重要性を身をもって感じている。本職では,上記のホスト研究者でしかできない研究・役割を全うすることに加え,国内外からの共同研究者の受け入れに力をいれ,実験全体がスムーズに進められることに尽力することで,還元していきたい。
%
%引き続き,学生の指導と研究者の育成にも力を入れたい。上記の実験は最適な教育の場である。実験が大型化・長期化している現在の素粒子実験研究において,実験の準備から物理結果導出まで一通り経験できる機会は稀有となっている。そのような機会を与えることこそが若手育成の最大の教材である。貴重な経験を通して「生きた」技術を会得させていきたい。
%自国でまた母国語で研究できるという貴重な機会を大事にしつつ,そこで閉じこもることなく,国際的な発信力とリーダーシップを持った研究者を育成していきたい。
%並行して,次世代の実験を可能とする新しい施設の検討をコミュニティをリードして進めていきたい。
%%より長い目で見れば,現行実験と将来実験の研究を並行して進めることで,物理成果を継続的に出しながら将来計画へとつなげていかねばならない。これは,(自分も含めた)人材育成に不可欠であると考える。
%%どのようにこれを実現していくか,自身の推進するプロジェクトの域を超えて,広く考え議論することで日本のそして国際的な素粒子実験コミュニティに貢献していきたい。
%%総研大や共同研究の学生のみならず

\vspace{1zw}
東京科学大における素粒子物理学の研究範囲を広げるだけではなく,でこれまで繰り広げてきたコライダー物理・ニュートリノ物理との物理的・技術的なシナジーにより新物理探索を深化させる。
学生に広い選択肢と視野を与え,統合的に新物理の仕組みをとらえていくことに貢献したい。
\newpage


\section{着任後の抱負}

%s02_purpose_plan_with_abstract
\noindent
%\textbf{(本文)}
%begin 研究目的と研究計画	====================
\vspace{-2zw}
\subsection{学術的興味と研究対象}

%%ガンマ線測定
%\paragraph{ガンマ線測定}
%ガンマ線測定は素粒子実験における根幹技術であるが,電気的に中性で透過率が高いため,その測定精度が多くの実験の感度を律速している。
%%エネルギー帯により異なる技術が用いられるが,
%$\mathcal{O}(10~\mathrm{MeV})$のガンマ線の検出・測定には従来よりカロリメータまたはペアスペクトロメータが使われてきた。
%カロリメータは結晶シンチレーターなどの有感物質中で電磁シャワーを起こさせ,そのエネルギーを測定する。高い検出効率が可能であるが,エネルギー分解能はより低いエネルギー領域で用いられる半導体検出器などに比べると劣る。また,ガンマ線の到来方向や反応位置に対する測定精度も悪い。
%一方,ペアスペクトロメータは重い物質による薄いコンバージョン層とそれに続く飛跡検出層で構成され,入射ガンマ線の対生成反応からの$\ee$対を磁気スペクトロメータで測定する。カロリメータに比べて,エネルギー・方向・反応位置の精密測定が可能であるが,一般に検出効率は桁違いに低くなる。
%検出効率を稼ぐためにコンバージョン層を厚くするとエネルギー分解能が悪くなるというトレードオフの関係が従来のペアスペクトロメータの限界といえる。
%
%ペアスペクトロメータは素粒子・原子核・天文などの分野において幅広く応用され科学技術の発展に寄与してきた。従来の限界をこえた高効率・高分解能ペアスペクトロメータの実現はガンマ線測定技術のブレークスルーとなり新たな実験を可能にする。本研究では特に,次世代の荷電レプトンフレーバー非保存事象探索実験への応用を目指す。

%\paragraph{荷電レプトンにおけるフレーバーの破れ}
力および物質の統一を図る大統一理論は非常に魅力的な理論である。%り,「この宇宙で大統一が実現されているか」という問いに答えを出したい。
しかし,どんなに美しい,または,都合の良い理論でも自然を再現しなければ意味がない。実験によって実証して初めて物理である。
超高エネルギーで成立している大統一を検証しようとすると,可能な実験は限られてくる。荷電レプトンのフレーバー混合(CLFV)は大統一に感度を有する数少ない現象でありユニークな研究対象である。また,新物理を探索するにはレプトンセクターが重要だと考える。カギとなるのはニュートリノ(の相方)の特殊性である。ニュートリノは,右巻きの相方が標準理論におけるゲージ一重項であり,この右巻きニュートリノが現在の宇宙の形成に重要な働きをしたと考えられる。この存在は荷電レプトンのフレーバー構造にも必ず影響を残す。CLFVを研究することで宇宙形成の謎に迫ることができる。

%荷電レプトン($\mathrm{e}, \muup, \tauup$)におけるフレーバー保存則の破れ(CLFV)は素粒子の標準模型では禁止されている。しかし,保存を保証する原理は無く,標準模型の適用範囲を超えた高いエネルギー領域で成立している新物理においては,フレーバー保存を破る相互作用が自然と存在することが期待される。したがって,CLFVを研究することで新物理の検証が可能であり,いまだ発見されていないCLFV過程の発見は新物理の確固たる証拠となる。
%電子陽電子コライダーLEPにおける超対称大統一理論の示唆,そして,ニュートリノ振動の発見は観測可能なCLFVを示唆するため,90年代末からCLFVの実験検証の重要性が認識され始めた。
%これに応えるように,国際共同実験MEGが2008--2016年に実施され,世界最高感度でCLFVを探索した。発見には至らず,従来の理論に対し厳しい制限を与える結果となった。その一方,
%さらに,
%ヒッグス粒子が超対称大統一理論の予想する質量領域で発見されたことやニュートリノ混合角$\theta_{13}$が大きかったことからCLFV探索はますます重要性を増している。%現在,COMET,Mu2e,Mu3e,そしてMEG IIと複数の実験が計画され,数年内に実験が開始されるという状況にある。

%一方,LHC実験により,超対称粒子に代表される新粒子がこれまで期待されてきた質量スケール(sub TeV)には存在せず,より重たいことが判明してきているため,CLFV過程は抑制され,現行の実験では感度が及ばない可能性もある。その場合,超対称大統一理論を徹底的に検証するには,超対称粒子に対して数十TeVまで感度を持たせた新しい実験が必要となる。

%本研究は現行実験の物理感度を飛躍的(数十倍)改善する新世代CLFV実験を可能にする実験技術を開発することで,
このようにCLFV実験は学術的に非常に重要で,今後の素粒子物理研究のメインストリームとなりえる,必ず遂行すべきプログラムである。MEG IIに続きCOMET実験を成功させ,自らの手で「\emph{力の大統一は実現されているか}」,「\emph{現在までに分かっている電弱スケールと大統一スケールを結びつける物理法則は何か}」という問いに答えを出したい。
%さらにその先の将来計画につながる研究をすすめたい。本職では,この目的を直接的に果たす研究ができる。それが本職を志望する動機である。
%現行CLFV実験の物理感度を数十倍改善する次世代実験を今準備し,現行実験と間髪をあけずに実施することが重要である。

\subsection{COMET実験}
COMET実験はCLFV過程の一つである$\muup$--e転換過程を探索する実験で,超対称大統一理論など多くの新物理モデルを検証できる実験である。そのPhase I実験は現在のリミットを2桁更新する計画であり,またさらに2桁高い感度を目指すPhase II実験へ向けた重要なステップでもある。
物理的には,MEG IIを超えた感度で超対称大統一理論を検証できるPhase IIまで実現させることが重要となる。そのためにも,\emph{Phase Iの早期開始と成功が現在の最重要課題}である。
%これまでにCOMETコラボレーションにより,Phase I実現にあと一歩のところまで準備が進められてきた。
私が着任した場合は,新規にこのプロジェクトに加わり,\emph{Phase Iの計画通りの実現に向けた新たな原動力}となりたい。そのために必要なことには何にでも挑戦していく所存である。
また,その結果と経験からPhase IIの10年以内の実現に向けて尽力したい。
MEG・MEG II実験で実験遂行・物理解析をこの手で行ってきた実績から,それを成し遂げる能力は十分あると自負している。

\paragraph{ビームライン・施設建設}
実験開始に向けてビームラインや実験エリアの建設が急務である。これは本職でしか成し遂げられない重要な仕事である。
%実験建設では各グループとの連携が大切である。そのためには,実験全体を把握することとコミュニケーションが大事になってくる。参入後数か月で,共同研究者と積極的にコミュニケーションを取り情報を得るとともに信頼関係を構築する。また,外部の業者や行政との相互理解にも尽力したい。
新しい施設を立ち上げる際には様々な問題が生じる。実際にやってみて初めてわかる問題も多々ある。現場でどれだけ早急に最適なソリューションを提供するかが成果に大きく関わってくる。
そこでは忍耐強さと,何としても計画を進める強い推進力が必要である。ここに私の経験と特性を活かすことができると思っている。


高品質なパルスミュービームが実験成功のカギを握る。
1次陽子加速器の運転スキームから,2次パイオンの生成・捕獲,高品質3次ミュービームまで各担当者と協力して作り上げていきたい。
実験中はビームの強度・安定性・extinction・純度を常時計測し,安定したデータ収集を保障する。
データを解析し,ビームに起因する背景事象の見積もりと統計量の計算を行う。
背景事象のパルス時刻に対する時間分布などを分析することで,その起源や組成を調べたり,必要であればより詳細な解析を可能とする新たな測定や検出器の導入などを速やかに実施する。
Phase Iで得たデータと経験を元に,Phase II設計にフィードバックをかける。

大強度化の進むPhase II用ビームラインでは放射線耐性やメインテナンスのしやすい(もしくはいらない)設計が持続可能な施設として重要となる。
これらの研究領域に対して経験はないが,シミュレーション
を活用した設計などは検出器の設計の経験が生かせるし,経験者から教えてもらいながら,また綿密な議論をしながら一からシステムを組んでいくという作業はこれまでの実験くみ上げと同じであり一つ一つ経験を積みながら成し遂げたい。

J-PARCという世界に誇る施設を拡張して新たな実験エリアを作り上げることはCOMET実験を超えた意味をもち,その経験は今後,新たな施設や実験の計画・建設などに活かすことができるだろう。素粒子実験屋としての技量を大きく発展させることになると期待している。

\paragraph{物理解析}
施設建設・運営にとどまらず,実験における物理解析を主導していきたい。
$\muup^-\mathrm{N}\to\mathrm{e^-N}$過程探索のみならず,物理アウトプットを最大化する指揮を取りたい。Phase IIでは探索できないモードもあるため,Phase Iでしっかりと物理結果をパブリッシュしていくことが重要である。Mu2eに先行して結果を出していくことが重要である一方,競合する実験の研究者と徹底的に議論し,双方の解析や実験理解を共有しながらより良い計画にブラッシュアップしていく。結果をパブリッシュするには系統誤差の理解や様々なデータを用いた検証・確認など,実験準備やR\&Dとは全くことなる綿密な解析が不可欠であることをMEG/MEG IIの経験より学んできた。これまでCOMETを推進してきた研究者とは違った視点でデータや解析を見ることができ,必ずコラボレーションをより良い方向へと船頭できると自負している。複数の物理解析を主導してきた経験を活かしやり遂げたい。


%\paragraph{}
%
%ホスト機関の研究者として国際協力を牽引し,同時にMu2eとの国際競争にも負けないタフな研究者と自らなるとともに,この絶好の機会を活かして若手研究者を育成したい。そのためには,上記項目だけにとどまらず,実験全体を統括する役割(テクニカルコーディネーション,ランコーディネーション,解析コーディネーションなど)を果たしたい。これらの経験を活かしてPhase II実験を指揮する指導的研究者となる。


\subsection{ハドロン施設拡張計画}
ハドロングループの一員としてハドロン施設拡張計画にも積極的に参加したい。
優れた実験施設を何十年にもわたって活用し,そこから創出される物理成果を維持していくには,施設の改善・拡張を段階的に行っていくことが不可欠であり,またそうすることで
当初の目的を達成するのみならず施設の価値を何倍にも高めることができる。
さまざまなレベルに置ける物質の構造究明を可能にする世界からも注目されている実験施設を実現し,物理学の発展に寄与したい。



\subsection{コミュニティへの貢献}
私はこれまでの研究の大半を実験現場であるスイス・PSI研究所に滞在して行ってきた。そこでは様々な側面でホスト機関および現地研究者のサポートを受けてはじめて研究を進めることができた。ホスト機関の重要性を身をもって感じている。本職では,上記のホスト研究者でしかできない研究・役割を全うすることに加え,国内外からの共同研究者の受け入れに力をいれ,実験全体がスムーズに進められることに尽力することで,還元していきたい。

引き続き,学生の指導と研究者の育成にも力を入れたい。COMET実験は最適な教育の場でもありうる。実験が大型化・長期化している現在の素粒子実験研究において,実験の準備から物理結果導出まで一通り経験できる機会は稀有となっている。貴重な経験を通して「生きた」技術を会得させていきたい。
自国でまた母国語で研究できるという貴重な機会を大事にしつつ,そこで閉じこもることなく,国際的な発信力とリーダーシップを持った研究者を育成していきたい。

%より長い目で見れば,現行実験と将来実験の研究を並行して進めることで,物理成果を継続的に出しながら将来計画へとつなげていかねばならない。これは,(自分も含めた)人材育成に不可欠であると考える。
%どのようにこれを実現していくか,自身の推進するプロジェクトの域を超えて,広く考え議論することで日本のそして国際的な素粒子実験コミュニティに貢献していきたい。
%総研大や共同研究の学生のみならず


%
%\renewcommand{\refname}{}
%	\vspace{2zw}
%	\begin{thebibliography}{99}
%		\setlength{\itemsep}{-1pt}
%		\bibitem{matsuoka_RPC-PMT} 松岡広大,「高時間分解能ガス電子増幅型光検出器RPC-PMT用光電面の開発」 新学術領域研究 19H05099.
%		\bibitem{gasPMT}K. Matsumoto \textit{et al.}, ``Ion-feedback suppression for gaseous photomultipliers with micro pattern gas detectors,'' Phys.\ Procedia \textbf{37} (2012) 499--505 
%		\bibitem{oya}A. Oya, ``Development of ultra-low material RPC for background identification in MEG II experiment,''  3rd Int.\ Conf.\ on Charged Lepton Flavor Violation, 2019
%		\bibitem{ILCMRPC} Z. Liu \textit{et\ al.}, ``Multigap Resistive Plate Chamber read out by $1\times1~\mathrm{cm^2}$ pads with the NINO ASIC,'' Nucl.\ Instrum.\ Methods A \textbf{920} (2019) 115--118
%	\end{thebibliography}
%end 研究目的と研究計画	====================

%\input{pieces/p01_purpose_plan_01}
%
%%#Split: 02_background  
%%#PieceName: p02_background
%% p02_background_00.tex
\KLBeginSubject{02}{2}{2 本研究の着想に至った経緯など}{1}{F}{}{jsps-subject-header}{jsps-default-header}

%\section{2 本研究の着想に至った経緯など}
%%    <<最大 1ページ>>
%\vspace{-1zw}
%%s03_background
%%begin 本研究の着想に至った経緯など ====================
%\subsection{着想に至った経緯と準備状況}
%		本研究グループはこれまでMEG・MEG II実験においてガンマ線を測定する液体キセノンガンマ線検出器を開発してきた。液体キセノンシンチレーションカロリメータは,高い一様性・放射線耐性・検出器のスケーラビリティなどの点で従来の結晶シンチレーターカロリメータと一線を画し,50~MeV付近のガンマ線に対して高い性能を発揮した。さらに,検出器の改善策を徹底的に研究しMEG II実験へのアップグレードに成功した。
%		%キセノンの発する真空紫外光に感度を持ったSiPMを開発することでアップグレードに成功したが,
%		その一方で,これ以上の大幅性能向上は難しいことも分かってきた。
%		実際,アップグレードにより一様性などの改善は達成したが本質的なエネルギー分解能は850~keVから変わっていない。$\mathrm{e^+}$の分解能が300 keVから80 keVへと改善するのと対照的であり,同じエネルギーに対して分解能は10倍も悪い。ガンマ線測定の難しさを物語っている。
%		次世代の\megc 実験には異なる手法によるガンマ線測定が不可欠であるとの認識に至った。
%		
%		MEG II実験では現在利用可能なミュー粒子ビームの強度を最大限利用するため,さらなる高感度化には新しいミュー粒子源が必要となる。2012年よりPSI研究所において,強度を最大2桁増強する新しいミュー粒子ビームラインの建設計画(HIMB計画)が進められてきた。計画では2025年にビームラインが完成する見込みである。
%		そこで,本研究グループではこのHIMBビームラインを活用した新しい実験の構想に2014年より着手し,検討を進めてきた。前実験(MEGA実験)で採用されたペアスペクトロメータの検討も進めたが,パッシブなコンバーターを用いた設計ではどうしても検出効率とエネルギー分解能の両立という従来からの課題を解決できない。アクティブコンバーターのアイデアは検討当初より持っていたが,ここ2年でMEG II実験用の超軽量・高計数率DLC RPCの開発が進み実用化の目途が見えてきたことと,松岡が開発を進めているガスPMの技術の応用を松岡と議論することで,ACPSの開発という本申請に至った。
%		ACPSは挑戦的な設計であるが,各要素のベースとなる技術は本研究グループおよび協力研究者のこれまでの研究で得られている。
%		
%\subsection{関連する国内外の研究動向と本研究の位置づけ}
%	上記HIMB計画以外にも,米フェルミ研究所の次期加速器計画PIP IIにおいて大強度DCミュー粒子源が検討されており,次世代\megc 実験の検討は日・欧・米で進められている(例えば\cite{quest})。現在進行中の米素粒子物理将来計画策定プロセスでも次世代\megc 実験案が議論されているが,提案されている案はどれも従来のパッシブコンバーターを用いたペアスペクトロメータをガンマ線検出器に想定している。%その性能見積もりは過大評価されており,明らかにシミュレーション等を用いた詳細な検証が欠如していると言わざるを得ない。
%	本研究グループではシミュレーションや過去のMEGA実験の詳細な検証を通して,これまで提案されてきた案ではシミュレーションにおいて重要なプロセスが無視されていることにより性能を過大評価している点を指摘し\cite{JPS},より現実的な検出器設計を検討してきた。本研究でACPSが開発されれば他の案とは一線を画す実験設計が出来上がる。
%	
%	MEG II実験は2024年頃にデータ収集を終了する。その時期に合わせるように本研究で新しい実験技術を確立することで,次世代の実験へとスムーズにつなげていくことが可能となる。
%	より長期スパンで計画されているCOMET/Mu2eやMu3e実験より一桁以上高い物理感度の実験をこれらの実験と同じタイムスケールでおこなうことにより,本研究グループが世界をリードしてきたCLFV実験物理領域をこれからも最先端でリードしていく。本研究はその基盤となるものである。
%	
%	
%	
%\renewcommand{\refname}{}
%	\vspace{0cm}
%	\begin{thebibliography}{99}
%		\setlength{\itemsep}{-1pt}
%		\bibitem{quest} G. Cavoto \textit{et\ al.}, ``The quests for $\meg$ and its experimental limiting factors at future high intensity muon beams,'' Eur.\ Phys.\ J.\ C \textbf{78} (2018) 37 
%		\bibitem{JPS} \underline{\underline{内山雄祐}},\underline{家城佳},\underline{岩本敏幸} 他,「崩壊分岐比感度$10^{-15}$の新しい\megc 探索実験の検討」,日本物理学会2014年秋季大会
%	\end{thebibliography}
%%end 本研究の着想に至った経緯など ====================



\section{受賞,表彰,学会活動,社会貢献の実績一覧}
\subsection{受賞・表彰}
\begin{enumerate}
\setcounter{enumi}{66}
%  \setlength{\parskip}{0cm} % 段落間
%  \setlength{\itemsep}{0cm} % 項目間
	\setlength{\itemsep}{-1pt}
\bibitem{clfv2013} Best poster award in the 1st workshop on the Charged Lepton Flavor Violation (2013) %\cite{rmd1}
\bibitem{jps} 第5回 日本物理学会若手奨励賞 (素粒子実験領域) (2011)
\bibitem{jahep} 第12回 高エネルギー物理学奨励賞 (2010)
\bibitem{pisa2009} 第11回ピサ会議 NIM-A Young Scientist Award (2009) %\cite{gamma_rec}
\end{enumerate}

\subsection{学会活動,社会貢献}
\begin{enumerate}
\setcounter{enumi}{70}
%  \setlength{\parskip}{0cm} % 段落間
%  \setlength{\itemsep}{0cm} % 項目間
	\setlength{\itemsep}{-1pt}

\bibitem{tsukuba} KEKつくばキャンパスの将来計画所内検討委員会委員 (2025--)
\bibitem{summer} KEKサマーチャレンジ実行委員 (2025--)
\bibitem{deeme} J-PARC MLF S型課題 (DeeMe) 審査委員 (2023--)
\end{enumerate}

\newpage

\section{参考意見をうかがえる人}

\begin{table}[hbtp]
  \centering
  \begin{tabular}{lc}
    \hline
    氏名   & {\bf \Large{森 俊則}} \\
    \hline
    \multirow{2}{*}{所属・職名} & 東京大学素粒子物理国際研究センター・特任研究員(名誉教授) \\
    & (MEG II実験共同代表)\\
    \hline
    \multirow{3}{*}{連絡先} & e-mail: mori@icepp.s.u-tokyo.ac.jp \\
    & 電話: 03-3815-8384 \\
    & 住所: 113-0033 東京都文京区本郷7-3-1
東京大学理学部1号館西棟10階\\
    \hline
\\
    \hline
    氏名   & {\bf \Large{Stefan Ritt}} \\
    \hline
    \multirow{2}{*}{所属・職名} &  Paul Scherrer Institut, Research Scientist  \\
    & (Group Head ``Muon Physics'') \\
    \hline
    \multirow{3}{*}{連絡先} & e-mail: stefan.ritt@psi.ch \\
    & 電話: +41 56 310 37 28 \\
    \hline
\\
       \hline
    氏名   & {\bf \Large{Alessandro M. Baldini}} \\
    \hline
    \multirow{2}{*}{所属・職名} &  INFN Pisa, Senior Associate  \\
    & (MEG II実験共同代表) \\
    \hline
    \multirow{3}{*}{連絡先} & e-mail: alessandro.baldini@pi.infn.it \\
    & 電話: +39 050 2214 303 \\
    \hline
 \\
    \hline
    氏名   & {\bf \Large{三原 智}} \\
    \hline
    \multirow{2}{*}{所属・職名} & 高エネルギー加速器研究機構素粒子原子核研究所・教授 \\
    & (COMET実験プロジェクトマネージャー)\\
    \hline
   \multirow{3}{*}{連絡先} & e-mail: satoshi.mihara@kek.jp \\
    & 電話: 029-864-5679 \\
    & 住所: 305-0801 茨城県つくば市大穂1-1\\
    \hline
 
    \end{tabular}
\end{table}

\section{着任可能時期}
2026年4月1日

%#Split: 99_tail
\input{pieces/hook9} % pieces
\end{document}

