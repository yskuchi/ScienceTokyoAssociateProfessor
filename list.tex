%begin 研究業績リスト ====================

\section{研究業績一覧}
	
\subsection{学術論文(査読あり) \textcolor{red}{$\bigcirc$}付き項目は主要論文, 被引用数はinspirehep調べ}

	\begin{enumerate}
			\setlength{\itemsep}{-1pt}

\bibitem{YU:shield}
        \me, Y.~Fukao, M.~Shunsuke, S.~Mihara, 
        ``Radiation Shielding System for the COMET Pion Capture Solenoid'', 
        JPS Conf.\ Proc. (in press), \textbf{ 責任著者}, 研究責任者.
\bibitem{YU:commissioning}
         Y.~Fukao, ..., \me\ \textit{et al.}, 
        ``Construction and Beam Commissioning for the COMET Experiment'', 
        JPS Conf.\ Proc. (in press).
\bibitem{YU:target}
         S.~Makimura, ..., \me\ \textit{et al.}, 
        ``Recent Upgrade on Muon Target at J-PARC'', 
        JPS Conf.\ Proc. (in press).

\bibitem{acps_2025}
R. Sakakibara, ..., \me, {\it et al.},
``Development of an active converter pair spectrometer for the future search for \megc,'' Nucl.\ Instrum.\ Methods A {\bf 1082} (2026) 170961,
\href{https://doi.org/10.1016/j.nima.2025.170961}{doi:10.1016/j.nima.2025.170961}
\bibitem{x17_2025} 
K. Afanaciev, ..., \me, {\it et al}.\ (MEG II Collaboration),
  ``Search for the X17 particle in $^{7}\textrm{Li}(\textrm{p},\textrm{e}^+ \textrm{e}^{-}) ^{8}\textrm{Be}$ processes with the MEG II detector,''
  Eur.\ Phys.\ J.\ C {\bf 85}  (2025) 763,
  \href{https://doi.org/10.1140/epjc/s10052-025-14345-0}{doi:10.1140/epjc/s10052-025-14345-0}.

\bibitem{cdch_2024}
A.~M. Baldini, ..., \me, {\it et al.},
``Performances of a new generation tracking detector: the MEG II cylindrical drift chamber,''
  Eur.\ Phys.\ J.\ C {\bf 84}  (2024) 473,
\href{https://doi.org/10.1140/epjc/s10052-024-12711-y}{doi:10.1140/epjc/s10052-024-12711-y}.

\bibitem{meg2_2024} 
\textcolor{red}{$\bigcirc$} 
K. Afanaciev, ..., \me, {\it et al}.\ (MEG II Collaboration),
  ``A search for \megc\ with the first dataset of the MEG II experiment,''
  Eur.\ Phys.\ J.\ C {\bf 84}  (2024) 216,
  \href{https://doi.org/10.1140/epjc/s10052-024-12416-2}{doi:10.1140/epjc/s10052-024-12416-2},
  被引用数109.

\bibitem{meg2detector_2024} 
\textcolor{red}{$\bigcirc$} 
K. Afanaciev, ..., \me, {\it et al}.\ (MEG II Collaboration),
  ``Operation and performance of the MEG II detector,''
  Eur.\ Phys.\ J.\ C {\bf 84}  (2024) 190,
  \href{https://doi.org/10.1140/epjc/s10052-024-12415-3}{doi:10.1140/epjc/s10052-024-12415-3},
  \textbf{協同責任著者},  Sect.\ 4.4(ドリフトチェンバー再構成), 5(タイミングカウンター), 9 (コンピューティング)執筆担当, 被引用数26.

\bibitem{ptc_2024}
T. Yonemoto, ..., \me, {\it et al.},
``The latest performance and refurbishment of the pixelated Timing Counter (pTC) in the MEG II experiment,'' Nucl.\ Instrum.\ Methods A {\bf 1068} (2024) 169720,
\href{https://doi.org/10.1016/j.nima.2024.169720}{doi:10.1016/j.nima.2024.169720},
研究責任者, 責任著者T.~Yonemoto(大学院生)の指導.

\bibitem{alignment_2024}
A. Ventrurini, ..., \me, {\it et al.},
``Alignment of the MEG II cylindrical drift chamber,'' Nucl.\ Instrum.\ Methods A {\bf 1067} (2024) 169680,
\href{https://doi.org/10.1016/j.nima.2024.169680}{doi:10.1016/j.nima.2024.169680}.

\bibitem{cdch_2023}
M.~Chiappini, ..., \me, {\it et al.},
``The Cylindrical Drift Chamber of the MEG II experiment,'' Nucl.\ Instrum.\ Methods A {\bf 1047} (2023) 167740,
\href{https://doi.org/10.1016/j.nima.2022.167740}{doi:10.1016/j.nima.2022.167740}.

\bibitem{ptc_2023}
P.~W.~Cattaneo, ..., \me, {\it et al.},
``Operational results with the pixelated Time Detector of MEG II experiment during the first year of physics data taking,'' Nucl.\ Instrum.\ Methods A {\bf 1046} (2023) 167751,
\href{https://doi.org/10.1016/j.nima.2022.167751}{doi:10.1016/j.nima.2022.167751},
研究責任者.
\bibitem{trigger_2023}
M.~Francesconi, ..., \me, {\it et al.},
``The trigger system for the MEG II experiment,'' Nucl.\ Instrum.\ Methods A {\bf 1046} (2023) 167736,
\href{https://doi.org/10.1016/j.nima.2022.167736}{doi:10.1016/j.nima.2022.167736}.

\bibitem{xec_2023}
T.~Iwamoto, ..., \me, {\it et al.},
``The liquid xenon detector for the MEG II experiment to detect 52.8 MeV $\gamma$ with large area VUV-sensitive MPPCs,'' Nucl.\ Instrum.\ Methods A {\bf 1046} (2023) 167720,
\href{https://doi.org/10.1016/j.nima.2022.167720}{doi:10.1016/j.nima.2022.167720}.


%\bibitem{himb}
%M.~Abe, ..., \me, {\it et al.},
%``Science case for the new High-Intensity Muon Beam HIMB at PSI,''
%arXiv:2111.05788, 
%%\href{https://arxiv.org/abs/2111.05788}{arXiv:2111.05788} 
%\href{https://doi.org/10.48550/arXiv.2111.0578}{doi:10.48550/arXiv.2111.0578} 
%(2021), 将来実験構想.

\bibitem{symmetry}
A.~M.~Baldini, ..., \me, {\it et al.} (MEG II Collaboration),
``The search for \megc\ with $10^{-14}$ sensitivity: the upgrade of the MEG experiment,''
Symmetry {\bf 13} (2021) 1591
\href{https://doi.org/10.3390/sym13091591}{doi:10.3390/sym13091591}, 
\textbf{協同責任著者}, 被引用数54.

\bibitem{radiation}
G.~Boca, P.~W.~Cattaneo, M.~De~Gerone, F.~Gatti, M.~Nakao, M.~Nishimura, W.~Ootani, M.~Rossella, \me, M.~Usami, K.~Yanai, ``Timing resolution of a plastic scintillator counter read out by radiation damaged SiPMs connected in series,'' Nucl.\ Instrum.\ Methods A {\bf 999} (2021) 165173,
\href{https://doi.org/10.1016/j.nima.2021.165173}{doi:10.1016/j.nima.2021.165173},
研究責任者, 責任著者M.~Usami(大学院生)の指導.

\bibitem{MEx2G}
A.~M.~Baldini, ..., \me, {\it et al}.\ (MEG Collaboration),
``Search for lepton flavour violating muon decay mediated by a new light particle in the MEG experiment,'' Eur.\ Phys.\ J.\ C \textbf{80} (2020) 858,
\href{https://doi.org/10.1140/epjc/s10052-020-8364-1}{doi:10.1140/epjc/s10052-020-8364-1},
コラボレーション内評価・編集委員,  責任著者M.~Nakao(大学院生)の指導, 被引用数27.

\bibitem{vci2019}
M.~Nishimura, ..., \me, {\it et al}., 
``Full system of positron timing counter in MEG II having time resolution below 40 ps with fast plastic scintillator readout by SiPMs,''
Nucl.\ Instrum.\ Methods A \textbf{958} (2020) 162785,
\href{https://doi.org/10.1016/j.nima.2019.162785}{doi:10.1016/j.nima.2019.162785},
研究責任者,  責任著者M.~Nishimura(大学院生)の指導, 被引用数18.

\bibitem{laser} 
  G.~Boca, P.~W.~Cattaneo, M.~De~Gerone, M.~Francesconi, L.~Galli, F.~Gatti, J.~Koga,
  M.~Nakao, M.~Nishimura, W.~Ootani, M.~Rossella, \me, M.~Usami, K.~Yanai, K.~Yoshida,
  ``The laser-based time calibration system for the MEG II pixelated Timing Counter,''
  Nucl.\ Instrum.\ Methods A {\bf 947}  (2019) 162672,
  \href{https://doi.org/10.1016/j.nima.2019.162672}{doi:10.1016/j.nima.2019.162672},
  研究責任者,  責任著者M.~Nakao(大学院生)の指導.
	 
\bibitem{Baldini:2018nnn}  
A.\,M.~Baldini, ..., \me, {\it et al}.\ (MEG II Collaboration),
  ``The design of the MEG II experiment,''
  Eur.\ Phys.\ J.\ C {\bf 78}  (2018) 380,
  \href{https://doi.org/10.1140/epjc/s10052-018-5845-6}{doi:10.1140/epjc/s10052-018-5845-6},
  \textbf{協同責任著者}, Sect.\ 1.2(物理動機), 5(タイミングカウンター)執筆担当, 被引用数461.

\bibitem{vci2016} 
  \me, G.~Boca, P.~W.~Cattaneo, M.~De~Gerone F.~Gatti, M.~Nakao, M.~Nishimura, W.~Ootani, G.~Pizzigoni, M.~Rossella, M.~Simonetta, K.~Yoshida, 
  ``30-ps time resolution with segmented scintillation counter for MEG II,''
  Nucl.\ Instrum.\ Methods A {\bf 845} (2017) 507--510,
  \href{https://doi.org/10.1016/j.nima.2016.06.072}{doi:10.1016/j.nima.2016.06.072},
  \textbf{責任著者}, 研究責任者.

\bibitem{btf2014} 
  P.~W.~Cattaneo, M.~De~Gerone F.~Gatti, M.~Nishimura, W.~Ootani, M.~Rossella, S.~Shirabe, \me, 
  ``Time resolution of time-of-flight detector based on multiple scintillation counters readout by SiPMs,''
  Nucl.\ Instrum.\ Meth.\ A {\bf 828} (2016) 191--200,
  \href{http://dx.doi.org/10.1016/j.nima.2016.05.038}{doi:10.1016/j.nima.2016.05.038},
  \textbf{責任著者}, 研究責任者.

\bibitem{TheMEG:2016wtm} 
\textcolor{red}{$\bigcirc$} 
  A.\,M.~Baldini, ..., \me, {\it et al}.\ (MEG Collaboration),
  ``Search for the lepton flavour violating decay $\megc$ with the full dataset of the MEG experiment,''
  Eur.\ Phys.\ J.\ C {\bf 76} (2016) 434,
  \href{https://doi.org/10.1140/epjc/s10052-016-4271-x}{doi:10.1140/epjc/s10052-016-4271-x},
 Sect.\ 3.1(ガンマ線再構成), 4.4(崩壊探索背景事象), 4.6(規格化)執筆担当, 被引用数1229.

\bibitem{polarization}
  A.~M.~Baldini, ..., \me, {\it et al}.\ (MEG Collaboration),
  ``Muon polarization in the MEG experiment: predictions and measurements,''
  Eur.\ Phys.\ J.\ C {\bf 76} (2016) 223,
  \href{http://dx.doi.org/10.1140/epjc/s10052-016-4047-3}{doi:10.1140/epjc/s10052-016-4047-3}, 被引用数35.

\bibitem{Adam:2013gfn} 
\textcolor{red}{$\bigcirc$}
  A.~M.~Baldini, ..., \me, {\it et al}.\ (MEG Collaboration),
  ``Measurement of the radiative decay of polarized muons in the MEG experiment,''
  Eur.\ Phys.\ J.\ C {\bf 76} (2016) 108,
  \href{https://doi.org/10.1140/epjc/s10052-016-3947-6}{doi:10.1140/epjc/s10052-016-3947-6},
\textbf{責任著者}, 被引用数35.

\bibitem{Ootani:2015cia} 
  W.~Ootani, ..., \me,  {\it et al}., %K.~Ieki, T.~Iwamoto, D.~Kaneko, T.~Mori, S.~Nakaura, M.~Nishimura, S.~Ogawa, R.~Sawada, N.~Shibata, \me, K.~Yoshida, K.~Sato, R.~Yamada,
  ``Development of deep-UV sensitive MPPC for liquid xenon scintillation detector,''
  Nucl.\ Instrum.\ Methods A {\bf 787} (2015) 220--223,
  \href{https://doi.org/10.1016/j.nima.2014.12.007}{doi:10.1016/j.nima.2014.12.007}, 被引用数20.

  \bibitem{tc-single} 
  \textcolor{red}{$\bigcirc$} 
  P.\ W.\ Cattaneo, M.\ De Gerone F.~Gatti, M.~Nishimura, W.~Ootani, M.~Rossella, \me, 
  ``Development of high precision timing counter based on plastic scintillator with SiPM readout,''
  IEEE Trans.\ Nucl.\ Sci.\ {\bf 61} (2014) 2657--2666,
  \href{http://dx.doi.org/10.1109/TNS.2014.2347576}{doi:10.1109/TNS.2014.2347576},
  \textbf{責任著者}, 研究責任者, 被引用数67.

\bibitem{rmd1}
  J.~Adam, ..., \me, {\it et al.},
  ``Measurement of inner Bremsstrahlung in polarized muon decay with MEG,''
  Nucl.\ Phys.\ B Proc.\ Suppl.\ \textbf{248--250} (2014) 108--111,
  \href{https://doi.org/10.1016/j.nuclphysbps.2014.02.019}{doi:10.1016/j.nuclphysbps.2014.02.019},
  \textbf{責任著者}.

\bibitem{Adam:2013vqa} 
  J.~Adam, ..., \me,  {\it et al.},
  ``The MEG detector for \megc\ decay search,''
  Eur.\ Phys.\ J.\ C {\bf 73} (2013) 2365,
  \href{https://doi.org/10.1140/epjc/s10052-013-2365-2}{doi:10.1140/epjc/s10052-013-2365-2}, 被引用数148.

\bibitem{Adam:2013mnn} 
  J.~Adam, ..., \me,  {\it et al}.\ (MEG Collaboration),
  ``New constraint on the existence of the \megc\ decay,''
  Phys.\ Rev.\ Lett.\  {\bf 110} (2013) 201801,
  \href{https://doi.org/10.1103/PhysRevLett.110.201801}{doi:10.1103/PhysRevLett.110.201801}, 被引用数712.

%\bibitem{upgrade_proposal}
% A.~M.~Baldini, ..., \me, {\it et al}.\ (MEG Collaboration),
% ``MEG upgrade proposal,'' Research proposal to PSI R-99-05.2 (2012),
% \href{https://doi.org/10.48550/arXiv.1301.7225}{doi:10.48550/arXiv.1301.7225},
% 被引用数346, Sect.\ VI-B(タイミングカウンター), XIII-D(シリコンバーテックス検出器)執筆担当.

\bibitem{TC2}
M.~De~Gerone, ..., \me, {\it et al}. %, S.~Dussoni, K.~Fratini, F.~Gatti, R.~Valle, G.~Boca, P.~W.~Cattaneo, R.~Nardo, 
%M.~Rossella, L.~Galli, M.~Grassi, D.~Nicolo, \me, D.~Zanello, 
``Development and commissioning of the Timing Counter for the MEG experiment,''
IEEE Trans.\ Nucl.\ Sci.\ \textbf{59} (2012) 379--388,
\href{http://dx.doi.org/10.1109/TNS.2012.2187311}{doi:10.1109/TNS.2012.2187311}.

\bibitem{meg2011}
J.~Adam, ..., \me, ,  {\it et al}.\ (MEG Collaboration),
``New limit on the lepton-flavor-violating decay \megc,''
Phys.\ Rev.\ Lett.\ \textbf{107} (2011) 171801,
\href{http://dx.doi.org/10.1103/PhysRevLett.107.171801}{doi:10.1103/PhysRevLett.107.171801}, 被引用数333.

\bibitem{cw}
J.~Adam, ..., \me, {\it et al}.\ (MEG Collaboration),
``Calibration and monitoring of the MEG experiment by a proton beam from a Cockcroft--Walton accelerator,''
Nucl.\ Instrum.\ Methods A \textbf{641} (2011) 19--32,
\href{https://doi.org/10.1016/j.nima.2011.03.048}{doi:10.1016/j.nima.2011.03.048}, 被引用数39.

\bibitem{TC1}
M.~De~Gerone, ..., \me, {\it et al}., %S.~Dussoni, K.~Fratini, F.~Gatti, R.~Valle, G.~Boca, P.~W.~Cattaneo, 
%M.~Rossella, R. Nardo, A.~Papa, G.~Signorelli, G.~Cavoto, G.~Pirreda, F.~Renga, C.~Voena, \me, 
``The MEG timing counter calibration and performance,''
Nucl.\ Instrum.\ Methods A \textbf{638} (2011) 41--46,
\href{https://doi.org/10.1016/j.nima.2011.02.044}{doi: 10.1016/j.nima.2011.02.044}.

\bibitem{meg2010}
J.~Adam, ..., \me, ,  {\it et al}.\ (MEG Collaboration),
``A limit for the $\meg$ decay from MEG experiment,''
Nucl.\ Phys.\ B \textbf{834} (2010) 1--12,
\href{https://doi.org/10.1016/j.nuclphysb.2010.03.030}{doi:10.1016/j.nuclphysb.2010.03.030},
博士論文内容相当, 被引用数126.

\bibitem{gamma_rec}
\me, ``Gamma ray reconstruction with liquid xenon gamma ray calorimeter for the MEG experiment,''
Nucl.\ Instrum.\ Methods A \textbf{617} (2010) 118--119,
\href{https://doi.org/10.1016/j.nima.2009.09.100}{doi:10.1016/j.nima.2009.09.100},
\textbf{責任著者}.

\bibitem{LXeStorage}
T.~Iwamoto, ..., \me,  {\it et al}.,  %, R.~Sawada, T.~Haruyama, S.~Mihara, T.~Doke, Y.~Hisamatsu, K.~Kasami, A.~Maki, T.~Mori, H.~Natori, H.~Nishiguchi, Y.~Nishimura, W.~Ootani, K.~Terasawa, \me, S.~Yamada,
``Development of a large volume zero boil-off liquid xenon storage system for muon rare decay experiment (MEG),''
Cryogenics \textbf{49} (2009) 254--258,\\
\href{https://doi.org/10.1016/j.cryogenics.2008.09.003}{doi:10.1016/j.cryogenics.2008.09.003}.

\bibitem{LXePump}
S.~Mihara, ..., \me,  {\it et al}., %T.~Haruyama, T.~Iwamoto, \me, W.~Ootani, K.~Kasami, R.~Sawada, K.~Terasawa, T.~Doke, H.~Natori, H.~Nishiguchi, A.~Maki, T.~Mori, S.~Yamada,
``Development of a method for liquid xenon purification using a cryogenic centrifugal pump,''
Cryogenics \textbf{46} (2006) 688--693,
\href{https://doi.org/10.1016/j.cryogenics.2006.04.003}{doi:10.1016/j.cryogenics.2006.04.003}, 被引用数29.

\end{enumerate}



\subsection{国際会議発表, 国内会議発表}
\begin{enumerate}
\setcounter{enumi}{38}
	\setlength{\itemsep}{-1pt}
	
	\bibitem{hql2025} 
 	 \me, ``Muon LFV/LFU measurements at J-PARC, PSI, FNAL,''
  	17th Int.\ Conf.\ on Heavy Quarks and Leptons, Beijing, Sep. 2025, \textbf{招待講演}.

	\bibitem{phits2025-2} 
  	\underline{内山雄祐},  	``J-PARC COMET実験における放射線計算,''
  	第2回 EGS5-Geant4-PHITS合同研究会, つくば,  2025年5月.
  	\bibitem{jps2025s} 
  	\underline{内山雄祐},  	``COMET実験パイオン生成システムの遮蔽設計,''
  	日本物理学会2025年春季大会, online,  2025年3月.
  	\bibitem{iac2025} 
  	\me, ``COMET status,''
 	 J-PARC International Advisory Committee, Tokai, Feb. 2025.
	\bibitem{phits2025-1} 
  	\me, ``Radiation study for the pion production system of the COMET experiment at J-PARC,''
 	 PHITS Workshop and Intermediate Course, Tokai, Feb. 2025.
  	\bibitem{fpws2024} 
  	\underline{内山雄祐},  	``レプトンフレーバー物理,''
  	Flavor Physics Workshop 2024, 蒲郡,  2024年12月, \textbf{招待講演}.
  	\bibitem{jparc2024}
  	\me, ``Radiation Shielding System for the COMET Pion Capture Solenoid,''
  	4th J-PARC Symposium, Mito, Oct. 2024.
	\bibitem{hokkaido2024} 
 	 \me, ``Search for charged lepton flavor violation with muons,''
  	Hokkaido Workshop on Particle Physics at Crossroads, Sapporo, Mar. 2024, \textbf{招待講演}.
	\bibitem{kekeph2023} 
 	 \me, ``High energy physics with muons,''
  	KEK Theory Meeting on Particle Physics Phenomenology, Tsukuba, Nov. 2023, \textbf{招待講演}.
	\bibitem{twoinfinities2023} 
 	 \me, ``Particle Physics with Muons,''
  	Int.\ Conf.\ on the Physics of the Two Infinities, Kyoto, Mar. 2023, \textbf{招待講演}.
  	\bibitem{jps2023s} 
  	\underline{内山雄祐},  	``\megc 探索実験MEG II Run 2022のまとめと今後の展望,''
  	日本物理学会2023年春季大会, online,  2025年3月.
  	\bibitem{jps2022a} 
  	\underline{内山雄祐},  	``MEG II 陽電子スペクトロメータにおける機械学習を活用したヒット再構成の改善,''
  	日本物理学会2022年秋季大会,  岡山,  2022年9月.
%  	\bibitem{jps2021a} 
%  	\underline{内山雄祐},  	``MEG II 実験:2021 年エンジニアリングランの現状と計画,''
%  	日本物理学会2021年秋季大会,  online,  2021年9月.
%  	\bibitem{jps2021s} 
%  	\underline{内山雄祐},  	``機械学習を活用した高計数率ドリフトチェンバーのヒット再構成,''
%  	日本物理学会第76回年次大会,  online,  2021年3月.
	\bibitem{aps2019} 
 	 \me, ``The MEG II experiment in serach of $\meg$,''
  	APS Division of Particles \& Fields Meeting, Boston, Jul. 2019.
  	
%  	\bibitem{jps2019a} 
%  	\underline{内山雄祐},  	``\megc 探索実験 MEG II 現状と今後の見込み,''
%  	日本物理学会2019年秋季大会,  山形,  2019年9月.

	\bibitem{icasipm2018} 
 	 \me, ``Large scale characterization of SiPMs in the MEG II experiment,''
  	Int. Conf. on the Advancement of Silicon Photomultipliers, Schwetzingen, Jun. 2018, \textbf{招待講演}.

%  	\bibitem{jps2018s} 
%  	\underline{内山雄祐},  	``\megc 探索実験 MEG II 2018年度の展望,''
%  	日本物理学会第73回年次大会,  野田,  2018年3月.
	\bibitem{vci2016} 
 	 \me, ``30-ps Time Resolution with Segmented Scintillation Counter for MEG II,''
  	14th Vienna Conf. on Instrumentation, Vienna, Feb. 2016.
	\bibitem{ckm2014} 
 	 \me, ``Lepton Flavor Violating Muon Processes,''
  	8th Int. Workshop on the CKM Unitarity Triangl, Vienna, Sep. 2014, \textbf{招待講演}.
   	\bibitem{jps2014a} 
  	\underline{内山雄祐},  	``崩壊分岐比感度$10^{-15}$の新しい\megc\ 探索実験の検討,''
  	日本物理学会2014年秋季大会,  佐賀,  2014年9月.	
	\bibitem{ieee2013} 
 	 \me, ``High Precision Measurement of Positron Time in MEG Upgrade,''
  	IEEE NSS/MIC/RTSD, Seoul, Oct. 2013.
	\bibitem{eps2013} 
 	 \me, ``Upgrade of MEG experiment,''
  	European Physical Society Conf. on High Energy Physics, Stockholm, Jul. 2013.


  	\bibitem{jps2013s} 
  	\underline{内山雄祐},  	``荷電レプトンフレーバー非保存探索による LHC 時代の素粒子物理シンポジウム: DCミューオンビームによる cLFV 探索,''
  	日本物理学会第68回年次大会,  広島,  2013年3月, \textbf{招待講演}.
  	\bibitem{jps2011a} 
  	\underline{内山雄祐},  	``Analysis of the First MEG Physics Data to Search for the Decay \megc,''
  	日本物理学会2011年秋季大会,  弘前,  2011年9月, \textbf{招待講演}.


\end{enumerate}

\subsection{総説・解説}
\begin{enumerate}
\setcounter{enumi}{59}
	\setlength{\itemsep}{-1pt}
 \bibitem{cern_courier} 
  A. Papa, F. Renga, \me,
  ``Hunting the muon's forbidden decay,''
  CERN Courier May/June (2019) 45--47
 \bibitem{highenergy} 
  家城佳, \underline{内山雄祐},
  ``MEG II 実験 ---分岐比$10^{-14}$ 台での$\megc$崩壊探索---,''
  高エネルギーニュース {\bf 37} No.\,1 (2018) 1--10
\end{enumerate}

\vspace{-8mm}
\subsection{著書}
\vspace{-3mm}
なし
\vspace{-8mm}
\subsection{特許}
\vspace{-3mm}
なし
\vspace{-4mm}

\subsection{全学術論文の被引用回数の合計, h-index}
\noindent
総被引用回数: 1,895, h-index: 16 (Scopus, 2025年9月28日付)

\newpage


%end 研究業績リスト ====================

\section{競争的研究資金および外部研究資金の獲得リスト}
\subsection{科学研究費補助金}
\begin{enumerate}
\setcounter{enumi}{61}
%  \setlength{\parskip}{0cm} % 段落間
%  \setlength{\itemsep}{0cm} % 項目間
	\setlength{\itemsep}{-1pt}

\bibitem{kibans} 基盤研究(S), 「高分解能キセノン測定器と大強度パイ中間子ビームによるレプトン普遍性破れの精密検証」, 2024年4月--2029年3月, 分担, 森俊則, 分担額:6,500千円
\bibitem{singakujutsu} 新学術領域研究(ニュートリノで拓く素粒子と宇宙)公募研究, 「極低物質量・高計数率飛跡検出器で挑む荷電レプトンフレーバーの破れの探索」, 2021年4月--2023年3月, 代表, 総額:6,370千円
\bibitem{kibana} 基盤研究(A), 「高分解能大型液体キセノン測定器によるレプトン普遍性の破れの精密検証」, 2020年4月--2024年3月, 分担, 森俊則, 分担額:4,800千円
\bibitem{wakate} 若手研究(B),  「ミュー粒子稀崩壊探索実験のさらなる高輝度化に向けたソフトウェアトリガーの開発」, 2017年4月--2019年3月, 代表, 総額:4,030千円
\bibitem{dc2} 特別研究員奨励費 「ミュー粒子稀崩壊探索実験による超対称性理論の検証」研究代表者2007年4月--2009年3月, 代表, 総額:1,800千円
\end{enumerate}

\subsection{受託研究費}
なし
\subsection{その他の競争的資金}
なし
