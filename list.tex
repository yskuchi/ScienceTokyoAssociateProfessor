%begin 研究業績リスト ====================

\section{研究業績一覧}
	
\subsection{学術論文(査読あり) \textcolor{red}{$\bigcirc$}付き項目は主要論文}

	\begin{enumerate}
			\setlength{\itemsep}{-1pt}


\bibitem{cdch_2023}
M.~Chiappini, ..., \me, {\it et al.},
``The Cylindrical Drift Chamber of the MEG II experiment,'' Nucl.\ Instrm.\ Methods A {\bf 1047} (2023) 167740,
\href{https://doi.org/10.1016/j.nima.2022.167740}{doi:10.1016/j.nima.2022.167740},
査読有.

\bibitem{ptc_2023}
P.~W.~Cattaneo, G.~Boca, M.~De~Gerone, M.~Francesconi, L.~Galli, F.~Gatti, W.~Ootani, M.~Rossella, \me, Y.~Uchiyama, M.~Usami, T.~Yonemoto,
``Operational results with the pixelated Time Detector of MEG II experiment during the first year of physics data taking,'' Nucl.\ Instrm.\ Methods A {\bf 1046} (2023) 167751,
\href{https://doi.org/10.1016/j.nima.2022.167751}{doi:10.1016/j.nima.2022.167751},
査読有, 研究責任者.
\bibitem{trigger_2023}
M.~Francesconi, ..., \me, {\it et al.},
``The trigger system for the MEG II experiment,'' Nucl.\ Instrm.\ Methods A {\bf 1046} (2023) 167736,
\href{https://doi.org/10.1016/j.nima.2022.167736}{doi:10.1016/j.nima.2022.167736},
査読有.

\bibitem{xec_2023}
T.~Iwamoto, ..., \me, {\it et al.},
``The liquid xenon detector for the MEG II experiment to detect 52.8 MeV $\gamma$ with large larea VUV-sensitive MPPCs,'' Nucl.\ Instrm.\ Methods A {\bf 1046} (2023) 167720,
\href{https://doi.org/10.1016/j.nima.2022.167720}{doi:10.1016/j.nima.2022.167720},
査読有.


\bibitem{himb}
M.~Abe, ..., \me, {\it et al.},
``Science case for the new High-Intensity Muon Beam HIMB at PSI,''
arXiv:2111.05788, 
%\href{https://arxiv.org/abs/2111.05788}{arXiv:2111.05788} 
\href{https://doi.org/10.48550/arXiv.2111.0578}{doi:10.48550/arXiv.2111.0578} 
(2021), 将来実験構想.

\bibitem{symmetry}
A.~M.~Baldini, ..., \me, {\it et al.} (MEG II Collaboration),
``The search for \megc\ with $10^{-14}$ sensitivity: the upgrade of the MEG experiment,''
Symmetry {\bf 13} (2021) 1591
\href{https://doi.org/10.3390/sym13091591}{doi:10.3390/sym13091591}, 
査読有, 共同筆頭著者.

\bibitem{radiation}
G.~Boca, P.~W.~Cattaneo, M.~De~Gerone, F.~Gatti, M.~Nakao, M.~Nishimura, W.~Ootani, M.~Rossella, \me, M.~Usami, K.~Yanai, ``Timing resolution of a plastic scintillator counter read out by radiation damaged SiPMs connected in series,'' Nucl.\ Instrum.\ Methods A {\bf 999} (2021) 165173,
\href{https://doi.org/10.1016/j.nima.2021.165173}{doi:10.1016/j.nima.2021.165173},
査読有, 研究責任者, 筆頭著者M.~Usami(大学院生)の指導.

\bibitem{MEx2G}
\textcolor{red}{$\bigcirc$} 
A.~M.~Baldini, ..., \me, {\it et al}.\ (MEG Collaboration),
``Search for lepton flavour violating muon decay mediated by a new light particle in the MEG experiment,'' Eur.\ Phys.\ J.\ C \textbf{80} (2020) 858,
\href{https://doi.org/10.1140/epjc/s10052-020-8364-1}{doi:10.1140/epjc/s10052-020-8364-1},
査読有, コラボレーション内評価・編集委員,  筆頭著者M.~Nakao(大学院生)の指導.

\bibitem{vci2019}
M.~Nishimura, ..., \me, {\it et al}., 
``Full system of positron timing counter in MEG II having time resolution below 40 ps with fast plastic scintillator readout by SiPMs,''
Nucl.\ Instrum.\ Methods A \textbf{958} (2020) 162785,
\href{https://doi.org/10.1016/j.nima.2019.162785}{doi:10.1016/j.nima.2019.162785},
査読有, 研究責任者,  筆頭著者M.~Nishimura(大学院生)の指導.

\bibitem{laser} 
  G.~Boca, P.~W.~Cattaneo, M.~De~Gerone, M.~Francesconi, L.~Galli, F.~Gatti, J.~Koga,
  M.~Nakao, M.~Nishimura, W.~Ootani, M.~Rossella, \me, M.~Usami, K.~Yanai, K.~Yoshida,
  ``The laser-based time calibration system for the MEG II pixelated Timing Counter,''
  Nucl.\ Instrum.\ Methods A {\bf 947}  (2019) 162672,
  \href{https://doi.org/10.1016/j.nima.2019.162672}{doi:10.1016/j.nima.2019.162672},
  査読有, 研究責任者,  筆頭著者M.~Nakao(大学院生)の指導.
	 
\bibitem{Baldini:2018nnn} 
\textcolor{red}{$\bigcirc$} 
A.\,M.~Baldini, ..., \me, {\it et al}.\ (MEG II Collaboration),
  ``The design of the MEG II experiment,''
  Eur.\ Phys.\ J.\ C {\bf 78}  (2018) 380,
  \href{https://doi.org/10.1140/epjc/s10052-018-5845-6}{10.1140/epjc/s10052-018-5845-6},
  査読有, 引用数265, コラボレーション内編集委員, Sect.\ 1.2(物理動機), 5(タイミングカウンター)執筆担当.

\bibitem{vci2016} 
  \me, G.~Boca, P.~W.~Cattaneo, M.~De~Gerone F.~Gatti, M.~Nakao, M.~Nishimura, W.~Ootani, G.~Pizzigoni, M.~Rossella, M.~Simonetta, K.~Yoshida, 
  ``30-ps time resolution with segmented scintillation counter for MEG II,''
  Nucl.\ Instrum.\ Methods A {\bf 845} (2017) 507--510,
  \href{https://doi.org/10.1016/j.nima.2016.06.072}{doi:10.1016/j.nima.2016.06.072},
  査読有, 筆頭著者, 研究責任者.

\bibitem{btf2014} 
  P.~W.~Cattaneo, M.~De~Gerone F.~Gatti, M.~Nishimura, W.~Ootani, M.~Rossella, S.~Shirabe, \me, 
  ``Time resolution of time-of-flight detector based on multiple scintillation counters readout by SiPMs,''
  Nucl.\ Instrum.\ Meth.\ A {\bf 828} (2016) 191--200,
  \href{http://dx.doi.org/10.1016/j.nima.2016.05.038}{doi:10.1016/j.nima.2016.05.038},
  査読有, 筆頭著者, 研究責任者.

\bibitem{TheMEG:2016wtm} 
\textcolor{red}{$\bigcirc$} 
  A.\,M.~Baldini, ..., \me, {\it et al}.\ (MEG Collaboration),
  ``Search for the lepton flavour violating decay $\megc$ with the full dataset of the MEG experiment,''
  Eur.\ Phys.\ J.\ C {\bf 76} (2016) 434,
  \href{https://doi.org/10.1140/epjc/s10052-016-4271-x}{doi:10.1140/epjc/s10052-016-4271-x}
  査読有, 引用数881, Sect.\ 3.1(ガンマ線再構成), 4.4(崩壊探索背景事象), 4.6(規格化)執筆担当.

\bibitem{polarization}
  A.~M.~Baldini, ..., \me, {\it et al}.\ (MEG Collaboration),
  ``Muon polarization in the MEG experiment: predictions and measurements,''
  Eur.\ Phys.\ J.\ C {\bf 76} (2016) 223,
  \href{http://dx.doi.org/10.1140/epjc/s10052-016-4047-3}{doi:10.1140/epjc/s10052-016-4047-3}, 査読有.

\bibitem{Adam:2013gfn} 
\textcolor{red}{$\bigcirc$}
  A.~M.~Baldini, ..., \me, {\it et al}.\ (MEG Collaboration),
  ``Measurement of the radiative decay of polarized muons in the MEG experiment,''
  Eur.\ Phys.\ J.\ C {\bf 76} (2016) 108,
  \href{https://doi.org/10.1140/epjc/s10052-016-3947-6}{doi:10.1140/epjc/s10052-016-3947-6},
査読有, 筆頭著者.

\bibitem{Ootani:2015cia} 
  W.~Ootani, ..., \me,  {\it et al}., %K.~Ieki, T.~Iwamoto, D.~Kaneko, T.~Mori, S.~Nakaura, M.~Nishimura, S.~Ogawa, R.~Sawada, N.~Shibata, \me, K.~Yoshida, K.~Sato, R.~Yamada,
  ``Development of deep-UV sensitive MPPC for liquid xenon scintillation detector,''
  Nucl.\ Instrum.\ Methods A {\bf 787} (2015) 220--223,
  \href{https://doi.org/10.1016/j.nima.2014.12.007}{doi:10.1016/j.nima.2014.12.007},
  査読有.

  \bibitem{tc-single} 
  \textcolor{red}{$\bigcirc$} 
  P.~W.~Cattaneo, M.~De~Gerone F.~Gatti, M.~Nishimura, W.~Ootani, M.~Rossella, \me, 
  ``Development of high precision timing counter based on plastic scintillator with SiPM readout,''
  IEEE Trans.\ Nucl.\ Sci.\ {\bf 61} (2014) 2657--2666,
  \href{http://dx.doi.org/10.1109/TNS.2014.2347576}{doi:10.1109/TNS.2014.2347576},
  査読有, 引用数53, 筆頭著者, 研究責任者.

\bibitem{rmd1}
  J.~Adam, ..., \me, {\it et al.},
  ``Measurement of inner Bremsstrahlung in polarized muon decay with MEG,''
  Nucl.\ Phys.\ B Proc.\ Suppl.\ \textbf{248--250} (2014) 108--111,
  \href{https://doi.org/10.1016/j.nuclphysbps.2014.02.019}{doi:10.1016/j.nuclphysbps.2014.02.019},
  査読有, 筆頭著者.

\bibitem{Adam:2013vqa} 
  J.~Adam, ..., \me,  {\it et al.},
  ``The MEG detector for \megc\ decay search,''
  Eur.\ Phys.\ J.\ C {\bf 73} (2013) 2365,
  \href{https://doi.org/10.1140/epjc/s10052-013-2365-2}{doi:10.1140/epjc/s10052-013-2365-2},
  査読有, 引用数129.

\bibitem{Adam:2013mnn} 
  J.~Adam, ..., \me,  {\it et al}.\ (MEG Collaboration),
  ``New constraint on the existence of the \megc\ decay,''
  Phys.\ Rev.\ Lett.\  {\bf 110} (2013) 201801,
  \href{https://doi.org/10.1103/PhysRevLett.110.201801}{doi:10.1103/PhysRevLett.110.201801},
  査読有, 引用数680.

\bibitem{upgrade_proposal}
 A.~M.~Baldini, ..., \me, {\it et al}.\ (MEG Collaboration),
 ``MEG upgrade proposal,'' Research proposal to PSI R-99-05.2 (2012),
 \href{https://doi.org/10.48550/arXiv.1301.7225}{doi:10.48550/arXiv.1301.7225},
 引用数346, Sect.\ VI-B(タイミングカウンター), XIII-D(シリコンバーテックス検出器)執筆担当.

\bibitem{TC2}
M.~De~Gerone, ..., \me, {\it et al}. %, S.~Dussoni, K.~Fratini, F.~Gatti, R.~Valle, G.~Boca, P.~W.~Cattaneo, R.~Nardo, 
%M.~Rossella, L.~Galli, M.~Grassi, D.~Nicolo, \me, D.~Zanello, 
``Development and commissioning of the Timing Counter for the MEG experiment,''
IEEE Trans.\ Nucl.\ Sci.\ \textbf{59} (2012) 379--388,
\href{http://dx.doi.org/10.1109/TNS.2012.2187311}{doi:10.1109/TNS.2012.2187311},
査読有.

\bibitem{meg2011}
J.~Adam, ..., \me, ,  {\it et al}.\ (MEG Collaboration),
``New limit on the lepton-flavor-violating decay \megc,''
Phys.\ Rev.\ Lett.\ \textbf{107} (2011) 171801,
\href{http://dx.doi.org/10.1103/PhysRevLett.107.171801}{doi:10.1103/PhysRevLett.107.171801},
査読有, 引用数325.

\bibitem{cw}
J.~Adam, ..., \me, {\it et al}.\ (MEG Collaboration),
``Calibration and monitoring of the MEG experiment by a proton beam from a Cockcroft--Walton accelerator,''
Nucl.\ Instrum.\ Methods A \textbf{641} (2011) 19--32,
\href{https://doi.org/10.1016/j.nima.2011.03.048}{doi:10.1016/j.nima.2011.03.048},
査読有.

\bibitem{TC1}
M.~De~Gerone, ..., \me, {\it et al}., %S.~Dussoni, K.~Fratini, F.~Gatti, R.~Valle, G.~Boca, P.~W.~Cattaneo, 
%M.~Rossella, R. Nardo, A.~Papa, G.~Signorelli, G.~Cavoto, G.~Pirreda, F.~Renga, C.~Voena, \me, 
``The MEG timing counter calibration and performance,''
Nucl.\ Instrum.\ Methods A \textbf{638} (2011) 41--46,
\href{https://doi.org/10.1016/j.nima.2011.02.044}{doi: 10.1016/j.nima.2011.02.044},
査読有.

\bibitem{meg2010}
J.~Adam, ..., \me, ,  {\it et al}.\ (MEG Collaboration),
``A limit for the $\meg$ decay from MEG experiment,''
Nucl.\ Phys.\ B \textbf{834} (2010) 1--12,
\href{https://doi.org/10.1016/j.nuclphysb.2010.03.030}{doi:10.1016/j.nuclphysb.2010.03.030},
査読有, 引用数124, 博士論文内容相当.

\bibitem{gamma_rec}
\me, ``Gamma ray reconstruction with liquid xenon gamma ray calorimeter for the MEG experiment,''
Nucl.\ Instrum.\ Methods A \textbf{617} (2010) 118--119,
\href{https://doi.org/10.1016/j.nima.2009.09.100}{doi:10.1016/j.nima.2009.09.100},
査読有, 筆頭著者.

\bibitem{LXeStorage}
T.~Iwamoto, ..., \me,  {\it et al}.,  %, R.~Sawada, T.~Haruyama, S.~Mihara, T.~Doke, Y.~Hisamatsu, K.~Kasami, A.~Maki, T.~Mori, H.~Natori, H.~Nishiguchi, Y.~Nishimura, W.~Ootani, K.~Terasawa, \me, S.~Yamada,
``Development of a large volume zero boil-off liquid xenon storage system for muon rare decay experiment (MEG),''
Cryogenics \textbf{49} (2009) 254--258,\\
\href{https://doi.org/10.1016/j.cryogenics.2008.09.003}{doi:10.1016/j.cryogenics.2008.09.003},
査読有.

\bibitem{LXePump}
S.~Mihara, ..., \me,  {\it et al}., %T.~Haruyama, T.~Iwamoto, \me, W.~Ootani, K.~Kasami, R.~Sawada, K.~Terasawa, T.~Doke, H.~Natori, H.~Nishiguchi, A.~Maki, T.~Mori, S.~Yamada,
``Development of a method for liquid xenon purification using a cryogenic centrifugal pump,''
Cryogenics \textbf{46} (2006) 688--693,
\href{https://doi.org/10.1016/j.cryogenics.2006.04.003}{doi:10.1016/j.cryogenics.2006.04.003},
査読有.

\end{enumerate}



\subsection{国際会議発表、国内会議発表}
\begin{enumerate}
\setcounter{enumi}{30}
	\setlength{\itemsep}{-1pt}
\bibitem{Mori:highenergy} 
  家城佳, \underline{内山雄祐},
  ``MEG II 実験 ---分岐比$10^{-14}$ 台での$\megc$崩壊探索---,''
  高エネルギーニュース {\bf 37} No.\,1 (2018) 1--10
\end{enumerate}

\subsection{総説・解説}
\begin{enumerate}
\setcounter{enumi}{30}
	\setlength{\itemsep}{-1pt}
\bibitem{Mori:highenergy} 
  家城佳, \underline{内山雄祐},
  ``MEG II 実験 ---分岐比$10^{-14}$ 台での$\megc$崩壊探索---,''
  高エネルギーニュース {\bf 37} No.\,1 (2018) 1--10
\end{enumerate}

\subsection{著書}
なし

\subsection{特許}
なし

\subsection{全学術論文の被引用回数の合計,h-index}

\newpage


%end 研究業績リスト ====================

\section{競争的研究資金および外部研究資金の獲得リスト}
\subsection{科学研究費補助金}
\begin{itemize}
  \setlength{\parskip}{0cm} % 段落間
  \setlength{\itemsep}{0cm} % 項目間
\item 基盤研究(S)、「高分解能キセノン測定器と大強度パイ中間子ビームによるレプトン普遍性破れの精密検証」、2024年4月--2029年3月、分担、森俊則、分担額:6,500千円
\item 新学術領域研究(ニュートリノで拓く素粒子と宇宙)公募研究、「極低物質量・高計数率飛跡検出器で挑む荷電レプトンフレーバーの破れの探索」、2021年4月--2023年3月、代表、総額:6,370千円
\item 基盤研究(A)、「高分解能大型液体キセノン測定器によるレプトン普遍性の破れの精密検証」、2020年4月--2024年3月、分担、森俊則、分担額:4,800千円
\item 若手研究(B) 、「ミュー粒子稀崩壊探索実験のさらなる高輝度化に向けたソフトウェアトリガーの開発」、2017年4月--2019年3月、代表、総額:4,030千円
\item 特別研究員奨励費 「ミュー粒子稀崩壊探索実験による超対称性理論の検証」研究代表者2007年4月--2009年3月、代表、総額:1,800千円
\end{itemize}

\subsection{受託研究費}
なし
\subsection{その他の競争的資金}
なし
