% inst_general.tex
%--------------------------------------------------------------------
% For writing instructions
%--------------------------------------------------------------------
\newcommand{\KLInstWOGeneral}[1]{%
	\noindent
 ーー ※留意事項 ーーーーーーーーーーーーーーーーーーーーーーーーーーーーーーーーー\\
		#1\\
 ーーーーーーーーーーーーーーーーーーーーーーーーーーーーーーーーーーーーーーーーーー
}

\newcommand{\KLInst}[1]{%
	\noindent
	\ifthenelse{\equal{#1}{}}{%
 ーー ※留意事項 ーーーーーーーーーーーーーーーーーーーーーーーーーーーーーーーーーー\\
	}{%
 ーー ※留意事項\textcircled{1} ーーーーーーーーーーーーーーーーーーーーーーーーーーーーーーーーー\\
		#1\\
		
	\noindent
 ーー ※留意事項\textcircled{2} ーーーーーーーーーーーーーーーーーーーーーーーーーーーーーーーーー\\
	}
}

\newcommand{\GeneralInstructions}{%
  1.作成に当たっては、研究計画調書作成・記入要領を必ず確認すること。\\
  2.本文全体は11ポイント以上の大きさの文字等を使用すること。\\
  3.各頁の上部のタイトルと指示書きは動かさないこと。\\
  4. 指示書きで定められた頁数は超えないこと。なお、空白の頁が生じても削除しないこと。\\
  \textcolor{red}{5.本留意事項は、研究計画調書の作成時には削除すること。(\texttt{\textbackslash JSPSInstructions}を消す)}\\
 ーーーーーーーーーーーーーーーーーーーーーーーーーーーーーーーーーーーーーーーーーー
}

\newcommand{\PapersInstructions}{%
 ーー ※留意事項 ーーーーーーーーーーーーーーーーーーーーーーーーーーーーーーーーーー\\
1. 研究業績(論文、著書、産業財産権、招待講演等)は、網羅的に記載するのではなく、\\
 本研究計画の実行可能性を説明する上で、その根拠となる文献等の主要なものを適宜記\\
 載すること。\\
2. 研究業績の記述に当たっては、当該研究業績を同定するに十分な情報を記載すること。\\
 例として、学術論文の場合は論文名、著者名、掲載誌名、巻号や頁等、発表年(西暦)、\\
 著書の場合はその書誌情報、など。\\
3. 論文は、既に掲載されているもの又は掲載が確定しているものに限って記載すること。\\
\textcolor{red}{4. 本留意事項は、研究計画調書の作成時には削除すること。
 (\texttt{\textbackslash PapersInstructions}を消す)}\\
 ーーーーーーーーーーーーーーーーーーーーーーーーーーーーーーーーーーーーーーーーーー\\
}