
\section{着任後の抱負}

%s02_purpose_plan_with_abstract
\noindent
%\textbf{(本文)}
%begin 研究目的と研究計画	====================
\vspace{-2zw}
\subsection{学術的興味と研究対象}

%%ガンマ線測定
%\paragraph{ガンマ線測定}
%ガンマ線測定は素粒子実験における根幹技術であるが,電気的に中性で透過率が高いため,その測定精度が多くの実験の感度を律速している。
%%エネルギー帯により異なる技術が用いられるが,
%$\mathcal{O}(10~\mathrm{MeV})$のガンマ線の検出・測定には従来よりカロリメータまたはペアスペクトロメータが使われてきた。
%カロリメータは結晶シンチレーターなどの有感物質中で電磁シャワーを起こさせ,そのエネルギーを測定する。高い検出効率が可能であるが,エネルギー分解能はより低いエネルギー領域で用いられる半導体検出器などに比べると劣る。また,ガンマ線の到来方向や反応位置に対する測定精度も悪い。
%一方,ペアスペクトロメータは重い物質による薄いコンバージョン層とそれに続く飛跡検出層で構成され,入射ガンマ線の対生成反応からの$\ee$対を磁気スペクトロメータで測定する。カロリメータに比べて,エネルギー・方向・反応位置の精密測定が可能であるが,一般に検出効率は桁違いに低くなる。
%検出効率を稼ぐためにコンバージョン層を厚くするとエネルギー分解能が悪くなるというトレードオフの関係が従来のペアスペクトロメータの限界といえる。
%
%ペアスペクトロメータは素粒子・原子核・天文などの分野において幅広く応用され科学技術の発展に寄与してきた。従来の限界をこえた高効率・高分解能ペアスペクトロメータの実現はガンマ線測定技術のブレークスルーとなり新たな実験を可能にする。本研究では特に,次世代の荷電レプトンフレーバー非保存事象探索実験への応用を目指す。

%\paragraph{荷電レプトンにおけるフレーバーの破れ}
力および物質の統一を図る大統一理論は非常に魅力的な理論である。%り,「この宇宙で大統一が実現されているか」という問いに答えを出したい。
しかし,どんなに美しい,または,都合の良い理論でも自然を再現しなければ意味がない。実験によって実証して初めて物理である。
超高エネルギーで成立している大統一を検証しようとすると,可能な実験は限られてくる。荷電レプトンのフレーバー混合(CLFV)は大統一に感度を有する数少ない現象でありユニークな研究対象である。また,新物理を探索するにはレプトンセクターが重要だと考える。カギとなるのはニュートリノ(の相方)の特殊性である。ニュートリノは,右巻きの相方が標準理論におけるゲージ一重項であり,この右巻きニュートリノが現在の宇宙の形成に重要な働きをしたと考えられる。この存在は荷電レプトンのフレーバー構造にも必ず影響を残す。CLFVを研究することで宇宙形成の謎に迫ることができる。

%荷電レプトン($\mathrm{e}, \muup, \tauup$)におけるフレーバー保存則の破れ(CLFV)は素粒子の標準模型では禁止されている。しかし,保存を保証する原理は無く,標準模型の適用範囲を超えた高いエネルギー領域で成立している新物理においては,フレーバー保存を破る相互作用が自然と存在することが期待される。したがって,CLFVを研究することで新物理の検証が可能であり,いまだ発見されていないCLFV過程の発見は新物理の確固たる証拠となる。
%電子陽電子コライダーLEPにおける超対称大統一理論の示唆,そして,ニュートリノ振動の発見は観測可能なCLFVを示唆するため,90年代末からCLFVの実験検証の重要性が認識され始めた。
%これに応えるように,国際共同実験MEGが2008--2016年に実施され,世界最高感度でCLFVを探索した。発見には至らず,従来の理論に対し厳しい制限を与える結果となった。その一方,
%さらに,
%ヒッグス粒子が超対称大統一理論の予想する質量領域で発見されたことやニュートリノ混合角$\theta_{13}$が大きかったことからCLFV探索はますます重要性を増している。%現在,COMET,Mu2e,Mu3e,そしてMEG IIと複数の実験が計画され,数年内に実験が開始されるという状況にある。

%一方,LHC実験により,超対称粒子に代表される新粒子がこれまで期待されてきた質量スケール(sub TeV)には存在せず,より重たいことが判明してきているため,CLFV過程は抑制され,現行の実験では感度が及ばない可能性もある。その場合,超対称大統一理論を徹底的に検証するには,超対称粒子に対して数十TeVまで感度を持たせた新しい実験が必要となる。

%本研究は現行実験の物理感度を飛躍的(数十倍)改善する新世代CLFV実験を可能にする実験技術を開発することで,
このようにCLFV実験は学術的に非常に重要で,今後の素粒子物理研究のメインストリームとなりえる,必ず遂行すべきプログラムである。MEG IIに続きCOMET実験を成功させ,自らの手で「\emph{力の大統一は実現されているか}」,「\emph{現在までに分かっている電弱スケールと大統一スケールを結びつける物理法則は何か}」という問いに答えを出したい。
%さらにその先の将来計画につながる研究をすすめたい。本職では,この目的を直接的に果たす研究ができる。それが本職を志望する動機である。
%現行CLFV実験の物理感度を数十倍改善する次世代実験を今準備し,現行実験と間髪をあけずに実施することが重要である。

\subsection{COMET実験}
COMET実験はCLFV過程の一つである$\muup$--e転換過程を探索する実験で,超対称大統一理論など多くの新物理モデルを検証できる実験である。そのPhase I実験は現在のリミットを2桁更新する計画であり,またさらに2桁高い感度を目指すPhase II実験へ向けた重要なステップでもある。
物理的には,MEG IIを超えた感度で超対称大統一理論を検証できるPhase IIまで実現させることが重要となる。そのためにも,\emph{Phase Iの早期開始と成功が現在の最重要課題}である。
%これまでにCOMETコラボレーションにより,Phase I実現にあと一歩のところまで準備が進められてきた。
私が着任した場合は,新規にこのプロジェクトに加わり,\emph{Phase Iの計画通りの実現に向けた新たな原動力}となりたい。そのために必要なことには何にでも挑戦していく所存である。
また,その結果と経験からPhase IIの10年以内の実現に向けて尽力したい。
MEG・MEG II実験で実験遂行・物理解析をこの手で行ってきた実績から,それを成し遂げる能力は十分あると自負している。

\paragraph{ビームライン・施設建設}
実験開始に向けてビームラインや実験エリアの建設が急務である。これは本職でしか成し遂げられない重要な仕事である。
%実験建設では各グループとの連携が大切である。そのためには,実験全体を把握することとコミュニケーションが大事になってくる。参入後数か月で,共同研究者と積極的にコミュニケーションを取り情報を得るとともに信頼関係を構築する。また,外部の業者や行政との相互理解にも尽力したい。
新しい施設を立ち上げる際には様々な問題が生じる。実際にやってみて初めてわかる問題も多々ある。現場でどれだけ早急に最適なソリューションを提供するかが成果に大きく関わってくる。
そこでは忍耐強さと,何としても計画を進める強い推進力が必要である。ここに私の経験と特性を活かすことができると思っている。


高品質なパルスミュービームが実験成功のカギを握る。
1次陽子加速器の運転スキームから,2次パイオンの生成・捕獲,高品質3次ミュービームまで各担当者と協力して作り上げていきたい。
実験中はビームの強度・安定性・extinction・純度を常時計測し,安定したデータ収集を保障する。
データを解析し,ビームに起因する背景事象の見積もりと統計量の計算を行う。
背景事象のパルス時刻に対する時間分布などを分析することで,その起源や組成を調べたり,必要であればより詳細な解析を可能とする新たな測定や検出器の導入などを速やかに実施する。
Phase Iで得たデータと経験を元に,Phase II設計にフィードバックをかける。

大強度化の進むPhase II用ビームラインでは放射線耐性やメインテナンスのしやすい(もしくはいらない)設計が持続可能な施設として重要となる。
これらの研究領域に対して経験はないが,シミュレーション
を活用した設計などは検出器の設計の経験が生かせるし,経験者から教えてもらいながら,また綿密な議論をしながら一からシステムを組んでいくという作業はこれまでの実験くみ上げと同じであり一つ一つ経験を積みながら成し遂げたい。

J-PARCという世界に誇る施設を拡張して新たな実験エリアを作り上げることはCOMET実験を超えた意味をもち,その経験は今後,新たな施設や実験の計画・建設などに活かすことができるだろう。素粒子実験屋としての技量を大きく発展させることになると期待している。

\paragraph{物理解析}
施設建設・運営にとどまらず,実験における物理解析を主導していきたい。
$\muup^-\mathrm{N}\to\mathrm{e^-N}$過程探索のみならず,物理アウトプットを最大化する指揮を取りたい。Phase IIでは探索できないモードもあるため,Phase Iでしっかりと物理結果をパブリッシュしていくことが重要である。Mu2eに先行して結果を出していくことが重要である一方,競合する実験の研究者と徹底的に議論し,双方の解析や実験理解を共有しながらより良い計画にブラッシュアップしていく。結果をパブリッシュするには系統誤差の理解や様々なデータを用いた検証・確認など,実験準備やR\&Dとは全くことなる綿密な解析が不可欠であることをMEG/MEG IIの経験より学んできた。これまでCOMETを推進してきた研究者とは違った視点でデータや解析を見ることができ,必ずコラボレーションをより良い方向へと船頭できると自負している。複数の物理解析を主導してきた経験を活かしやり遂げたい。


%\paragraph{}
%
%ホスト機関の研究者として国際協力を牽引し,同時にMu2eとの国際競争にも負けないタフな研究者と自らなるとともに,この絶好の機会を活かして若手研究者を育成したい。そのためには,上記項目だけにとどまらず,実験全体を統括する役割(テクニカルコーディネーション,ランコーディネーション,解析コーディネーションなど)を果たしたい。これらの経験を活かしてPhase II実験を指揮する指導的研究者となる。


\subsection{ハドロン施設拡張計画}
ハドロングループの一員としてハドロン施設拡張計画にも積極的に参加したい。
優れた実験施設を何十年にもわたって活用し,そこから創出される物理成果を維持していくには,施設の改善・拡張を段階的に行っていくことが不可欠であり,またそうすることで
当初の目的を達成するのみならず施設の価値を何倍にも高めることができる。
さまざまなレベルに置ける物質の構造究明を可能にする世界からも注目されている実験施設を実現し,物理学の発展に寄与したい。



\subsection{コミュニティへの貢献}
私はこれまでの研究の大半を実験現場であるスイス・PSI研究所に滞在して行ってきた。そこでは様々な側面でホスト機関および現地研究者のサポートを受けてはじめて研究を進めることができた。ホスト機関の重要性を身をもって感じている。本職では,上記のホスト研究者でしかできない研究・役割を全うすることに加え,国内外からの共同研究者の受け入れに力をいれ,実験全体がスムーズに進められることに尽力することで,還元していきたい。

引き続き,学生の指導と研究者の育成にも力を入れたい。COMET実験は最適な教育の場でもありうる。実験が大型化・長期化している現在の素粒子実験研究において,実験の準備から物理結果導出まで一通り経験できる機会は稀有となっている。貴重な経験を通して「生きた」技術を会得させていきたい。
自国でまた母国語で研究できるという貴重な機会を大事にしつつ,そこで閉じこもることなく,国際的な発信力とリーダーシップを持った研究者を育成していきたい。

%より長い目で見れば,現行実験と将来実験の研究を並行して進めることで,物理成果を継続的に出しながら将来計画へとつなげていかねばならない。これは,(自分も含めた)人材育成に不可欠であると考える。
%どのようにこれを実現していくか,自身の推進するプロジェクトの域を超えて,広く考え議論することで日本のそして国際的な素粒子実験コミュニティに貢献していきたい。
%総研大や共同研究の学生のみならず


%
%\renewcommand{\refname}{}
%	\vspace{2zw}
%	\begin{thebibliography}{99}
%		\setlength{\itemsep}{-1pt}
%		\bibitem{matsuoka_RPC-PMT} 松岡広大,「高時間分解能ガス電子増幅型光検出器RPC-PMT用光電面の開発」 新学術領域研究 19H05099.
%		\bibitem{gasPMT}K. Matsumoto \textit{et al.}, ``Ion-feedback suppression for gaseous photomultipliers with micro pattern gas detectors,'' Phys.\ Procedia \textbf{37} (2012) 499--505 
%		\bibitem{oya}A. Oya, ``Development of ultra-low material RPC for background identification in MEG II experiment,''  3rd Int.\ Conf.\ on Charged Lepton Flavor Violation, 2019
%		\bibitem{ILCMRPC} Z. Liu \textit{et\ al.}, ``Multigap Resistive Plate Chamber read out by $1\times1~\mathrm{cm^2}$ pads with the NINO ASIC,'' Nucl.\ Instrum.\ Methods A \textbf{920} (2019) 115--118
%	\end{thebibliography}
%end 研究目的と研究計画	====================

%% p01_purpose_plan_01.tex
\KLEndSubject{F}

%
%%#Split: 02_background  
%%#PieceName: p02_background
%% p02_background_00.tex
\KLBeginSubject{02}{2}{2 本研究の着想に至った経緯など}{1}{F}{}{jsps-subject-header}{jsps-default-header}

%\section{2 本研究の着想に至った経緯など}
%%    <<最大 1ページ>>
%\vspace{-1zw}
%%s03_background
%%begin 本研究の着想に至った経緯など ====================
%\subsection{着想に至った経緯と準備状況}
%		本研究グループはこれまでMEG・MEG II実験においてガンマ線を測定する液体キセノンガンマ線検出器を開発してきた。液体キセノンシンチレーションカロリメータは,高い一様性・放射線耐性・検出器のスケーラビリティなどの点で従来の結晶シンチレーターカロリメータと一線を画し,50~MeV付近のガンマ線に対して高い性能を発揮した。さらに,検出器の改善策を徹底的に研究しMEG II実験へのアップグレードに成功した。
%		%キセノンの発する真空紫外光に感度を持ったSiPMを開発することでアップグレードに成功したが,
%		その一方で,これ以上の大幅性能向上は難しいことも分かってきた。
%		実際,アップグレードにより一様性などの改善は達成したが本質的なエネルギー分解能は850~keVから変わっていない。$\mathrm{e^+}$の分解能が300 keVから80 keVへと改善するのと対照的であり,同じエネルギーに対して分解能は10倍も悪い。ガンマ線測定の難しさを物語っている。
%		次世代の\megc 実験には異なる手法によるガンマ線測定が不可欠であるとの認識に至った。
%		
%		MEG II実験では現在利用可能なミュー粒子ビームの強度を最大限利用するため,さらなる高感度化には新しいミュー粒子源が必要となる。2012年よりPSI研究所において,強度を最大2桁増強する新しいミュー粒子ビームラインの建設計画(HIMB計画)が進められてきた。計画では2025年にビームラインが完成する見込みである。
%		そこで,本研究グループではこのHIMBビームラインを活用した新しい実験の構想に2014年より着手し,検討を進めてきた。前実験(MEGA実験)で採用されたペアスペクトロメータの検討も進めたが,パッシブなコンバーターを用いた設計ではどうしても検出効率とエネルギー分解能の両立という従来からの課題を解決できない。アクティブコンバーターのアイデアは検討当初より持っていたが,ここ2年でMEG II実験用の超軽量・高計数率DLC RPCの開発が進み実用化の目途が見えてきたことと,松岡が開発を進めているガスPMの技術の応用を松岡と議論することで,ACPSの開発という本申請に至った。
%		ACPSは挑戦的な設計であるが,各要素のベースとなる技術は本研究グループおよび協力研究者のこれまでの研究で得られている。
%		
%\subsection{関連する国内外の研究動向と本研究の位置づけ}
%	上記HIMB計画以外にも,米フェルミ研究所の次期加速器計画PIP IIにおいて大強度DCミュー粒子源が検討されており,次世代\megc 実験の検討は日・欧・米で進められている(例えば\cite{quest})。現在進行中の米素粒子物理将来計画策定プロセスでも次世代\megc 実験案が議論されているが,提案されている案はどれも従来のパッシブコンバーターを用いたペアスペクトロメータをガンマ線検出器に想定している。%その性能見積もりは過大評価されており,明らかにシミュレーション等を用いた詳細な検証が欠如していると言わざるを得ない。
%	本研究グループではシミュレーションや過去のMEGA実験の詳細な検証を通して,これまで提案されてきた案ではシミュレーションにおいて重要なプロセスが無視されていることにより性能を過大評価している点を指摘し\cite{JPS},より現実的な検出器設計を検討してきた。本研究でACPSが開発されれば他の案とは一線を画す実験設計が出来上がる。
%	
%	MEG II実験は2024年頃にデータ収集を終了する。その時期に合わせるように本研究で新しい実験技術を確立することで,次世代の実験へとスムーズにつなげていくことが可能となる。
%	より長期スパンで計画されているCOMET/Mu2eやMu3e実験より一桁以上高い物理感度の実験をこれらの実験と同じタイムスケールでおこなうことにより,本研究グループが世界をリードしてきたCLFV実験物理領域をこれからも最先端でリードしていく。本研究はその基盤となるものである。
%	
%	
%	
%\renewcommand{\refname}{}
%	\vspace{0cm}
%	\begin{thebibliography}{99}
%		\setlength{\itemsep}{-1pt}
%		\bibitem{quest} G. Cavoto \textit{et\ al.}, ``The quests for $\meg$ and its experimental limiting factors at future high intensity muon beams,'' Eur.\ Phys.\ J.\ C \textbf{78} (2018) 37 
%		\bibitem{JPS} \underline{\underline{内山雄祐}},\underline{家城佳},\underline{岩本敏幸} 他,「崩壊分岐比感度$10^{-15}$の新しい\megc 探索実験の検討」,日本物理学会2014年秋季大会
%	\end{thebibliography}
%%end 本研究の着想に至った経緯など ====================
