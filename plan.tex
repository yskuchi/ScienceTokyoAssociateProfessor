
\section{教育に関する実績および着任後の抱負}

%s02_purpose_plan_with_abstract
\noindent
%\textbf{(本文)}
%begin 研究目的と研究計画	====================
\vspace{-2zw}

\subsection{教育実績}
東京大学の大学院生指導を13年間にわたり行い,学生の教育と研究者育成にも力を注いできた。MEG実験のデータ解析やMEG IIタイミングカウンターの開発を研究した学生とは,文字通り寝食を共にしながら研究を進め,各局面における課題を解決する実践力を身に着けさせることを意識して指導をしてきた。%主体性を育むために常に学生自身に調べ・考え・手を動かすことを求めている。
これまでにMEG・MEG II物理解析で博士論文7本(うち3本日本物理学会若手奨励賞受賞),タイミングカウンターの開発で修士論文6本(うち1本測定器開発優秀修士論文賞受賞),博士論文2本を直接指導し輩出した実績をもつ。
現在KEKにおいては,総研大生および東海キャンパス分室に常駐している大阪大学の博士課程学生の指導をしている。また,COMET日本グループにおける大学間の学生交流を促進する目的でCOMET日本グループ会合を主催し,年3,4回定期的に開催し,自らも学生と積極的に交流を深め彼らがどういうことを考え求め悩んでいるか理解し,日々の研究にそれらを反映させるよう働きかけている。

東京大学大学院生を対象にした「稀崩壊探索実験に必要な統計解析」講座(全8回)やソフトウェア・解析講習会(全7回)を実施したり,PSI研究所を訪れた総研大や九州大の学生を含めた学生ゼミを主催したりした経験がある。神戸大学では博士課程大学院生を対象とした「高エネルギー物理学IIA」を山崎教授と協同で担当した。
エキスパティーズとしては,素粒子物理学・素粒子実験・粒子検出器・光検出器・遮蔽計算および設計・データ解析・統計解析・プログラミング・真空技術など。

\subsection{着任後の抱負}

東京科学大理学院物理系の教育における使命は,(1)論理的思考を身に着け,物理的視点で問題に取り組み解決する力を身に着けること。(2)物理(自然)の美しさを感じ,より本質的な物理法則を解き明かそうとする志と知識・技術を身に着けること。(3)研究者を育成すること。であると私は考える。この使命を果たすため,以下のことを意識して教育にあたりたい。

\subsubsection{学部教育}
学部教育では(1)現代科学(物理)を理解する,またそのために必要な基礎知識を習得すること,(2)最先端の研究をするための基礎を学ぶこと,(3)幅広い分野にふれ自分の興味ある研究対象を見つけること,を各学生が達成できることが目標となる。共通する大事な点は,学生が素粒子原子核物理学に興味を持つことであるので,いかに興味を与えられるかを常に意識して教育にあたりたい。

講義や演習・学生実験では,常に目的や全体像をしっかり理解してもらってから詳細に取り掛かるように心がけたい。途中で興味を失ったり,ついていけなくなる場合の多くは,目的や全体像を見失ってしまった時である。式や現象の意味するところをかみ砕いて説明する工夫をする。理解することが興味をもつ最短コースである。

ゼミや卒業研究では,研究最前線を,きらびやかに盛ることなく,具体的にイメージできるような情報を与えたい。また,今「勉強」していることが今後の「研究」でどう役に立つのかを示す。学部時代は意外と「研究をしていると何が楽しいか」という点に触れる機会が少なかったように思う。教員たちが楽しんで研究をしていることを実感してもらい,その一端を体感できるような内容に工夫をする。

\subsubsection{大学院教育}

大学院の教育では,各人の希望進路に関係なく,研究者となるための指導をしていく。そこで身に着ける技術や論理的思考力はどの進路においても役立つはずだ。研究をする上では自発的な取り組みが不可欠である。最初は課題を与えるが,それを自分で発展させる。
%研究課題を自分で考えてもらう際には,どうしてその課題が大事かを考えさせる。「まだ誰もやっていない」というのはそれ自体には価値がない。物理意義をしっかり考えさせる。
個性を尊重することも大事だが,総合力をつけることも重要だ。特に博士課程に進学する場合は,測定器から物理解析,論文執筆から国際チームの牽引まで,学生のうちに幅広く経験してもらう機会を与える。博士論文の質の維持も長期的に見た場合に分野にとって重要である。

研究がある程度進んでくると,自分のやっていることや分野全体,またはより具体的な技術(たとえば英語や統計手法など)を体系だって教えてほしいという状況になる。しかし,大学院での講義が本来それに相当すべきである。タイミングと学生の意識の問題がある。意識に関しては講義を能動的に(ここでいう能動は講義中になにかをしてもらうというのではなく,聴講する際の意識の問題)することが求められる。学生が実際に取り組んでいる(取り組んでいく)研究にふれ,関連性や必要性を説きながら興味を刺激する講義にしたい。タイミングに関しては公式なカリキュラムにおける講義だけでなく,ゼミや講座をこれまでやってきたように適宜開くことでそのような学生のニーズに応えていきたい。


日本における研究者育成,特に当該分野での問題点として以下を実感している:
(1)	実験の長期化と細分化による思い描いていたことと現実のギャップ。自分ではどうにもできないことで学位取得やキャリアが左右されてしまうことが多々ある。
(2)	キャリアパスへの学生の目はシビアになってきており,研究者としてやっていけるか,(経済的な面も含めて)魅力ある職業かと悩んだ末,進路を変更する。
若手に魅力を感じさせない分野は衰退していく。以下の取り組みにより,これらの問題に対処しながら,今後のこの分野を担う人材を輩出していきたい。

研究者に向いているかどうかは,その人の性格や志向・興味によるので一人ひとりに向き合って指導をしていく。物理研究者になるには,根本的な志向として,基礎サイエンスに興味があることが大事だと思う。「世の中の役に立ちたい」もしくは「世間で役に立つ技術を取得したい」という(工学的な)志向が強い場合は,アカデミア以外の進路のほうが向いているかもしれない。物理研究者を育てるということに固執せず,個々にあった進路に進めるように広い選択肢を与えることが重要である。

一方で,興味の対象や研究を面白いと感じるかどうかは,環境・外的要因にも大きく依存する。あることが分かったことで興味が増したり,ある課題に取り組むことで研究に没頭するようになることもよくある。学生には要所要所で成功体験を与えられるような研究を助言していきたい。長期的な実験は特に注意が必要だ。「検出器のパーツ試験をしていたら博士課程3年間が過ぎていた」ということが無いよう,修士と博士でテーマを変える・複数のプロジェクトに参画するなど,他の教員と連携しながら学生の興味にあった研究および博士論文を進められるようにする。ただし,おいしいところだけやらせるのでは上記の総合力は身につかない。重要な実験を成功に導く多くは泥臭い作業であり,時に計画通りには進まないこともある。そのようなときでも,常に大目標を忘れずに意識させ,そのうえで長期・中期・短期目標をしっかり設定させて研究を進めてもらえばモチベーションは維持できるはずである。

キャリアパスに関しては広い視野と人脈が大事になる。身近な先輩が活躍している姿を見れば憧れを抱き,不安も軽減するだろう。自分の課題や担当だけに籠らずに,まずは研究室内やコラボレーション内で縦横のつながりをしっかりと築くよう個々の個性を見極めて指導したい。キャリアをイメージするには広い知識が必要となる。%世界でどんな面白い研究がされているかにいかに多く触れるかが大事になる。
ゼミなどを通して広い視野を与えたい。そのうえで,国際会議や研究会に積極的に参加する機会を与えたい。そこでは国際化やコミュニケーションが重要となる。研究に関してしっかりと議論できるような姿勢とコミュニケーション(言語だけでなく文化やプレゼンの仕方など)をときには直接,ときには背中で伝えていきたい。
\newpage