
\subsection{教育に関する実績および着任後の抱負}

%s02_purpose_plan_with_abstract
\noindent
%\textbf{(本文)}
%begin 研究目的と研究計画	====================
\vspace{-2zw}

\subsection{教育実績}
東京大学の大学院生指導を13年間にわたり行い,学生の教育と研究者育成にも力を注いできた。MEG実験のデータ解析やMEG IIタイミングカウンターの開発を研究した学生とは,文字通り寝食を共にしながら研究を進め,各局面における課題を解決する実践力を身に着けさせることを意識して指導をしてきた。%主体性を育むために常に学生自身に調べ・考え・手を動かすことを求めている。
これまでにMEG実験物理解析で博士論文6本(うち3本日本物理学会若手奨励賞受賞),タイミングカウンターの開発で修士論文6本(うち1本測定器開発優秀修士論文賞受賞),博士論文2本を直接指導し輩出した実績をもつ。

大学院生を対象にした「稀崩壊探索実験に必要な統計解析」講座(全8回)や素粒子実験用ソフトウェア・解析講習会(全7回)を実施したり,夏にPSI研究所を訪れた総研大や九州大の学生を含めた学生ゼミを主催したりした経験がある。エキスパティーズとしては,素粒子物理学・素粒子実験・粒子検出器・光検出器・データ解析・統計解析・プログラミング・真空技術など。

\newpage